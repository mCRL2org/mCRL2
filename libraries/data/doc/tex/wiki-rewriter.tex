
\documentclass{article}
%%%%%%%%%%%%%%%%%%%%%%%%%%%%%%%%%%%%%%%%%%%%%%%%%%%%%%%%%%%%%%%%%%%%%%%%%%%%%%%%%%%%%%%%%%%%%%%%%%%%%%%%%%%%%%%%%%%%%%%%%%%%%%%%%%%%%%%%%%%%%%%%%%%%%%%%%%%%%%%%%%%%%%%%%%%%%%%%%%%%%%%%%%%%%%%%%%%%%%%%%%%%%%%%%%%%%%%%%%%%%%%%%%%%%%%%%%%%%%%%%%%%%%%%%%%%
\usepackage{amssymb}
\usepackage{amsfonts}
\usepackage{amsmath}
\usepackage{a4wide}
\usepackage[british]{babel}

\setcounter{MaxMatrixCols}{10}
%TCIDATA{OutputFilter=LATEX.DLL}
%TCIDATA{Version=5.50.0.2890}
%TCIDATA{<META NAME="SaveForMode" CONTENT="1">}
%TCIDATA{BibliographyScheme=Manual}
%TCIDATA{Created=Monday, October 19, 2009 15:45:27}
%TCIDATA{LastRevised=Thursday, October 29, 2009 15:00:37}
%TCIDATA{<META NAME="GraphicsSave" CONTENT="32">}
%TCIDATA{<META NAME="DocumentShell" CONTENT="Standard LaTeX\Standard LaTeX Article">}
%TCIDATA{CSTFile=40 LaTeX article.cst}

\newtheorem{theorem}{Theorem}
\newtheorem{acknowledgement}[theorem]{Acknowledgement}
\newtheorem{algorithm}[theorem]{Algorithm}
\newtheorem{axiom}[theorem]{Axiom}
\newtheorem{case}[theorem]{Case}
\newtheorem{claim}[theorem]{Claim}
\newtheorem{conclusion}[theorem]{Conclusion}
\newtheorem{condition}[theorem]{Condition}
\newtheorem{conjecture}[theorem]{Conjecture}
\newtheorem{corollary}[theorem]{Corollary}
\newtheorem{criterion}[theorem]{Criterion}
\newtheorem{definition}[theorem]{Definition}
\newtheorem{example}[theorem]{Example}
\newtheorem{exercise}[theorem]{Exercise}
\newtheorem{lemma}[theorem]{Lemma}
\newtheorem{notation}[theorem]{Notation}
\newtheorem{problem}[theorem]{Problem}
\newtheorem{proposition}[theorem]{Proposition}
\newtheorem{remark}[theorem]{Remark}
\newtheorem{solution}[theorem]{Solution}
\newtheorem{summary}[theorem]{Summary}
\newenvironment{proof}[1][Proof]{\noindent\textbf{#1.} }{\ \rule{0.5em}{0.5em}}
\input{tcilatex}
\begin{document}

\title{Rewriter Implementation Notes}
\author{Wieger Wesselink}
\maketitle

\section{Introduction}

This document describes rewrite algorithms that can be applied in the mCRL2
tool set. Currently only a prototype implementation in python is available.
Most of the content is based on \cite{weerdenburg2009}.

\subsection{Simple terms}

Simple terms are terms with the following syntax:%
\begin{equation}
t:=x\ |\ f\ |\ f(t,\cdots ,t),  \label{eq:simple_terms}
\end{equation}%
where $t$ is a term, $x$ is a variable and $f$ is a function symbol.

\subsection{Flattened terms}

Flattened terms are an extension of simple terms:%
\begin{equation}
t:=x\ |\ f\ |\ f(t,\cdots ,t)\ |\ x(t,\cdots ,t).  \label{eq:flattened_terms}
\end{equation}%
N.B. The notion of flattened terms doesn't seem very useful. The only reason
we introduce it here is that several algorithms in \cite{weerdenburg2009}
are only defined for flattened terms.

\subsection{Applicative terms}

Applicative terms are an extension of flattened terms:%
\begin{equation}
t:=x\ |\ f\ |\ t(t,\cdots ,t).  \label{eq:applicative_terms}
\end{equation}

The set of all variables is denoted by $\mathbb{V}$, the set of all function
symbols by $\mathbb{F}$ and the set of all terms by $\mathbb{T}$. In this
document we use the convention that $x,y\in \mathbb{V}$, that $t,u\in 
\mathbb{T}$, and that $f,g\in \mathbb{F}$.

We write $var(t)$ for the set of variables that occur in $t$. Formally:%
\begin{equation*}
\begin{array}{lll}
var(x) & = & \{x\} \\ 
var(f) & = & \emptyset \\ 
var(t(t_{1},\cdots ,t_{n}) & = & var(t)\cup \dbigcup\limits_{i=1\cdots
n}var(t_{i}).%
\end{array}%
\end{equation*}

\subsection{Subterms}

To facilitate operations on subterms we inductively define positions ($%
\mathbb{P}$) as follows. A position is either $\epsilon $ (the empty
position) or an index $i$ (from 1,2,$\cdots $) combined with a position $\pi 
$, notation $i\cdot \pi $. We lift $\cdot $ to an associative operator on
positions with $\epsilon $ as its unit element and often write just $i$ for
the position $i\cdot \epsilon $. We write the subterm of $t$ at position $%
\pi $ as $t|_{\pi }$ and we write term $t$ with the subterm at position $\pi 
$ replaced by $u$ as $t[u]_{\pi }$. These operations are defined as follows.%
\begin{equation*}
\begin{array}{llll}
t|_{\epsilon } & = & t &  \\ 
t(t_{1},\cdots ,t_{n})|_{i\cdot \pi } & = & t_{i}|_{\pi } & \text{if }1\leq
i\leq n \\ 
t[u]_{\epsilon } & = & u &  \\ 
x[u]_{i\cdot \pi } & = & x &  \\ 
f(t_{1},\cdots ,t_{n})[u]_{i\cdot \pi } & = & f(t_{1},\cdots
,t_{i-1},t_{i}[u]_{\pi },t_{i+1,}\cdots ,t_{n}) & \text{if }i\leq n \\ 
f(t_{1},\cdots ,t_{n})[u]_{i\cdot \pi } & = & f(t_{1},\cdots ,t_{n}) & \text{%
if }i>n%
\end{array}%
\end{equation*}

Some examples are:%
\begin{equation*}
\begin{array}{l}
f(x,g(y))|_{1}=x \\ 
f(x,g(y))|_{2\cdot 1}=y \\ 
f(x,g(y))[h(x)]_{2}=f(x,h(x))%
\end{array}%
\end{equation*}

\subsection{Substitutions}

A substitution is a function $\sigma :\mathbb{V\rightarrow T}$. A
substitution $\sigma $ can also be applied to a term $t$. This is denoted by 
$t\sigma $ and it is defined as%
\begin{equation*}
\begin{array}{lll}
x\sigma & = & \sigma (x) \\ 
f\sigma & = & f \\ 
t(t_{1},\cdots ,t_{n})\sigma & = & t\sigma (t_{1}\sigma ,\cdots ,t_{n}\sigma
).%
\end{array}%
\end{equation*}

\subsection{Rewrite rules}

A rewrite rule is a rule $l\rightarrow r\ \mathbf{if}\ c$, with $l,r,c\in 
\mathbb{T}$. We put three restrictions on rewrite rules:%
\begin{equation*}
\begin{array}{ll}
1) & l\text{ is a simple term} \\ 
2) & l\notin \mathbb{V} \\ 
3) & var(r)\cup var(c)\subseteq var(l)%
\end{array}%
\end{equation*}

For a set $R$ of rewrite rules we define the rewrite relation $\rightarrow
_{R}$ as follows: $t\rightarrow _{R}u$ if there is a rule $l\rightarrow r\ 
\mathbf{if}\ c$ in $R$ $,$ a position $\pi $ and a substitution $\sigma $
such that%
\begin{equation}
t|_{\pi }=l\sigma \wedge u=t[r\sigma ]_{\pi }\wedge \eta (c\sigma ),
\label{eq:rewriting}
\end{equation}%
where $\eta $ is a boolean function that determines if a condition is true.
We write $\rightarrow $ instead of $\rightarrow _{R}$ if no confusion can
occur. We write $\rightarrow _{R}^{\ast }$ for the reflexive and transitive
closure of $\rightarrow _{R}$ and $t\nrightarrow _{R}$ if there is no $u$
such that $t\rightarrow _{R}u$. A normal form is a term $u$ such that $%
t\rightarrow _{R}^{\ast }u$ and $u\nrightarrow _{R}$.

\subsection{Flatten operator}

Let $\rightarrow _{flatten}$ be the rewrite relation given by%
\begin{equation*}
\begin{array}{c}
t(t_{1},\cdots ,t_{m})(t_{m+1},\cdots ,t_{n})\rightarrow
_{flatten}t(t_{1},\cdots ,t_{n}).%
\end{array}%
\end{equation*}%
We define the operator $flatten$ such that $flatten(t)$ is the (unique)
normal form of $t$ corresponding to $\rightarrow _{flatten}^{\ast }$.

\subsection{Rewrite algorithm}

We now fomulate an abstract rewrite algorithm $rewrite$, where we assume
that $R$ is a given, fixed set of rewrite rules.%
\begin{equation*}
\begin{array}{l}
\text{\textbf{function} }rewrite(t) \\ 
u:=t \\ 
\text{\textbf{while} }\{v\ |\ u\rightarrow _{R}v\}\neq \emptyset \text{ 
\textbf{do}} \\ 
\qquad \text{\textbf{choose} }v\text{ \textbf{such that} }u\rightarrow _{R}v
\\ 
\qquad u:=v \\ 
\text{\textbf{return} }u%
\end{array}%
\end{equation*}%
Note that this algorithm does not need to terminate. In practice we are also
interested in an algorithm $rewrite(t,\sigma )$, that applies a substitution 
$\sigma $ to the variables in $t$ during rewriting. The specification of
this algorithm is simply%
\begin{equation*}
rewrite(t,\sigma )=rewrite(t\sigma ).
\end{equation*}%
The reason we are interested in such an algorithm is that it can be
implemented more efficiently than the straightforward solution to first
compute $u=t\sigma $ and then compute $rewrite(u)$.

\section{Match trees}

A match tree is a tree structure that represents a number of rewrite rules
that have left hand sides with the same function symbol as head. It is used
to compute all possible results of applying one of these rules to a term.
Currently match trees are only defined for flattened terms. A match tree
consists of nodes of the following types:

\begin{itemize}
\item $F(f,T,U):$ If the current term has the form $f(t_{1},\cdots ,t_{n})$
replace the top of the stack by $t_{1}\rhd \cdots \rhd t_{n}$ and continue
with $T$, otherwise continue with $U$.

\item $S(x,T):$ Assign the current term to variable $x$ and continue with $T$%
.

\item $M(x,T,U):$ If the current term is equal to $x$ continue with $T$,
otherwise continue with $U$.

\item $R(Q):$ Return $Q$.

\item $X:$ Return the empty set.

\item $N(n,T):$ Remove $n$ elements from the stack and continue with $T$. We
abbreviate $N(1,T)$ as $N(T)$.

\item $E(T,U):$ If the stack is not empty continue with $T$, otherwise
continue with $U$.

\item $C(t,T,U):$ If $t$ evaluates to $true$, continue with $T$, otherwise
continue with $U$.
\end{itemize}

where $f$ is a function symbol, $x$ is a variable, $t$ is a term, $Q$ is a
set of terms annotated with a rewrite rule, and $T$ and $U$ are match tree
nodes. Match trees are only defined for flattened terms.

\subsection{Evaluating a match tree}

Let $l$ be a sequence of terms, and let $\sigma $ be an arbitrary
substitution function. Then the evaluation of a match tree with arguments $l$
and $\sigma $ is a set of terms and is defined as follows:%
\begin{equation*}
\begin{array}{lll}
F(f,T,U)(l,\sigma ) & = & \left\{ 
\begin{array}{ll}
\emptyset  & \text{if }l=[] \\ 
T(m,\sigma ) & \text{if }l=f\rhd m \\ 
T(t_{1}\rhd \cdots \rhd t_{n}\rhd m,\sigma ) & \text{if }l=f(t_{1},\cdots
,t_{n})\rhd m \\ 
U(l,\sigma ) & \text{if }l=g(t_{1},\cdots ,t_{n})\rhd m\wedge f\neq g \\ 
U(l,\sigma ) & \text{if }l=x\rhd m \\ 
U(l,\sigma ) & \text{if }l=x(t_{1},\cdots ,t_{n})\rhd m%
\end{array}%
\right.  \\ 
X(l,\sigma ) & = & \emptyset  \\ 
R(Q)(l,\sigma ) & = & \left\{ 
\begin{array}{ll}
\{\sigma (t)\ |\ t^{\alpha }\in Q\} & \text{if }l=[] \\ 
\emptyset  & \text{if }l\neq \lbrack ]%
\end{array}%
\right.  \\ 
S(x,T) & = & \left\{ 
\begin{array}{ll}
\emptyset  & \text{if }l=[] \\ 
T(l,\sigma \lbrack x\rightarrow t]) & \text{if }l=t\rhd m%
\end{array}%
\right.  \\ 
M(x,T,U)(l,\sigma ) & = & \left\{ 
\begin{array}{ll}
\emptyset  & \text{if }l=[] \\ 
T(l,\sigma ) & \text{if }l=t\rhd m\wedge \sigma (x)=t \\ 
U(l,\sigma ) & \text{if }l=t\rhd m\wedge \sigma (x)\neq t%
\end{array}%
\right.  \\ 
N(n,T)(l,\sigma ) & = & \left\{ 
\begin{array}{ll}
\emptyset  & \text{if }|l|\ <n \\ 
T(m,\sigma ) & \text{if }l=t_{1}\rhd \cdots \rhd t_{n}\rhd m%
\end{array}%
\right.  \\ 
E(T,U)(l,\sigma ) & = & \left\{ 
\begin{array}{ll}
U(l,\sigma ) & \text{if }l=[] \\ 
T(l,\sigma ) & \text{if }l=t\rhd m%
\end{array}%
\right.  \\ 
C(t,T,U)(l,\sigma ) & = & \left\{ 
\begin{array}{ll}
T(l,\sigma ) & \text{if }t\sigma \text{ evaluates to }true \\ 
U(l,\sigma ) & \text{if }t\sigma \text{ does not evalute to }true%
\end{array}%
\right. 
\end{array}%
\end{equation*}%
where $T$ and $U$ are match trees, $f$ and $g$ are function symbols, $l$ and 
$m$ are sequences of terms and $t$ and $t_{i}$ are terms. The evaluation of
a match tree $T$ in a single term $t$ with substitution $\sigma $ is defined
as $T([t],\sigma )$.

\subsection{Building a match tree}

Let $\alpha $ be a rewrite rule given by $l\rightarrow r$.Then we define the
match tree $match\_tree(\alpha )=\gamma ([l],\{r^{\alpha }\},\emptyset )$,
where $\gamma $ is defined as:

\begin{equation*}
\begin{array}{lll}
\gamma ([],Q,V) & = & R(Q) \\ 
\gamma (x\rhd s,Q,V) & = & \left\{ 
\begin{array}{ll}
S(x,N(\gamma (s,Q,V\cup \{x\}))) & \text{if }x\notin V \\ 
M(x,N(\gamma (s,Q,V\cup \{x\})),X) & \text{if }x\in V%
\end{array}%
\right. \\ 
\gamma (f(t_{1},\cdots ,t_{n})\rhd s,Q,V) & = & F(f,\gamma (t_{1}\rhd \cdots
\rhd t_{n}\rhd s,Q,V),X)%
\end{array}%
\end{equation*}%
Match trees are only defined for rewrite rules with simple terms at the left
hand side.

\subsection{Joining match trees}

Two match trees $left$ and $right$ can be joined into one using the operator 
$||$, which is defined as follows: $left\ ||\ right=$%
\begin{equation*}
\begin{array}{lllll}
right & \text{if} & head(left)=X &  &  \\ 
left & \text{if} & head(right)=X &  &  \\ 
E(left,right) & \text{if} & head(right)=R &  &  \\ 
E(right,left) & \text{if} & head(left)=R &  &  \\ 
R(Q\cup Q^{\prime }) & \text{if} & left=R(Q) & \text{and} & 
right=R(Q^{\prime }) \\ 
S(x,T\ ||\ right) & \text{if} & left=S(x,T) & \text{and} & head(right)\in
\{F,S,U\} \\ 
M(y,left\ ||\ U,left) & \text{if} & left=S(x,T) & \text{and} & right=M(y,U,V)
\\ 
M(x,T\ ||\ right,T^{\prime }\ ||\ right) & \text{if} & left=M(x,T,T^{\prime
}) & \text{and} & head(right)\in \{F,M,N,S\} \\ 
S(x,left\ ||\ U) & \text{if} & left=F(f,T,T^{\prime }) & \text{and} & 
right=S(x,U) \\ 
M(x,left\ ||\ U,left) & \text{if} & left=F(f,T,T^{\prime }) & \text{and} & 
right=M(x,U,U^{\prime }) \\ 
F(f,T\ ||\ U,T^{\prime }) & \text{if} & left=F(f,T,T^{\prime }) & \text{and}
& right=F(f,U,U^{\prime }) \\ 
F(f,T,T^{\prime }\ ||\ right) & \text{if} & left=F(f,T,T^{\prime }) & \text{%
and} & right=F(g,U,U^{\prime }),~f\neq g \\ 
F(f,T\ ||\ N(ar(f),U),T^{\prime }\ ||\ right) & \text{if} & 
left=F(f,T,T^{\prime }) & \text{and} & right=N(U) \\ 
S(x,left\ ||\ U) & \text{if} & left=N(T) & \text{and} & right=S(x,U) \\ 
M(x,left\ ||\ U,left\ ||\ U^{\prime }) & \text{if} & left=N(T) & \text{and}
& right=M(x,U,U^{\prime }) \\ 
F(f,N(ar(f),T)\ ||\ U,left) & \text{if} & left=N(T) & \text{and} & 
right=F(f,U,X) \\ 
N(T\ ||\ U) & \text{if} & left=N(T) & \text{and} & right=N(U) \\ 
E(T,right\ ||\ T^{\prime }) & \text{if} & left=E(T,T^{\prime }) & \text{and}
& head(right)\in \{F,M,N,R,S\},%
\end{array}%
\end{equation*}%
where $head$ is defined as $head(F(f,T,U))=F$, $head(R(Q))=R$ etc.

\subsection{Optimizing match trees}

The result of joining match trees is often not optimal. This section gives
two algorithms $reduce$ and $clean$ to optimize match trees.%
\begin{equation*}
\begin{array}{l}
\begin{array}{llll}
reduce(X) & = & X &  \\ 
reduce(F(f,T,U)) & = & reduce_{F}(F(f,T,U),\emptyset ) &  \\ 
reduce(S(x,T)) & = & reduce_{S}(S(x,T),\emptyset ) &  \\ 
reduce(M(x,T,U)) & = & reduce_{M}(M(x,T,U),\emptyset ,\emptyset ) &  \\ 
reduce(C(t,T,U)) & = & C(t,reduce(T),reduce(U)) &  \\ 
reduce(N(n,T)) & = & N(n,reduce(T)) &  \\ 
reduce(E(T,U)) & = & E(t,reduce(T),reduce(U)) &  \\ 
reduce(R(Q)) & = & R(Q) &  \\ 
&  &  &  \\ 
reduce_{F}(X,F) & = & F &  \\ 
reduce_{F}(F(f,T,U),F) & = & reduce_{F}(U,F) & \text{if }f\in F \\ 
reduce_{F}(F(f,T,U)) & = & F(f,reduce(T),reduce_{F}(U,F\cup \{f\})) & \text{%
if }f\notin F \\ 
reduce_{F}(N(n,T)) & = & reduce_{M}(M(x,T,U),\emptyset ,\emptyset ) &  \\ 
&  &  &  \\ 
reduce_{S}(X,\emptyset ) & = & X &  \\ 
reduce_{S}(X,\{x\}\cup V) & = & S(x,reduce(X[x/V],\emptyset )) &  \\ 
reduce_{S}(F(f,T,U),\emptyset ) & = & reduce_{F}(F(f,T,U),\emptyset ) &  \\ 
reduce_{S}(F(F,T,U),\{x\}\cup V) & = & S(x,reduce_{F}(F(f,T,U)[x/V],%
\emptyset ) &  \\ 
reduce_{S}(S(x,T),V) & = & reduce_{S}(T,V\cup \{x\}) &  \\ 
reduce_{S}(N(n,T),\emptyset ) & = & reduce(N(n,T),\emptyset ) &  \\ 
reduce_{S}(N(n,T),\{x\}\cup V) & = & S(x,reduce(N(n,T)[x/V])) &  \\ 
&  &  &  \\ 
reduce_{M}(X,M_{t},M_{f}) & = & reduce(X) &  \\ 
reduce_{M}(F(f,T,U),M_{t},M_{f}) & = & reduce_{F}(F(f,T,U),\emptyset ) &  \\ 
reduce_{M}(S(x,T),M_{t},M_{f}) & = & reduce_{S}(S(x,T),\emptyset ) &  \\ 
reduce_{M}(M(x,T,U),M_{t},M_{f}) & = & reduce_{M}(T,M_{t},M_{f}) & \text{if }%
x\in M_{t} \\ 
reduce_{M}(M(x,T,U),M_{t},M_{f}) & = & reduce_{M}(U,M_{t},M_{f}) & \text{if }%
x\in M_{f} \\ 
reduce_{M}(M(x,T,U),M_{t},M_{f}) & = & M(x,reduce_{M}(T,M_{t}\cup
\{x\},M_{f}), & \text{if }x\notin M_{t}\wedge x\notin M_{f} \\ 
&  & reduce_{M}(U,M_{t}\cup \{x\},M_{f}\cup \{x\})) &  \\ 
reduce_{M}(N(n,T)) & = & reduce(N(n,T)), & 
\end{array}%
\end{array}%
\end{equation*}%
with%
\begin{equation*}
\begin{array}{llll}
X[x/V] & = & X &  \\ 
F(f,T,U)[x/V] & = & F(f,T[x/V],U[x/V]) &  \\ 
S(x,T)[y/V] & = & S(x,T[y/(V\setminus \{x\})]) &  \\ 
M(x,T,U)[y/V] & = & M(y,T[y/V],U[y/V]) & \text{if }x\in V \\ 
M(x,T,U)[y/V] & = & M(x,T[y/V],U[y/V]) & \text{if }x\notin V \\ 
C(t,T,U)[x/V] & = & C(t[x/y:y\in V],T[x/V],U[x/V) &  \\ 
N(n,T)[x/V] & = & N(n,T[x/V]) &  \\ 
E(T,U)[x/V] & = & E(t,T[x/V],U[x/V]) &  \\ 
R(Q)[x/V] & = & R(Q[x/y:y\in V]) & 
\end{array}%
\end{equation*}%
\begin{equation*}
\begin{array}{llll}
clean(T) & = & T^{\prime } & \text{if }\chi (T)=\left\langle T^{\prime
},V\right\rangle ,%
\end{array}%
\end{equation*}%
where $\chi $ is defined as%
\begin{equation*}
\begin{array}{llll}
\chi (X) & = & \left\langle X,\emptyset \right\rangle  &  \\ 
\chi (F(f,T,U)) & = & \left\langle F(f,T^{\prime },U^{\prime }),V\cup
W\right\rangle  & \text{if }\chi (T)=\left\langle T^{\prime },V\right\rangle
\wedge \chi (U)=\left\langle U^{\prime },W\right\rangle  \\ 
\chi (S(x,T)) & = & \left\langle S(x,T^{\prime }),V\setminus
\{x\}\right\rangle  & \text{if }\chi (T)=\left\langle T^{\prime
},V\right\rangle \wedge x\in V \\ 
\chi (S(x,T)) & = & \left\langle T^{\prime },V\right\rangle  & \text{if }%
\chi (T)=\left\langle T^{\prime },V\right\rangle \wedge x\notin V \\ 
\chi (M(x,T,U)) & = & \left\langle T^{\prime },V\right\rangle  & \text{if }%
\chi (T)=\left\langle T^{\prime },V\right\rangle \wedge \chi
(U)=\left\langle U^{\prime },W\right\rangle \wedge T^{\prime }=U^{\prime }
\\ 
\chi (M(x,T,U)) & = & \left\langle M(x,T^{\prime },U^{\prime }),V\cup W\cup
\{x\}\right\rangle  & \text{if }\chi (T)=\left\langle T^{\prime
},V\right\rangle \wedge \chi (U)=\left\langle U^{\prime },W\right\rangle
\wedge T^{\prime }\neq U^{\prime } \\ 
\chi (C(t,T,U)) & = & \left\langle T^{\prime },V\right\rangle  & \text{if }%
\chi (T)=\left\langle T^{\prime },V\right\rangle \wedge \chi
(U)=\left\langle U^{\prime },W\right\rangle \wedge T^{\prime }=U^{\prime }
\\ 
\chi (C(t,T,U)) & = & \left\langle C(t,T^{\prime },U^{\prime }),V\cup W\cup
var(t)\right\rangle  & \text{if }\chi (T)=\left\langle T^{\prime
},V\right\rangle \wedge \chi (U)=\left\langle U^{\prime },W\right\rangle
\wedge T^{\prime }\neq U^{\prime } \\ 
\chi (N(n,T)) & = & \left\langle N(n,T^{\prime }),V\right\rangle  & \text{if 
}\left\langle T^{\prime },V\right\rangle =\chi (T) \\ 
\chi (E(T,U)) & = & \left\langle T^{\prime },V\right\rangle  & \text{if }%
\chi (T)=\left\langle T^{\prime },V\right\rangle \wedge \chi
(U)=\left\langle X,W\right\rangle  \\ 
\chi (E(T,U)) & = & \left\langle U^{\prime },W\right\rangle  & \text{if }%
\chi (T)=\left\langle X,V\right\rangle \wedge \chi (U)=\left\langle
U^{\prime },W\right\rangle  \\ 
\chi (E(T,U)) & = & \left\langle E(T^{\prime },U^{\prime }),V\cup
W\right\rangle  & \text{if }\chi (T)=\left\langle T^{\prime },V\right\rangle
\wedge \chi (U)=\left\langle U^{\prime },W\right\rangle \wedge T^{\prime
}\neq X\wedge U^{\prime }\neq X \\ 
\chi (R(Q)) & = & \left\langle R(Q),var(Q)\right\rangle  & 
\end{array}%
\end{equation*}

\subsection{Prioritized rewrite rules}

By adding priorities to rewrite rules, the selection of rewrite rules that
is considered for a term can be reduced. We model priorities of rewrite
rules using a function $\varphi $, that returns the rules of highest
priority for a set of rules. So $\varphi (R)\subseteq R$ and $\varphi
(R)=\emptyset $ if and only if $R=\emptyset $. We define a function $prior$
that applies a priority function $\varphi $ to a match tree:%
\begin{equation*}
\begin{array}{lll}
prior(X,\varphi ) & = & X \\ 
prior(F(f,T,U),\varphi ) & = & F(f,prior(T,\varphi ),prior(U,\varphi )) \\ 
prior(S(x,T),\varphi ) & = & S(x,prior(T,\varphi )) \\ 
prior(M(x,T,U),\varphi ) & = & M(x,prior(T,\varphi ),prior(U,\varphi )) \\ 
prior(C(t,T,U),\varphi ) & = & C(t,prior(T,\varphi ),prior(U,\varphi )) \\ 
prior(N(n,T),\varphi ) & = & N(n,prior(T,\varphi )) \\ 
prior(R(Q),\varphi ) & = & R(\varphi (Q)) \\ 
prior(E(T,U),\varphi ) & = & E(prior(T,\varphi ),prior(U,\varphi )).%
\end{array}%
\end{equation*}%
The effect of applying $prior$ to a match tree is that the $R$-nodes will
contain less elements.  This can be useful to remove unwanted results.
Consider for example the rewrite system%
\begin{equation*}
\left\{ 
\begin{array}{ccc}
x=x & \rightarrow  & true \\ 
x=y & \rightarrow  & false%
\end{array}%
\right. 
\end{equation*}%
This system can have both $true$ and $false$ as a result of rewriting the
term $true=true$. But if we give the first equation a higher priority than
the second, the undesired derivation $true=true\rightarrow false$ is
eliminated.

\subsection{Rewriting using match trees}

Suppose that we have a rewrite system, and that for each function symbol $f$
a match tree $M_{f}$ has been constructed that corresponds to rewrite rules
with head symbol $f$. We define the function $rewr_{M}$ as:%
\begin{equation*}
\begin{array}{lll}
rewr_{M}(x,\sigma ) & = & \sigma (x) \\ 
rewr_{M}(f,\sigma ) & = & \left\{ 
\begin{array}{ll}
f & \text{if }M_{f}([f],\sigma )=\emptyset  \\ 
u\in M_{f}([f],\sigma ) & \text{if }M_{f}([f],\sigma )\neq \emptyset 
\end{array}%
\right.  \\ 
rewr_{M}(f(t_{1},\cdots ,t_{n}),\sigma ) & = & \left\{ 
\begin{array}{ll}
f(t_{1},\cdots ,t_{n}) & \text{if }M_{f}([f(t_{1},\cdots ,t_{n})],\sigma
)=\emptyset  \\ 
u\in M_{f}([f(t_{1},\cdots ,t_{n})],\sigma ) & \text{if }M_{f}([f(t_{1},%
\cdots ,t_{n})],\sigma )\neq \emptyset 
\end{array}%
\right.  \\ 
rewr_{M}(x(t_{1},\cdots ,t_{n}),\sigma ) & = & x(t_{1},\cdots ,t_{n})\sigma 
\end{array}%
\end{equation*}%
The function $rewr_{M}$ is defined for flattened terms.

\section{Rewriting Strategies}

In this section we describe rewriting strategies. For the moment we only
consider innermost rewriting.

\subsection{Innermost rewriting}

We now define an algorithm $rewr_{I}$ for innermost rewriting. It is defined
for flattened terms.%
\begin{equation*}
\begin{array}{lll}
rewr_{I}(x(t_{1},\cdots ,t_{n}),\sigma ) & = & rewr_{M}(flatten(\sigma
(x)(rewr_{I}(t_{1},\sigma ),\cdots ,rewr_{I}(t_{n},\sigma ))) \\ 
rewr_{I}(f,\sigma ) & = & rewr_{M}(f,\sigma ) \\ 
rewr_{I}(f(t_{1},\cdots ,t_{n}),\sigma ) & = & rewr_{M}(f(rewr_{I}(t_{1},%
\sigma ),\cdots ,rewr_{I}(t_{n},\sigma )) \\ 
rewr_{I}(x,\sigma ) & = & \sigma (x)%
\end{array}%
\end{equation*}%
Here we assume that $\sigma (x)$ is in normal form already.

A possible extension to applicative terms might look as follows:%
\begin{equation*}
\begin{array}{lll}
rewr_{I}(x(t_{1},\cdots ,t_{n}),\sigma ) & = & rewr_{M}(flatten(\sigma
(x)(rewr_{I}(t_{1},\sigma ),\cdots ,rewr_{I}(t_{n},\sigma ))) \\ 
rewr_{I}(f,\sigma ) & = & rewr_{M}(f,\sigma ) \\ 
rewr_{I}(t(t_{1},\cdots ,t_{n}),\sigma ) & = & rewr_{M}(rewr_{I}(t,\sigma
)(rewr_{I}(t_{1},\sigma ),\cdots ,rewr_{I}(t_{n},\sigma ))).%
\end{array}%
\end{equation*}%
This requires an extension of $rewr_{M}$ to applicative terms as well.

\section{Further work}

\begin{itemize}
\item Extend the algorithms $rewr_{M}$ and $rewr_{I}$ to applicative terms.
For this we have to extend match trees to applicative terms.

\item Extend the definition of terms with lambda expressions and quantifier
expressions, and extend the algorithms so they can handle them.

\item Design an algorithm for rewriting using strategies as defined in \cite%
{weerdenburg2009}.

\item Extend the rewrite algorithms so they handle evaluation of conditions
(as is required in the evaluation of a $C$-node).

\item Collect examples of higher order rewrite systems for testing the
algorithms.
\end{itemize}

\bibliographystyle{alpha}
\bibliography{wiki-rewriter}

\end{document}
