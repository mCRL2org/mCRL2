\documentclass{article}
\usepackage{fullpage}
\usepackage{amsmath}
\usepackage{amsfonts}
\usepackage{amssymb}
\usepackage[dvipsnames]{xcolor}

\setcounter{MaxMatrixCols}{10}

\font \aap cmmi10
\newcommand{\at}[1]{\mbox{\aap ,} #1}
\newcommand{\ap}{{:}}
\newcommand{\concat}{\ensuremath{+\!\!+}}
\newcommand{\tuple}[1]{\ensuremath{\langle {#1} \rangle}}
\newcommand{\vars}{\mathit{vars}}
\newtheorem{theorem}{Theorem}
\newtheorem{acknowledgement}[theorem]{Acknowledgement}
\newtheorem{algorithm}[theorem]{Algorithm}
\newtheorem{axiom}[theorem]{Axiom}
\newtheorem{case}[theorem]{Case}
\newtheorem{claim}[theorem]{Claim}
\newtheorem{conclusion}[theorem]{Conclusion}
\newtheorem{condition}[theorem]{Condition}
\newtheorem{conjecture}[theorem]{Conjecture}
\newtheorem{corollary}[theorem]{Corollary}
\newtheorem{criterion}[theorem]{Criterion}
\newtheorem{definition}[theorem]{Definition}
\newtheorem{example}[theorem]{Example}
\newtheorem{exercise}[theorem]{Exercise}
\newtheorem{lemma}[theorem]{Lemma}
\newtheorem{notation}[theorem]{Notation}
\newtheorem{problem}[theorem]{Problem}
\newtheorem{proposition}[theorem]{Proposition}
\newtheorem{remark}[theorem]{Remark}
\newtheorem{solution}[theorem]{Solution}
\newtheorem{summary}[theorem]{Summary}
\newenvironment{proof}[1][Proof]{\noindent\textbf{#1.} }{\ \rule{0.5em}{0.5em}}

\begin{document}

\title{PBES rewriters}
\author{Wieger Wesselink, Tim Willemse}
\maketitle

This document describes several rewriters on PBES expressions. We assume that a data
rewriter $r$ is given that rewrites data expressions.

\section{Simplifying rewriter}

We define a simplifying PBES rewriter $R$ recursively as follows. We assume
that $D$ is a non-empty data type, and we denote the free variables
appearing in $\varphi $ as $\mathsf{free}(\varphi )$. We assume that a data
rewriter $r$ is given that rewrites data terms.%
\begin{equation*}
\begin{array}{lcl}
b & \rightarrow & r(b) \\
\lnot \lnot \varphi & \rightarrow & \varphi \\
\varphi \wedge true & \rightarrow & \varphi \\
true\wedge \varphi & \rightarrow & \varphi \\
\varphi \wedge false & \rightarrow & false \\
false\wedge \varphi & \rightarrow & false \\
\varphi \wedge \varphi & \rightarrow & \varphi \\
\varphi \vee true & \rightarrow & true \\
true\vee \varphi & \rightarrow & true \\
\varphi \vee false & \rightarrow & \varphi \\
false\vee \varphi & \rightarrow & \varphi \\
\varphi \vee \varphi & \rightarrow & \varphi \\
\varphi \Rightarrow \psi & \rightarrow & \lnot \varphi \vee \psi \\
\forall _{d:D}.\varphi & \rightarrow & \varphi \text{ if }d\notin \mathsf{%
free}(\varphi ) \\
\forall _{d:D}.\lnot \varphi & \rightarrow & \lnot \exists _{d:D}.\varphi \\
\forall _{d:D}.\varphi \wedge \psi & \rightarrow & \forall _{d:D}.\varphi
\wedge \forall _{d:D}.\psi \\
\forall _{d:D}.\varphi \vee \psi & \rightarrow & (\forall _{d:D}.\varphi
)\vee \psi \text{ if }d\notin \mathsf{free}(\psi ) \\
\forall _{d:D}.\varphi \vee \psi & \rightarrow & \varphi \vee (\forall
_{d:D}.\psi )\text{ if }d\notin \mathsf{free}(\varphi ) \\
\exists _{d:D}.\varphi & \rightarrow & \varphi \text{ if }d\notin \mathsf{%
free}(\varphi ) \\
\exists _{d:D}.\lnot \varphi & \rightarrow & \lnot \forall _{d:D}.\varphi \\
\exists _{d:D}.\varphi \vee \psi & \rightarrow & \exists _{d:D}.\varphi \vee
\exists _{d:D}.\psi \\
\exists _{d:D}.\varphi \wedge \psi & \rightarrow & (\exists _{d:D}.\varphi
)\wedge \psi \text{ if }d\notin \mathsf{free}(\psi ) \\
\exists _{d:D}.\varphi \wedge \psi & \rightarrow & \varphi \wedge (\exists
_{d:D}.\psi )\text{ if }d\notin \mathsf{free}(\varphi ) \\
X(e) & \rightarrow & X(r(e))%
\end{array}%
\end{equation*}%
where $\varphi $ and $\psi $ are arbitrary pbes expressions, $b$ is a data
term of data sort $\mathbb{B}$, $true$ and $false$ are elements of data sort
$\mathbb{B}$, $X$ is a predicate variable, $e$ consists of zero or more data
sorts and $d$ is a data variable of sort $D$.

\newpage
\paragraph{Simplify}

The pbes expression rewrite system \textsc{Simplify} [Luc Engelen, 2007]
consists of the following rules\footnote{%
Todo: reformulate this rewrite system.}:%
\begin{eqnarray*}
false\wedge x &\rightarrow &false \\
x\wedge false &\rightarrow &false \\
true\wedge x &\rightarrow &x \\
x\wedge true &\rightarrow &x \\
\lnot true &\rightarrow &false \\
\lnot false &\rightarrow &true \\
ITE(true,x,y) &\rightarrow &x \\
ITE(false,x,y) &\rightarrow &y \\
x=x &\rightarrow &true \\
y=x &\rightarrow &x=y,\text{ provided }y\succ x
\end{eqnarray*}

\newpage
\section{PFNF Rewriter}

\begin{definition}
A predicate formula is said to be in \emph{Predicate Formula Normal Form}
(PFNF) if it has the following form:
\begin{equation*}
\mathsf{Q}_{1}v_{1}{:}V_{1}.\cdots \mathsf{Q}_{n}v_{n}{:}V_{n}.~h\wedge
\bigwedge\limits_{i\in I}\left( g_{i}\implies \bigvee\limits_{j\in
J_{i}}~X^{j}(e^{j})\right)
\end{equation*}%
where $X^{j}\in \chi $ ($\chi $ is a countable of sorted predicate
variables), $\mathsf{Q}_{i}\in \{\forall ,\exists \}$, $I$ is a (possibly
empty) finite index set, each $J_{i}$ is a non-empty finite index set, and $%
h $ and every $g_{i}$ are simple formulae.
\end{definition}

Note that here $J_{i}$ is used to index a set of occurrences of not
necessarily different variables. For instance, $(n>0\implies X(3)\vee
X(5)\vee Y(6))$ is a formula complying to the definition of PFNF. So long as
it does not lead to confusion, we stick to the convention to drop the typing
of the quantified variables $v_{i}$. An algorithm to compute a PFNF is:

\begin{equation*}
\begin{array}{lll}
p(c) & =_{def} & c \\
p(X(d)) & =_{def} & X(d) \\
p(\forall {x{:}D}.\varphi ) & =_{def} & \forall {x{:}D}.p(\varphi ) \\
p(\exists {x{:}D}.\varphi ) & =_{def} & \exists {x{:}D}.p(\varphi ) \\
&  &  \\
p(\varphi \wedge \psi ) & =_{def} &
\begin{array}{l}
\mathsf{Q}_{1}^{\varphi }\cdots \mathsf{Q}_{n^{\varphi }}^{\varphi }\mathsf{Q%
}_{1}^{\psi }\cdots \mathsf{Q}_{n^{\psi }}^{\psi }.~~\left( h^{\varphi
}\wedge h^{\psi }\right) \\
\wedge \bigwedge\limits_{i\in I^{\varphi }\cup I^{\psi }}\left(
g_{i}\implies \bigvee\limits_{j\in J_{i}}~X^{j}(e^{j})\right)%
\end{array}
\\
&  &  \\
p(\varphi \vee \psi ) & =_{def} &
\begin{array}{l}
\mathsf{Q}_{1}^{\varphi }\cdots \mathsf{Q}_{n^{\varphi }}^{\varphi }\mathsf{Q%
}_{1}^{\psi }\cdots \mathsf{Q}_{n^{\psi }}^{\psi }.\left( h^{\varphi }\vee
h^{\psi }\right) \\
\wedge \bigwedge\limits_{i\in I^{\varphi }}\left( \left( \lnot h^{\psi
}\wedge g_{i}\right) \implies \bigvee\limits_{j\in J_{i}}~X^{j}(e^{j})\right)
\\
\wedge \bigwedge\limits_{i\in I^{\psi }}\left( \left( \lnot h^{\varphi
}\wedge g_{i}\right) \implies \bigvee\limits_{j\in J_{i}}~X^{j}(e^{j})\right)
\\
\wedge \bigwedge\limits_{i\in I^{\varphi },k\in I^{\psi }}\left( \left(
g_{i}\wedge g_{k}\right) \implies \bigvee\limits_{j\in J_{i},m\in
J_{k}}~X^{j}(e^{j})\vee X^{m}(e^{m})\right)%
\end{array}%
\end{array}%
\end{equation*}%
where

\begin{equation*}
\begin{array}{lll}
p(\varphi ) & = & \mathsf{Q}_{1}^{\varphi }\cdots \mathsf{Q}_{n^{\varphi
}}^{\varphi }.~h^{\varphi }\wedge \bigwedge\limits_{i\in I^{\varphi }}\left(
g_{i}\implies \bigvee\limits_{j\in J_{i}}~X^{j}(e^{j})\right) \\
p(\psi ) & = & \mathsf{Q}_{1}^{\psi }\cdots \mathsf{Q}_{n^{\psi }}^{\psi
}.~h^{\psi }\wedge \bigwedge\limits_{i\in I^{\psi }}\left( g_{i}\implies
\bigvee\limits_{j\in J_{i}}~X^{j}(e^{j})\right) ,%
\end{array}%
\end{equation*}%
under the assumption that $I^{\varphi }$ and $I^{\psi }$ are disjoint, and $%
v_{i}^{\varphi }\neq v_{j}^{\psi }$ for all $i,j$.\newpage

\section{OnePointRule Quantifier Elimination Rewriter}

The function $Eq$ computes a set of equalities and inequalities for an
expression $\varphi $, such that the following holds:%
\begin{eqnarray*}
Eq\left( \varphi \right)  &=&\left( \left\{ b_{1}=e_{1},\cdots
,b_{n}=e_{n}\right\} ,W\right) \Rightarrow \varphi =\psi \wedge
\bigwedge\limits_{i=1}^{n}\left( b_{i}=e_{i}\right)  \\
Eq\left( \varphi \right)  &=&\left( V,\left\{ b_{1}\neq e_{1},\cdots
,b_{n}\neq e_{n}\right\} \right) \Rightarrow \varphi =\psi \vee
\bigvee\limits_{i=1}^{n}\left( b_{i}\neq e_{i}\right)
\end{eqnarray*}%
for some expression $\psi $.

The function $Eq$ is inductively defined as follows:

\begin{equation*}
\begin{array}{lll}
Eq(b) & = & \left( \left\{ b=true\right\} ,\emptyset \right) \\
Eq(d=e) & = & \left\{
\begin{array}{cc}
\left( \left\{ d=e\right\} ,\emptyset \right) & \text{if }d\notin FV(e) \\
\left( \emptyset ,\emptyset \right) & \text{otherwise}%
\end{array}%
\right. \\
Eq(e=d) & = & Eq(d=e) \\
Eq(d\neq e) & = & \left\{
\begin{array}{cc}
\left( \emptyset ,\left\{ d\neq e\right\} \right) & \text{if }d\notin FV(e)
\\
\left( \emptyset ,\emptyset \right) & \text{otherwise}%
\end{array}%
\right. \\
Eq(e\neq d) & = & Eq(d\neq e) \\
Eq\left( \lnot \varphi \right) & = & swap\left( Eq\left( \varphi \right)
\right) \\
Eq\left( \varphi \wedge \psi \right) & = & join\_and\left( \left( Eq\left(
\varphi \right) ,Eq\left( \psi \right) \right) \right) \\
Eq\left( \varphi \vee \psi \right) & = & join\_or\left( \left( Eq\left(
\varphi \right) ,Eq\left( \psi \right) \right) \right) \\
Eq\left( \varphi \Rightarrow \psi \right) & = & join\_or\left( \left(
swap\left( Eq\left( \varphi \right) \right) ,Eq\left( \psi \right) \right)
\right) \\
Eq\left( \forall x.\varphi \right) & = & delete\left( x,Eq\left( \varphi
\right) \right) \\
Eq\left( \exists x.\varphi \right) & = & delete\left( x,Eq\left( \varphi
\right) \right) \\
Eq\left( \varphi \right) & = & \left( \emptyset ,\emptyset \right) \text{
otherwise}%
\end{array}%
\end{equation*}%
where $b$ is a boolean data variable, $d$ is an arbitrary data variable, $e$
is a boolean data expression, and%
\begin{equation*}
\begin{array}{lll}
swap\left( \left( V,W\right) \right) & = & \left( W,V\right) \\
join\_and\left( \left( V_{1},W_{1}\right) ,\left( V_{2},W_{2}\right) \right)
& = & \left( V_{1}\cup V_{2},W_{1}\cap W_{2}\right) \\
join\_or\left( \left( V_{1},W_{1}\right) ,\left( V_{2},W_{2}\right) \right)
& = & \left( V_{1}\cap V_{2},W_{1}\cup W_{2}\right) \\
delete\left( x,\left( V,W\right) \right) & = &
\begin{array}{l}
\left( V_{1},W_{1}\right) \text{ where} \\
\left\{
\begin{array}{c}
V_{1}=\left\{ d=e\in V\mid d\neq x\wedge x\notin FV(e)\right\} \\
W_{1}=\left\{ d\neq e\in W\mid d\neq x\wedge x\notin FV(e)\right\}%
\end{array}%
\right.%
\end{array}%
\end{array}%
\end{equation*}

We define the OnePointRule rewriter $R$ inductively as follows:

\begin{equation*}
\begin{array}{lll}
R\left( \lnot \varphi \right) & = & \lnot R\left( \varphi \right) \\
R\left( \varphi \wedge \psi \right) & = & R\left( \varphi \right) \wedge
R\left( \psi \right) \\
R\left( \varphi \vee \psi \right) & = & R\left( \varphi \right) \vee R\left(
\psi \right) \\
R\left( \varphi \Rightarrow \psi \right) & = & R\left( \varphi \right)
\Rightarrow R\left( \psi \right) \\
R\left( \forall x.\varphi \right) & = & \left\{
\begin{array}{cc}
R\left( \varphi \right) \left[ x:=e\right] & \text{if }x\neq e\in W\text{,
where }\left( V,W\right) =Eq\left( \varphi \right) \\
\forall x.R\left( \varphi \right) & \text{otherwise}%
\end{array}%
\right. \\
R\left( \exists x.\varphi \right) & = & \left\{
\begin{array}{cc}
R\left( \varphi \right) \left[ x:=e\right] & \text{if }x=e\in V\text{, where
}\left( V,W\right) =Eq\left( \varphi \right) \\
\exists x.R\left( \varphi \right) & \text{otherwise}%
\end{array}%
\right. \\
R\left( \varphi \right) & = & \varphi \text{ otherwise}%
\end{array}%
\end{equation*}

\newpage
\section{Quantifier Inside Rewriter}

This rewriter was originally specified by Jan Friso Groote.
We define a rewriter that pushes universal and existential quantifiers into an
expression. A typical example is
\[ \exists x.\forall y.(f(y,y)\wedge (f(x,y)\vee f(x,x))) \]
which is rewritten to
\[(\forall y.f(y,y)) \wedge ((\exists x.\forall y.f(x,y)) \vee \exists x.f(x,x)).\]
If quantifiers are pushed inside formulas as much as possible, quantifier elimination leads to smaller
expressions and the one-point rewriter is more often applicable. There might be cases where the
application of this push-quantifiers-inside-rewriter can also have averse effects.

Below the definition of the rewrite rules are given. It is assumed that it is cheap to obtain the
free variable in each term, which are denoted by $\vars(\phi)$ for a term $\phi$.
If not, care needs to be taken as in the formulation below calculating the
variables of each term recursively on a by need basis can be very expensive.

Let $\phi$ be a pbes expression. We
define $R(\phi)$ using the functions $R_{\forall}(V,\phi)$ and
$R_{\exists}(V,\phi)$ where $V$ is a set of typed variables. The
definition of $R$ employs the structure of a formula:
\begin{equation*}
\begin{array}{lll}
R(\neg \phi)&=&\neg R(\phi)\\
R(\phi\wedge \psi)&=&R(\phi)\wedge R(\psi)\\
R(\phi\vee\psi)&=&R(\phi)\vee R(\psi)\\
R(\phi\Rightarrow\psi)&=&R(\phi)\Rightarrow R(\psi)\\
R(\forall W.\phi)&=&R_{\forall}(W,R(\phi))\\
R(\exists W.\phi)&=&R_{\exists}(W,R(\phi))\\
R(\phi)&=&\phi\textrm{  otherwise}.
\end{array}
\end{equation*}
Here $W$ is a set of typed variables.

The rewrite function $R_{\forall}$ is defined by

\begin{equation*}
\begin{array}{lll}
R_{\forall}(V,\neg \phi) &=& \neg R_{\exists}(V,\phi)\\

R_{\forall}(V,\phi\wedge \psi)&=&R_{\forall}(V,\phi)\wedge R_{\forall}(V,\psi)\\

R_{\forall}(V, \bigvee_i \phi_i)&=&
  \forall W.
    \left(
         R_\forall(V \setminus W, \bigvee_{i \in I_1}\phi_i) \lor \bigvee_{i \in I_2}R(\phi_i)
    \right) \\
  && \text{where } W = \vars(\phi_j) \cap V \text{ for some $j$ such that $|W|$ is minimal,} \\
  && \text{and where } I_1 = \{ i \mid \vars(\phi_i) \cap V \nsubseteq W \}, I_2 = \{ i \mid \vars(\phi_i) \cap V \subseteq W \} \\

R_{\forall}(V,\phi\Rightarrow\psi)&=&
\forall W.(R_{\exists}(V\setminus W,\phi)\Rightarrow{}
R_{\forall}(V \setminus W,\psi)), 
\text{ where } W = V \cap \vars(\phi) \cap \vars(\psi) \\

R_{\forall}(V,\forall W.\phi)&=&R_{\forall}(V \cup W,\phi) \\

R_{\forall}(V,\phi)&=&\forall V\cap \vars(\phi).\phi\textrm{ otherwise}.

\end{array}
\end{equation*}

The rewrite function $R_{\exists}$ is defined by
\begin{equation*}
\begin{array}{lll}
R_{\exists}(V,\neg \phi)&=&\neg R_{\forall}(V,\phi)\\

R_{\exists}(V, \bigwedge_i \phi_i)&=&
  \exists W.
    \left(
         R_\exists(V \setminus W, \bigvee_{i \in I_1}\phi_i) \lor \bigvee_{i \in I_2}R(\phi_i)
    \right) \\
  && \text{where } W = \vars(\phi_j) \cap V \text{ for some $j$ such that $|W|$ is minimal,} \\
  && \text{and where } I_1 = \{ i \mid \vars(\phi_i) \cap V \nsubseteq W \}, I_2 = \{ i \mid \vars(\phi_i) \cap V \subseteq W \} \\

R_{\exists}(V,\phi\vee \psi)&=&R_{\exists}(V,\phi)\vee R_{\exists}(V,\psi)\\

R_{\exists}(V,\phi\Rightarrow\psi)&=&
R_{\forall}(V,\phi)\Rightarrow R_{\exists}(V,\psi)\\

R_{\exists}(V,\exists W.\phi)&=&R_{\exists}(V \cup W,\phi)\\

R_{\exists}(V,\phi)&=&\exists V\cap\vars(\phi).\phi\textrm{ otherwise}.
\end{array}
\end{equation*}

\newpage
\section{Quantifier Expansion Rewriter}

\subsection{Conjunctions / disjunctions}

The function $CD$ computes a sequence of conjunctions and disjunctions for
an expression $\varphi $, such that the following holds:%
\begin{eqnarray*}
CD\left( \varphi \right) &=&\left( \left[ \varphi _{1},\cdots ,\varphi _{n}%
\right] ,W\right) \Rightarrow \varphi =\bigwedge\limits_{i=1}^{n}\varphi _{i}
\\
CD\left( \varphi \right) &=&\left( V,\left[ \varphi _{1},\cdots ,\varphi _{n}%
\right] \right) \Rightarrow \varphi =\bigvee\limits_{i=1}^{n}\varphi _{i}.
\end{eqnarray*}

The function $S$ is inductively defined as follows:

\begin{equation*}
\begin{array}{lll}
CD\left( \lnot \varphi \right) & = & negate\left( CD\left( \varphi \right)
\right) \\
CD\left( \varphi \wedge \psi \right) & = & \left( conjunctions\left(
CD\left( \varphi \right) \right) \concat conjunctions\left( CD\left( \psi
\right) \right) ,\emptyset \right) \\
CD\left( \varphi \vee \psi \right) & = & \left( \emptyset
,disjunctions\left( CD\left( \varphi \right) \right) \concat %
disjunctions\left( CD\left( \psi \right) \right) \right) \\
CD\left( \varphi \Rightarrow \psi \right) & = & \left( \emptyset
,disjunctions\left( negate(CD\left( \varphi \right) )\right) \concat %
disjunctions\left( CD\left( \psi \right) \right) \right) \\
CD\left( \varphi \right) & = & \left( \varphi ,\varphi \right) \text{
otherwise}%
\end{array}%
\end{equation*}%
with%
\begin{equation*}
\begin{array}{lll}
negate\left( \left( \left[ \varphi _{1},\cdots ,\varphi _{m}\right] ,\left[
\psi _{1},\cdots ,\psi _{n}\right] \right) \right) & = & \left( \left[ \lnot
\psi _{1},\cdots ,\lnot \psi _{n}\right] ,\left[ \lnot \varphi _{1},\cdots
,\lnot \varphi _{m}\right] \right) \\
conjunctions\left( \left( V,W\right) \right) & = & \left\{
\begin{array}{ll}
V & \text{if }V\neq \emptyset \\
\bigvee\limits_{w\in W}w & \text{otherwise}%
\end{array}%
\right. \\
disjunctions\left( \left( V,W\right) \right) & = & \left\{
\begin{array}{ll}
W & \text{if }W\neq \emptyset \\
\bigwedge\limits_{v\in V}v & \text{otherwise}%
\end{array}%
\right.%
\end{array}%
\end{equation*}

N.B. This has not been implemented yet.

\end{document}