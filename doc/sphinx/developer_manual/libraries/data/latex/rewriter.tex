
\documentclass{article}
%%%%%%%%%%%%%%%%%%%%%%%%%%%%%%%%%%%%%%%%%%%%%%%%%%%%%%%%%%%%%%%%%%%%%%%%%%%%%%%%%%%%%%%%%%%%%%%%%%%%%%%%%%%%%%%%%%%%%%%%%%%%%%%%%%%%%%%%%%%%%%%%%%%%%%%%%%%%%%%%%%%%%%%%%%%%%%%%%%%%%%%%%%%%%%%%%%%%%%%%%%%%%%%%%%%%%%%%%%%%%%%%%%%%%%%%%%%%%%%%%%%%%%%%%%%%
\usepackage{amssymb}
\usepackage{amsfonts}
\usepackage{amsmath}
\usepackage{a4wide}
\usepackage[british]{babel}

\setcounter{MaxMatrixCols}{10}
%TCIDATA{OutputFilter=LATEX.DLL}
%TCIDATA{Version=5.50.0.2890}
%TCIDATA{<META NAME="SaveForMode" CONTENT="1">}
%TCIDATA{BibliographyScheme=Manual}
%TCIDATA{Created=Monday, October 19, 2009 15:45:27}
%TCIDATA{LastRevised=Friday, January 15, 2010 10:28:48}
%TCIDATA{<META NAME="GraphicsSave" CONTENT="32">}
%TCIDATA{<META NAME="DocumentShell" CONTENT="Standard LaTeX\Standard LaTeX Article">}
%TCIDATA{CSTFile=40 LaTeX article.cst}

\newtheorem{theorem}{Theorem}
\newtheorem{acknowledgement}[theorem]{Acknowledgement}
\newtheorem{algorithm}[theorem]{Algorithm}
\newtheorem{axiom}[theorem]{Axiom}
\newtheorem{case}[theorem]{Case}
\newtheorem{claim}[theorem]{Claim}
\newtheorem{conclusion}[theorem]{Conclusion}
\newtheorem{condition}[theorem]{Condition}
\newtheorem{conjecture}[theorem]{Conjecture}
\newtheorem{corollary}[theorem]{Corollary}
\newtheorem{criterion}[theorem]{Criterion}
\newtheorem{definition}[theorem]{Definition}
\newtheorem{example}[theorem]{Example}
\newtheorem{exercise}[theorem]{Exercise}
\newtheorem{lemma}[theorem]{Lemma}
\newtheorem{notation}[theorem]{Notation}
\newtheorem{problem}[theorem]{Problem}
\newtheorem{proposition}[theorem]{Proposition}
\newtheorem{remark}[theorem]{Remark}
\newtheorem{solution}[theorem]{Solution}
\newtheorem{summary}[theorem]{Summary}
\newenvironment{proof}[1][Proof]{\noindent\textbf{#1.} }{\ \rule{0.5em}{0.5em}}
\input{include/tcilatex}
\begin{document}

\title{Rewriter Implementation Notes}
\author{Wieger Wesselink}
\maketitle

\section{Introduction}

This document describes rewrite algorithms that can be applied in the mCRL2
tool set. Currently only a prototype implementation in python is available.
Most of the content is based on \cite{weerdenburg2009} and on \cite%
{vanderwulp2009}.

\subsection{Higher order rewriting}

There are several formalisms for higher order rewriting. We choose
higher-order rewriting systems (HRSs) introduced by Nipkow. In \cite%
{DBLP:conf/rta/Raamsdonk01} HRSs are summarized as follows:

In a HRS we work modulo the $\beta \eta $-relation of simply typed $\lambda $%
-calculus. \emph{Types} are built from a non-empty set of base types and the
binary type constructor $\rightarrow $ as usual. For every type we assume a
countably infinite set of \emph{variables} of that type, written as $%
x,y,z,\ldots $. A \emph{signature} is a non-empty set of typed function
symbols. The set of \emph{preterms} of type $A$ over a signature $\Sigma $
consists exactly of the expressions $s$for which we can derive $s:A$ using
the following rules:

\begin{enumerate}
\item $x:A$ for a variable $x$ of type $A$,

\item $f:A$ for a function symbol $f$ of type $A$ in $\Sigma $,

\item if $A=A^{\prime }\rightarrow A^{\prime \prime }$, and $x:A^{\prime }$
and $s:A^{\prime \prime }$, then $(x.s):A$,

\item if $s:A^{\prime }\rightarrow A$ and $t:A^{\prime }$, then $(s$\ $t):A$.
\end{enumerate}

The abstraction operator $\_.\_$ binds variables, so occurrences of $x$ in $s
$ in the preterm $x.s$ are bound. We work modulo type-preserving $\alpha $%
-conversion and assume that bound variables are renamed whenever necessary
in order to avoid unintended capturing of free variables. Parentheses may be
omitted according to the usual conventions. We make use of the usual notions
of \emph{substitution} of a preterm $t$ for the free occurrences of a
variable $x$ in a preterm $s$, notation $s[x:=t]$, and \emph{replacement in
a context}, notation $C[t]$. We write $s\supseteq s^{\prime }$ if $s^{\prime
}$ is a subpreterm of $s$, and use $\supset $ for the strict subpreterm
relation.

The $\beta $\emph{-reduction relation}, notation $\rightarrow _{\beta }$, is
the smallest relation on preterms that is compatible with formation of
preterms and that satisfies the following:%
\begin{equation*}
(x,s)t\rightarrow _{\beta }s[x:=t]
\end{equation*}%
The \emph{restricted }$\eta $\emph{-expansion relation}, notation $%
\rightarrow _{\overline{\eta }}$, is defined as follows. We have%
\begin{equation*}
C[s]\rightarrow _{\overline{\eta }}C[x.(s\ x)]
\end{equation*}%
if $s:A\rightarrow B$, and $x:A$ is a fresh variable, and no $\beta $-redex
is created (hence the terminology restricted $\eta $-expansion). The latter
condition is satisfied if $s$ is not an abstraction (so not of the form $%
z.s^{\prime }$), and doesn't occur in $C[s]$ as the left part of an
application (so doesn't occur in a sub-preterm of the form $(s\ s^{\prime })$%
).

In the sequel we employ only preterms in $\overline{\eta }$-normal form,
where every sub-preterm has the right number of arguments. Instead of $%
s_{0}s_{1}\ldots s_{m}$ we often write $s_{0}(s_{1},\ldots ,s_{m})$. A
preterm is then of the form $x_{1}\ldots x_{n}.s_{0}(s_{1},\ldots ,s_{m})$
with $s_{0}(s_{1},\ldots ,s_{m})$ of base type and all $s_{i}$ in $\overline{%
\eta }$-normal form.

A \emph{term} is a preterm in $\beta $-normal form. It is also in $\overline{%
\eta }$-normal form because $\overline{\eta }$-normal forms are closed under 
$\beta $-reduction. A term is of the form $x_{1}\ldots x_{n}.a(s_{1},\ldots
,s_{m})$ with $a$ a function symbol or a variable. Because the $\beta 
\overline{\eta }$-reduction relation is confluent and terminating on the set
of preterms, every $\beta \overline{\eta }$-equivalence class of preterms
contains a unique term, which is taken as the representative of that class.

Because in the discussion we will often use preterms, we use here the
notation $s^{\sigma }$ for the replacement of variables according to the
substitution $\sigma $ (\emph{without} reduction to $\beta $-normal form),
and write explicitly $s^{\sigma }\downarrow _{\beta }$for its $\beta $%
-normal form. This is in contrast with the usual notations for HRSs.

A \emph{rewrite rule} is a pair of terms $(l,r)$, written as $l\rightarrow r$%
, satisfying the following requirements:

\begin{enumerate}
\item $l$ and $r$ are of the same base type,

\item $l$ is of the form $f(l_{1},\ldots ,l_{n})$,

\item all free variables in $r$ occur also in $l$,

\item a free variable $x$ in $l$ occurs in the form $x(y_{1},\ldots ,y_{n})$
with $y_{i}$ $\eta $-equivalent to different bound variables.
\end{enumerate}

The last requirement guarantees that the rewrite relation is decidable
because unification of patterns is decidable. The rewrite rules induce a
rewrite relation $\rightarrow $ on the set of terms which is defined by the
following rules:

\begin{enumerate}
\item if $s\rightarrow t$ then $x(\ldots ,s,\ldots )\rightarrow x(\ldots
,t,\ldots )$,

\item if $s\rightarrow t$ then $f(\ldots ,s,\ldots )\rightarrow f(\ldots
,t,\ldots )$,

\item if $s\rightarrow t$ then $x.s\rightarrow x.t$,

\item if $l\rightarrow r$ is a rewrite rule and $\sigma $ is a substitution
then $l^{\sigma }\downarrow _{\beta }\rightarrow r^{\sigma }\downarrow
_{\beta }$.
\end{enumerate}

The last clause in this definition shows that HRSs use higher-order pattern
matching, unlike AFSs, where matching is syntactic.

\subsection{mCRL2 terms}

In mCRL2 we have the following terms:%
\begin{equation*}
t:=x\ |\ f\ |\ t(t,\cdots ,t)\ |\ \lambda _{x}.t\ |\ \forall _{x}.t\ |\
\exists _{x}.t\ |\ t\ whr\ x=t^{\prime }
\end{equation*}%
where $t$ is a term, $x$ is a variable and $f$ is a function symbol.

\begin{remark}
This needs to be further elaborated. Terms are typed, and function symbols
(and terms?) have an arity. The term $t(t,\cdots ,t)$ is rather unusual, but
it is covered by HRSs (?).
\end{remark}

\begin{remark}
In fact the mCRL2 language uses slightly more general terms: $\lambda
_{x_{1}\ldots x_{n}}.t$, $\forall _{x_{1}\ldots x_{n}}.t$, $\exists
_{x_{1}\ldots x_{n}}.t$ and $t\ whr\ x_{1}=t_{1},\ldots ,x_{n}=t_{n}$.
\end{remark}

For a rewrite algorithm $rewr$ the following rules are suggested:%
\begin{eqnarray*}
rewr(\lambda _{x}.t,\sigma ) &=&\lambda _{x^{\prime }}.rewr(t,\sigma \lbrack
x:=x^{\prime }]) \\
rewr(\forall _{x}.t,\sigma ) &=&\forall _{x^{\prime }}.rewr(t,\sigma \lbrack
x:=x^{\prime }]) \\
rewr(\exists _{x}.t,\sigma ) &=&\exists _{x^{\prime }}.rewr(t,\sigma \lbrack
x:=x^{\prime }]) \\
rewr(t\ whr\ x &=&t^{\prime },\sigma )=rewr(t,\sigma \lbrack x:=t^{\prime }])
\end{eqnarray*}%
where $x^{\prime }$ is a fresh variable not appearing in $t$.

\begin{remark}
In a rewrite algorithm, the term types are unused. One can add correctness
checks for proper typing however.
\end{remark}

\begin{remark}
Types in mCRL2 need to be rewritten to normal form as well. A very simple
rewrite system can be defined for this. 
\end{remark}

\begin{remark}
What about normal forms for terms containing $\lambda $-expressions and/or
quantifiers? Expressions can be equal modulo alpha-conversion, so the ATerm
equality doesn't work here.
\end{remark}

\begin{remark}
In a rewrite algorithm one has to explicitly describe where $\alpha $%
-conversion and $\beta $-reduction is being done. Doing $\overline{\eta }$%
-expansion is probably not necessary.
\end{remark}

\subsection{Types}

A \emph{base type} is a non-function type, typical examples are the Booleans
or Natural numbers. Let $B$ be a non-empty set of base types and $b\in B$.
The set of types is inductively defined as follows:%
\begin{equation*}
type::=b\ |\ type\times type\ |\ type\rightarrow type\text{,}
\end{equation*}%
where $\rightarrow $ is the function-type constructor. The type constructor
associates to the left, for example:%
\begin{equation*}
b\rightarrow b\rightarrow b\text{ is the same as }(b\rightarrow
b)\rightarrow b\text{.}
\end{equation*}%
Product types are often not present in treatment of simply-typed lambda
calculus. We need them later to type non-lambda terms.

The \emph{arity} of a type $A$ is a natural number, denoted $arity(A)$,
which is inductively defined on the structure of $A$ as follows:

\begin{equation*}
\begin{array}{ll}
arity(A)=0 & \text{if }A\text{ is a base type,} \\ 
arity(A\rightarrow A^{\prime })=arity(A^{\prime })+1 & 
\end{array}%
\end{equation*}

A \emph{signature} $\Sigma $ is a non-empty set of \emph{function symbols}
each of which has a type. We write $f:A$ to denote that symbol $f$ has type $%
A$ and extend the notion arity to symbols such that if $f:A$ then $%
arity(f)=arity(A)$. Symbols with arity zero are called \emph{constants}.

Let $\Sigma $ be a signature and let $\chi _{A}$ be a countably finite set
of variables of type $A$ such that $\Sigma \cap \chi _{A}=\emptyset $, for
each type $A$. The set of terms over $\Sigma $, denoted $\mathcal{T}(\Sigma
) $, is inductively defined as

\begin{itemize}
\item Let $x\in \chi _{A}$ be a variable of type $A$ then $x\in \mathcal{T}%
(\Sigma )$.

\item Let $f\in \Sigma $ be a function symbol of type $A_{1}\times \ldots
\times A_{n}\rightarrow B$, and $t_{i}:A_{i}$ for all $i\in \{1,\ldots ,n\}$%
, then $f(t_{1},\ldots ,t_{n})\in \mathcal{T}(\Sigma )$ is a term of type $B$%
.

\item Let $t:A_{1}\times \ldots \times A_{n}\rightarrow B$, and $t_{i}:A_{i}$
for all $i\in \{1,\ldots ,n\}$, then $t(t_{1},\ldots ,t_{n})\in \mathcal{T}%
(\Sigma )$ is a term of type $B$.

\item Let $x\in \chi _{A}$ be a variable of type $A$ and $t$ a term of type $%
B$, then $\lambda _{x}.(t)\in \mathcal{T}(\Sigma )$ is a term of type $%
A\rightarrow B$.
\end{itemize}

\subsection{Simple terms}

Simple terms are terms with the following syntax:%
\begin{equation}
t:=x\ |\ f\ |\ f(t,\cdots ,t),  \label{eq:simple_terms}
\end{equation}%
where $t$ is a term, $x$ is a variable and $f$ is a function symbol.

\subsection{Applicative terms}

Applicative terms are an extension of simple terms:%
\begin{equation}
t:=x\ |\ f\ |\ t(t,\cdots ,t).  \label{eq:applicative_terms}
\end{equation}

The set of all variables is denoted by $\mathbb{V}$, the set of all function
symbols by $\mathbb{F}$ and the set of all terms by $\mathbb{T}$. In this
document we use the convention that $x,y\in \mathbb{V}$, that $t,u\in 
\mathbb{T}$, and that $f,g\in \mathbb{F}$.

We write $var(t)$ for the set of variables that occur in $t$. Formally:%
\begin{equation*}
\begin{array}{lll}
var(x) & = & \{x\} \\ 
var(f) & = & \emptyset \\ 
var(t(t_{1},\cdots ,t_{n}) & = & var(t)\cup \dbigcup\limits_{i=1\cdots
n}var(t_{i}).%
\end{array}%
\end{equation*}

\subsection{Subterms}

To facilitate operations on subterms we inductively define positions ($%
\mathbb{P}$) as follows. A position is either $\epsilon $ (the empty
position) or an index $i$ (from 1,2,$\cdots $) combined with a position $\pi 
$, notation $i\cdot \pi $. We lift $\cdot $ to an associative operator on
positions with $\epsilon $ as its unit element and often write just $i$ for
the position $i\cdot \epsilon $. We write the subterm of $t$ at position $%
\pi $ as $t|_{\pi }$ and we write term $t$ with the subterm at position $\pi 
$ replaced by $u$ as $t[u]_{\pi }$. These operations are defined as follows.%
\begin{equation*}
\begin{array}{llll}
t|_{\epsilon } & = & t &  \\ 
t(t_{1},\cdots ,t_{n})|_{i\cdot \pi } & = & t_{i}|_{\pi } & \text{if }1\leq
i\leq n \\ 
t[u]_{\epsilon } & = & u &  \\ 
x[u]_{i\cdot \pi } & = & x &  \\ 
f(t_{1},\cdots ,t_{n})[u]_{i\cdot \pi } & = & f(t_{1},\cdots
,t_{i-1},t_{i}[u]_{\pi },t_{i+1,}\cdots ,t_{n}) & \text{if }i\leq n \\ 
f(t_{1},\cdots ,t_{n})[u]_{i\cdot \pi } & = & f(t_{1},\cdots ,t_{n}) & \text{%
if }i>n%
\end{array}%
\end{equation*}

Some examples are:%
\begin{equation*}
\begin{array}{l}
f(x,g(y))|_{1}=x \\ 
f(x,g(y))|_{2\cdot 1}=y \\ 
f(x,g(y))[h(x)]_{2}=f(x,h(x))%
\end{array}%
\end{equation*}

\subsection{Substitutions}

A substitution is a function $\sigma :\mathbb{V\rightarrow T}$. A
substitution $\sigma $ can also be applied to a term $t$. This is denoted by 
$t\sigma $ and it is defined as%
\begin{equation*}
\begin{array}{lll}
x\sigma & = & \sigma (x) \\ 
f\sigma & = & f \\ 
t(t_{1},\cdots ,t_{n})\sigma & = & t\sigma (t_{1}\sigma ,\cdots ,t_{n}\sigma
).%
\end{array}%
\end{equation*}

\subsection{Rewrite rules}

A rewrite rule is a rule $l\rightarrow r\ \mathbf{if}\ c$, with $l,r,c\in 
\mathbb{T}$. We put three restrictions on rewrite rules:%
\begin{equation*}
\begin{array}{ll}
1) & l\text{ is a simple term} \\ 
2) & l\notin \mathbb{V} \\ 
3) & var(r)\cup var(c)\subseteq var(l)%
\end{array}%
\end{equation*}

For a set $R$ of rewrite rules we define the rewrite relation $\rightarrow
_{R}$ as follows: $t\rightarrow _{R}u$ if there is a rule $l\rightarrow r\ 
\mathbf{if}\ c$ in $R$ $,$ a position $\pi $ and a substitution $\sigma $
such that%
\begin{equation}
t|_{\pi }=l\sigma \wedge u=t[r\sigma ]_{\pi }\wedge \eta (c\sigma ),
\label{eq:rewriting}
\end{equation}%
where $\eta $ is a boolean function that determines if a condition is true.
We write $\rightarrow $ instead of $\rightarrow _{R}$ if no confusion can
occur. We write $\rightarrow _{R}^{\ast }$ for the reflexive and transitive
closure of $\rightarrow _{R}$ and $t\nrightarrow _{R}$ if there is no $u$
such that $t\rightarrow _{R}u$. A normal form is a term $u$ such that $%
t\rightarrow _{R}^{\ast }u$ and $u\nrightarrow _{R}$.

\subsection{Rewrite algorithm}

We now fomulate an abstract rewrite algorithm $rewrite$, where we assume
that $R$ is a given, fixed set of rewrite rules.%
\begin{equation*}
\begin{array}{l}
\text{\textbf{function} }rewrite(t) \\ 
u:=t \\ 
\text{\textbf{while} }\{v\ |\ u\rightarrow _{R}v\}\neq \emptyset \text{ 
\textbf{do}} \\ 
\qquad \text{\textbf{choose} }v\text{ \textbf{such that} }u\rightarrow _{R}v
\\ 
\qquad u:=v \\ 
\text{\textbf{return} }u%
\end{array}%
\end{equation*}%
Note that this algorithm does not need to terminate. In practice we are also
interested in an algorithm $rewrite(t,\sigma )$, that applies a substitution 
$\sigma $ to the variables in $t$ during rewriting. The specification of
this algorithm is simply%
\begin{equation*}
rewrite(t,\sigma )=rewrite(t\sigma ).
\end{equation*}%
The reason we are interested in such an algorithm is that it can be
implemented more efficiently than the straightforward solution to first
compute $u=t\sigma $ and then compute $rewrite(u)$.

\section{Match trees}

A match tree is a tree structure that represents a number of rewrite rules
that have left hand sides with the same function symbol as head. It is used
to compute all possible results of applying one of these rules to a term.
Currently match trees are only defined for simple terms. A match tree
consists of nodes of the following types:

\begin{itemize}
\item $F(f,T,U):$ If the current term has the form $f(t_{1},\cdots ,t_{n})$
replace the top of the stack by $t_{1}\rhd \cdots \rhd t_{n}$ and continue
with $T$, otherwise continue with $U$.

\item $S(x,T):$ Assign the current term to variable $x$ and continue with $T$%
.

\item $M(x,T,U):$ If the current term is equal to $x$ continue with $T$,
otherwise continue with $U$.

\item $R(Q):$ Return $Q$

\item $X:$ Return the empty set.

\item $N(n,T):$ Remove $n$ elements from the stack and continue with $T$. We
abbreviate $N(1,T)$ as $N(T)$.

\item $E(T,U):$ If the stack is not empty continue with $T$, otherwise
continue with $U$.

\item $C(t,T,U):$ If $t$ evaluates to $true$, continue with $T$, otherwise
continue with $U$.
\end{itemize}

where $f$ is a function symbol, $x$ is a variable, $t$ is a term, $Q$ is a
set of terms annotated with a rewrite rule, and $T$ and $U$ are match tree
nodes.

\subsection{Evaluating a match tree}

Let $l$ be a sequence of terms, and let $\sigma $ be an arbitrary
substitution function. Then the evaluation of a match tree with arguments $l$
and $\sigma $ is a set of terms and is defined as follows:%
\begin{equation*}
\begin{array}{lll}
F(f,T,U)(l,\sigma ) & = & \left\{ 
\begin{array}{ll}
\emptyset & \text{if }l=[] \\ 
T(m,\sigma ) & \text{if }l=f\rhd m \\ 
T(t_{1}\rhd \cdots \rhd t_{n}\rhd m,\sigma ) & \text{if }l=f(t_{1},\cdots
,t_{n})\rhd m \\ 
U(l,\sigma ) & \text{if }l=g(t_{1},\cdots ,t_{n})\rhd m\wedge f\neq g \\ 
U(l,\sigma ) & \text{if }l=x\rhd m%
\end{array}%
\right. \\ 
X(l,\sigma ) & = & \emptyset \\ 
R(Q)(l,\sigma ) & = & \left\{ 
\begin{array}{ll}
\{\sigma (t)\ |\ t^{\alpha }\in Q\} & \text{if }l=[] \\ 
\emptyset & \text{if }l\neq \lbrack ]%
\end{array}%
\right. \\ 
S(x,T) & = & \left\{ 
\begin{array}{ll}
\emptyset & \text{if }l=[] \\ 
T(l,\sigma \lbrack x\rightarrow t]) & \text{if }l=t\rhd m%
\end{array}%
\right. \\ 
M(x,T,U)(l,\sigma ) & = & \left\{ 
\begin{array}{ll}
\emptyset & \text{if }l=[] \\ 
T(l,\sigma ) & \text{if }l=t\rhd m\wedge \sigma (x)=t \\ 
U(l,\sigma ) & \text{if }l=t\rhd m\wedge \sigma (x)\neq t%
\end{array}%
\right. \\ 
N(n,T)(l,\sigma ) & = & \left\{ 
\begin{array}{ll}
\emptyset & \text{if }|l|\ <n \\ 
T(m,\sigma ) & \text{if }l=t_{1}\rhd \cdots \rhd t_{n}\rhd m%
\end{array}%
\right. \\ 
E(T,U)(l,\sigma ) & = & \left\{ 
\begin{array}{ll}
U(l,\sigma ) & \text{if }l=[] \\ 
T(l,\sigma ) & \text{if }l=t\rhd m%
\end{array}%
\right. \\ 
C(t,T,U)(l,\sigma ) & = & \left\{ 
\begin{array}{ll}
T(l,\sigma ) & \text{if }t\sigma \text{ evaluates to }true \\ 
U(l,\sigma ) & \text{if }t\sigma \text{ does not evalute to }true%
\end{array}%
\right.%
\end{array}%
\end{equation*}%
where $T$ and $U$ are match trees, $f$ and $g$ are function symbols, $l$ and 
$m$ are sequences of terms and $t$ and $t_{i}$ are terms. The evaluation of
a match tree $T$ in a single term $t$ with substitution $\sigma $ is defined
as $T([t],\sigma )$.

\subsection{Building a match tree}

Let $\alpha $ be a rewrite rule given by $l\rightarrow r$.Then we define the
match tree $match\_tree(\alpha )=\gamma ([l],\{r^{\alpha }\},\emptyset )$,
where $\gamma $ is defined as:

\begin{equation*}
\begin{array}{lll}
\gamma ([],Q,V) & = & R(Q) \\ 
\gamma (x\rhd s,Q,V) & = & \left\{ 
\begin{array}{ll}
S(x,N(\gamma (s,Q,V\cup \{x\}))) & \text{if }x\notin V \\ 
M(x,N(\gamma (s,Q,V\cup \{x\})),X) & \text{if }x\in V%
\end{array}%
\right. \\ 
\gamma (f(t_{1},\cdots ,t_{n})\rhd s,Q,V) & = & F(f,\gamma (t_{1}\rhd \cdots
\rhd t_{n}\rhd s,Q,V),X)%
\end{array}%
\end{equation*}%
Match trees are only defined for rewrite rules with simple terms at the left
hand side.

\subsection{Joining match trees}

Two match trees $left$ and $right$ can be joined into one using the operator 
$||$, which is defined as follows: $left\ ||\ right=$%
\begin{equation*}
\begin{array}{lllll}
right & \text{if} & head(left)=X &  &  \\ 
left & \text{if} & head(right)=X &  &  \\ 
E(left,right) & \text{if} & head(right)=R &  &  \\ 
E(right,left) & \text{if} & head(left)=R &  &  \\ 
R(Q\cup Q^{\prime }) & \text{if} & left=R(Q) & \text{and} & 
right=R(Q^{\prime }) \\ 
S(x,T\ ||\ right) & \text{if} & left=S(x,T) & \text{and} & head(right)\in
\{F,S,U\} \\ 
M(y,left\ ||\ U,left) & \text{if} & left=S(x,T) & \text{and} & right=M(y,U,V)
\\ 
M(x,T\ ||\ right,T^{\prime }\ ||\ right) & \text{if} & left=M(x,T,T^{\prime
}) & \text{and} & head(right)\in \{F,M,N,S\} \\ 
S(x,left\ ||\ U) & \text{if} & left=F(f,T,T^{\prime }) & \text{and} & 
right=S(x,U) \\ 
M(x,left\ ||\ U,left) & \text{if} & left=F(f,T,T^{\prime }) & \text{and} & 
right=M(x,U,U^{\prime }) \\ 
F(f,T\ ||\ U,T^{\prime }) & \text{if} & left=F(f,T,T^{\prime }) & \text{and}
& right=F(f,U,U^{\prime }) \\ 
F(f,T,T^{\prime }\ ||\ right) & \text{if} & left=F(f,T,T^{\prime }) & \text{%
and} & right=F(g,U,U^{\prime }),~f\neq g \\ 
F(f,T\ ||\ N(ar(f),U),T^{\prime }\ ||\ right) & \text{if} & 
left=F(f,T,T^{\prime }) & \text{and} & right=N(U) \\ 
S(x,left\ ||\ U) & \text{if} & left=N(T) & \text{and} & right=S(x,U) \\ 
M(x,left\ ||\ U,left\ ||\ U^{\prime }) & \text{if} & left=N(T) & \text{and}
& right=M(x,U,U^{\prime }) \\ 
F(f,N(ar(f),T)\ ||\ U,left) & \text{if} & left=N(T) & \text{and} & 
right=F(f,U,X) \\ 
N(T\ ||\ U) & \text{if} & left=N(T) & \text{and} & right=N(U) \\ 
E(T,right\ ||\ T^{\prime }) & \text{if} & left=E(T,T^{\prime }) & \text{and}
& head(right)\in \{F,M,N,R,S\},%
\end{array}%
\end{equation*}%
where $head$ is defined as $head(F(f,T,U))=F$, $head(R(Q))=R$ etc.

\subsection{Optimizing match trees}

The result of joining match trees is often not optimal. This section gives
two algorithms $reduce$ and $clean$ to optimize match trees.%
\begin{equation*}
\begin{array}{l}
\begin{array}{llll}
reduce(X) & = & X &  \\ 
reduce(F(f,T,U)) & = & reduce_{F}(F(f,T,U),\emptyset ) &  \\ 
reduce(S(x,T)) & = & reduce_{S}(S(x,T),\emptyset ) &  \\ 
reduce(M(x,T,U)) & = & reduce_{M}(M(x,T,U),\emptyset ,\emptyset ) &  \\ 
reduce(C(t,T,U)) & = & C(t,reduce(T),reduce(U)) &  \\ 
reduce(N(n,T)) & = & N(n,reduce(T)) &  \\ 
reduce(E(T,U)) & = & E(t,reduce(T),reduce(U)) &  \\ 
reduce(R(Q)) & = & R(Q) &  \\ 
&  &  &  \\ 
reduce_{F}(X,F) & = & F &  \\ 
reduce_{F}(F(f,T,U),F) & = & reduce_{F}(U,F) & \text{if }f\in F \\ 
reduce_{F}(F(f,T,U)) & = & F(f,reduce(T),reduce_{F}(U,F\cup \{f\})) & \text{%
if }f\notin F \\ 
reduce_{F}(N(n,T)) & = & reduce_{M}(M(x,T,U),\emptyset ,\emptyset ) &  \\ 
&  &  &  \\ 
reduce_{S}(X,\emptyset ) & = & X &  \\ 
reduce_{S}(X,\{x\}\cup V) & = & S(x,reduce(X[x/V],\emptyset )) &  \\ 
reduce_{S}(F(f,T,U),\emptyset ) & = & reduce_{F}(F(f,T,U),\emptyset ) &  \\ 
reduce_{S}(F(F,T,U),\{x\}\cup V) & = & S(x,reduce_{F}(F(f,T,U)[x/V],%
\emptyset ) &  \\ 
reduce_{S}(S(x,T),V) & = & reduce_{S}(T,V\cup \{x\}) &  \\ 
reduce_{S}(N(n,T),\emptyset ) & = & reduce(N(n,T),\emptyset ) &  \\ 
reduce_{S}(N(n,T),\{x\}\cup V) & = & S(x,reduce(N(n,T)[x/V])) &  \\ 
&  &  &  \\ 
reduce_{M}(X,M_{t},M_{f}) & = & reduce(X) &  \\ 
reduce_{M}(F(f,T,U),M_{t},M_{f}) & = & reduce_{F}(F(f,T,U),\emptyset ) &  \\ 
reduce_{M}(S(x,T),M_{t},M_{f}) & = & reduce_{S}(S(x,T),\emptyset ) &  \\ 
reduce_{M}(M(x,T,U),M_{t},M_{f}) & = & reduce_{M}(T,M_{t},M_{f}) & \text{if }%
x\in M_{t} \\ 
reduce_{M}(M(x,T,U),M_{t},M_{f}) & = & reduce_{M}(U,M_{t},M_{f}) & \text{if }%
x\in M_{f} \\ 
reduce_{M}(M(x,T,U),M_{t},M_{f}) & = & M(x,reduce_{M}(T,M_{t}\cup
\{x\},M_{f}), & \text{if }x\notin M_{t}\wedge x\notin M_{f} \\ 
&  & reduce_{M}(U,M_{t}\cup \{x\},M_{f}\cup \{x\})) &  \\ 
reduce_{M}(N(n,T)) & = & reduce(N(n,T)), & 
\end{array}%
\end{array}%
\end{equation*}%
with%
\begin{equation*}
\begin{array}{llll}
X[x/V] & = & X &  \\ 
F(f,T,U)[x/V] & = & F(f,T[x/V],U[x/V]) &  \\ 
S(x,T)[y/V] & = & S(x,T[y/(V\setminus \{x\})]) &  \\ 
M(x,T,U)[y/V] & = & M(y,T[y/V],U[y/V]) & \text{if }x\in V \\ 
M(x,T,U)[y/V] & = & M(x,T[y/V],U[y/V]) & \text{if }x\notin V \\ 
C(t,T,U)[x/V] & = & C(t[x/y:y\in V],T[x/V],U[x/V) &  \\ 
N(n,T)[x/V] & = & N(n,T[x/V]) &  \\ 
E(T,U)[x/V] & = & E(t,T[x/V],U[x/V]) &  \\ 
R(Q)[x/V] & = & R(Q[x/y:y\in V]) & 
\end{array}%
\end{equation*}%
$%
\begin{array}{llll}
clean(T) & = & T^{\prime } & \text{if }\chi (T)=\left\langle T^{\prime
},V\right\rangle ,%
\end{array}%
$

where $\chi $ is defined as%
\begin{equation*}
\begin{array}{llll}
\chi (X) & = & \left\langle X,\emptyset \right\rangle &  \\ 
\chi (F(f,T,U)) & = & \left\langle F(f,T^{\prime },U^{\prime }),V\cup
W\right\rangle & \text{if }\chi (T)=\left\langle T^{\prime },V\right\rangle
\wedge \chi (U)=\left\langle U^{\prime },W\right\rangle \\ 
\chi (S(x,T)) & = & \left\langle S(x,T^{\prime }),V\setminus
\{x\}\right\rangle & \text{if }\chi (T)=\left\langle T^{\prime
},V\right\rangle \wedge x\in V \\ 
\chi (S(x,T)) & = & \left\langle T^{\prime },V\right\rangle & \text{if }\chi
(T)=\left\langle T^{\prime },V\right\rangle \wedge x\notin V \\ 
\chi (M(x,T,U)) & = & \left\langle T^{\prime },V\right\rangle & \text{if }%
\chi (T)=\left\langle T^{\prime },V\right\rangle \wedge \chi
(U)=\left\langle U^{\prime },W\right\rangle \wedge T^{\prime }=U^{\prime }
\\ 
\chi (M(x,T,U)) & = & \left\langle M(x,T^{\prime },U^{\prime }),V\cup W\cup
\{x\}\right\rangle & \text{if }\chi (T)=\left\langle T^{\prime
},V\right\rangle \wedge \chi (U)=\left\langle U^{\prime },W\right\rangle
\wedge T^{\prime }\neq U^{\prime } \\ 
\chi (C(t,T,U)) & = & \left\langle T^{\prime },V\right\rangle & \text{if }%
\chi (T)=\left\langle T^{\prime },V\right\rangle \wedge \chi
(U)=\left\langle U^{\prime },W\right\rangle \wedge T^{\prime }=U^{\prime }
\\ 
\chi (C(t,T,U)) & = & \left\langle C(t,T^{\prime },U^{\prime }),V\cup W\cup
var(t)\right\rangle & \text{if }\chi (T)=\left\langle T^{\prime
},V\right\rangle \wedge \chi (U)=\left\langle U^{\prime },W\right\rangle
\wedge T^{\prime }\neq U^{\prime } \\ 
\chi (N(n,T)) & = & \left\langle N(n,T^{\prime }),V\right\rangle & \text{if }%
\left\langle T^{\prime },V\right\rangle =\chi (T) \\ 
\chi (E(T,U)) & = & \left\langle T^{\prime },V\right\rangle & \text{if }\chi
(T)=\left\langle T^{\prime },V\right\rangle \wedge \chi (U)=\left\langle
X,W\right\rangle \\ 
\chi (E(T,U)) & = & \left\langle U^{\prime },W\right\rangle & \text{if }\chi
(T)=\left\langle X,V\right\rangle \wedge \chi (U)=\left\langle U^{\prime
},W\right\rangle \\ 
\chi (E(T,U)) & = & \left\langle E(T^{\prime },U^{\prime }),V\cup
W\right\rangle & \text{if }\chi (T)=\left\langle T^{\prime },V\right\rangle
\wedge \chi (U)=\left\langle U^{\prime },W\right\rangle \wedge T^{\prime
}\neq X\wedge U^{\prime }\neq X \\ 
\chi (R(Q)) & = & \left\langle R(Q),var(Q)\right\rangle & 
\end{array}%
\end{equation*}

\subsection{Prioritized rewrite rules}

By adding priorities to rewrite rules, the selection of rewrite rules that
is considered for a term can be reduced. We model priorities of rewrite
rules using a function $\varphi $, that returns the rules of highest
priority for a set of rules. So $\varphi (R)\subseteq R$ and $\varphi
(R)=\emptyset $ if and only if $R=\emptyset $. We define a function $prior$
that applies a priority function $\varphi $ to a match tree:%
\begin{equation*}
\begin{array}{lll}
prior(X,\varphi ) & = & X \\ 
prior(F(f,T,U),\varphi ) & = & F(f,prior(T,\varphi ),prior(U,\varphi )) \\ 
prior(S(x,T),\varphi ) & = & S(x,prior(T,\varphi )) \\ 
prior(M(x,T,U),\varphi ) & = & M(x,prior(T,\varphi ),prior(U,\varphi )) \\ 
prior(C(t,T,U),\varphi ) & = & C(t,prior(T,\varphi ),prior(U,\varphi )) \\ 
prior(N(n,T),\varphi ) & = & N(n,prior(T,\varphi )) \\ 
prior(R(Q),\varphi ) & = & R(\varphi (Q)) \\ 
prior(E(T,U),\varphi ) & = & E(prior(T,\varphi ),prior(U,\varphi )).%
\end{array}%
\end{equation*}%
The effect of applying $prior$ to a match tree is that the $R$-nodes will
contain less elements. This can be useful to remove unwanted results.
Consider for example the rewrite system%
\begin{equation*}
\left\{ 
\begin{array}{ccc}
x=x & \rightarrow & true \\ 
x=y & \rightarrow & false%
\end{array}%
\right.
\end{equation*}%
This system can have both $true$ and $false$ as a result of rewriting the
term $true=true$. But if we give the first equation a higher priority than
the second, the undesired derivation $true=true\rightarrow false$ is
eliminated.

\section{Rewriting}

In this section we describe rewriting strategies. For the moment we only
consider innermost rewriting.

\subsection{Rewriting using match trees}

Suppose that we have a rewrite system, and that for each function symbol $f$
a match tree $M_{f}$ has been constructed that corresponds to rewrite rules
with head symbol $f$. We define the function $rewr_{M}$ as:%
\begin{equation*}
\begin{array}{lll}
rewr_{M}(x,\sigma ) & = & \sigma (x) \\ 
rewr_{M}(f,\sigma ) & = & \left\{ 
\begin{array}{ll}
f & \text{if }M_{f}([f],\sigma )=\emptyset \\ 
u\in M_{f}([f],\sigma ) & \text{if }M_{f}([f],\sigma )\neq \emptyset%
\end{array}%
\right. \\ 
rewr_{M}(f(t_{1},\cdots ,t_{n}),\sigma ) & = & \left\{ 
\begin{array}{ll}
f(t_{1},\cdots ,t_{n}) & \text{if }M_{f}([f(t_{1},\cdots ,t_{n})],\sigma
)=\emptyset \\ 
u\in M_{f}([f(t_{1},\cdots ,t_{n})],\sigma ) & \text{if }M_{f}([f(t_{1},%
\cdots ,t_{n})],\sigma )\neq \emptyset%
\end{array}%
\right. \\ 
rewr_{M}(x(t_{1},\cdots ,t_{n}),\sigma ) & = & x(t_{1},\cdots ,t_{n}) \\ 
rewr_{M}(u(u_{1},\cdots ,u_{m})(t_{1},\cdots ,t_{n}),\sigma ) & = & 
u(u_{1},\cdots ,u_{m})(t_{1},\cdots ,t_{n})%
\end{array}%
\end{equation*}

\subsection{Innermost rewriting}

We now define an algorithm $rewr_{I}$ for innermost rewriting. It is defined
for applicative terms. We assume that $\sigma (x)$ is always in normal form
already.%
\begin{equation*}
\begin{array}{lll}
rewr_{I}(x,\sigma ) & = & \sigma (x) \\ 
rewr_{I}(f,\sigma ) & = & rewr_{M}(f,\sigma ) \\ 
rewr_{I}(t(t_{1},\cdots ,t_{n}),\sigma ) & = & rewr_{M}(rewr_{I}(t,\sigma
)(rewr_{I}(t_{1},\sigma ),\cdots ,rewr_{I}(t_{n},\sigma )),\sigma )%
\end{array}%
\end{equation*}

\section{Further work}

\begin{itemize}
\item Extend the definition of terms with lambda expressions and quantifier
expressions, and extend the algorithms so they can handle them.

\item Design an algorithm for rewriting using strategies as defined in \cite%
{weerdenburg2009}.

\item Extend the rewrite algorithms so they handle evaluation of conditions
(as is required in the evaluation of a $C$-node).

\item Extend the rewrite algorithms for rewrite rules with more general left
hand sides.

\item Collect examples of higher order rewrite systems for testing the
algorithms.
\end{itemize}

\bibliographystyle{alpha}
\bibliography{rewriter}

\end{document}
