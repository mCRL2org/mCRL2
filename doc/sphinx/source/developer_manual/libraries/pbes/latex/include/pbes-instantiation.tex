%TCIDATA{Version=5.50.0.2890}
%TCIDATA{LaTeXparent=1,1,pbes-implementation-notes.tex}


\section{PBES instantiation}

In this section we describe several implementations of instantiation
algorithms for PBESs.

\subsection{Finite algorithm}

In this section we describe an implementation of the finite instantiation
algorithm \textsc{Pbes2besFinite} that eliminates data parameters with
finite sorts. It is implemented in the tool \textbf{pbesinst}. Let $\mathcal{%
E=(\sigma }_{1}X_{1}(d_{1}:D_{1},e_{1}:E_{1})=\varphi _{1})\cdots \mathcal{%
(\sigma }_{n}X_{n}(d_{n}:D_{n},e_{n}:E_{n})=\varphi _{n})$ be a PBES. We
assume that all data sorts $D_{i}$ are finite and all data sorts $E_{i}$ are
infinite. Let $r$ be a data rewriter, and let $\rho $ be an injective
function that creates a unique predicate variable from a predicate variable
name and a data value according to $\rho (X(d:D,e:E),d_{0})\rightarrow
Y(e:E) $, where $D$ is finite and $E$ is infinite and $d_{0}\in D$. Note
that $D$ and $D_{i}$ may be multi-dimensional sorts.%
\begin{equation*}
\begin{array}{l}
\text{\textsc{Pbes2besFinite(}}\mathcal{E}\text{, }r\text{, }\rho \text{%
\textsc{)}} \\
\text{\textbf{for }}i:=1\cdots n\text{ \textbf{do}} \\
\qquad \mathcal{E}_{i}:=\{\mathcal{\sigma }_{i}\rho (X_{i},d)=R(\varphi
_{k}[d_{k}:=d])\ |\ d\in D_{i}\} \\
\text{\textbf{return }}\mathcal{E}_{1}\cdots \mathcal{E}_{n},%
\end{array}%
\end{equation*}%
with $R$ a rewriter on pbes expressions that is defined as follows:%
\begin{eqnarray*}
R(b) &=&b \\
R(\lnot \varphi ) &=&\lnot R(\varphi ) \\
R(\varphi \oplus \psi ) &=&R(\varphi )\oplus R(\psi ) \\
R(X_{i}(d,e)) &=&\left\{
\begin{array}{cc}
\rho (X_{i},r(d))(r(e)) & \text{if }FV(d)=\emptyset \\
\dbigvee\limits_{d_{i}\in D_{i}}r(d=d_{i})\wedge \rho (X_{i},d_{i})(r(e)) &
\text{if }FV(d)\neq \emptyset%
\end{array}%
\right. \\
R(\forall _{d:D}.\varphi ) &=&\forall _{d:D}.R(\varphi ) \\
R(\exists _{d:D}.\varphi ) &=&\exists _{d:D}.R(\varphi )
\end{eqnarray*}%
where $\oplus \in \{\vee ,\wedge ,\Rightarrow \}$, $b$ a data expression and
$\varphi $ and $\psi $ pbes expressions and $FV(d)$ is the set of free
variables appearing in $d$.\newpage

\subsection{Lazy algorithm}

In this section we describe an implementation of the lazy instantiation
algorithm \textsc{Pbes2besLazy} that uses instantiation to compute a BES. It
is implemented in the tool \textbf{pbesinst}. It takes two extra parameters,
an injective function $\rho $ that renames proposition variables to
predicate variables, and a rewriter $R$ that eliminates quantifiers from
predicate formulae. Let $\mathcal{E=(\sigma }_{1}X_{1}(d_{1}:D_{1})=\varphi
_{1})\ldots \mathcal{(\sigma }_{n}X_{n}(d_{n}:D_{n})=\varphi _{n})$ be a
PBES, and $X_{init}(e_{init})$ an initial state.%
\begin{equation*}
\begin{array}{l}
\text{\textsc{Pbes2besLazy(}}\mathcal{E}\text{, }X_{init}(e_{init})\text{, }R%
\text{, }\rho \text{\textsc{)}} \\
\text{\textbf{for }}i:=1\cdots n\text{ \textbf{do }}\mathcal{E}_{i}:=\epsilon
\\
todo:=\{R(X_{init}(e_{init}))\} \\
done:=\emptyset \\
\text{\textbf{while }}todo\neq \emptyset \text{ \textbf{do}} \\
\qquad \text{\textbf{choose }}X_{k}(e)\in todo \\
\qquad todo:=todo\ \backslash \ \{X_{k}(e)\} \\
\qquad done:=done\cup \{X_{k}(e)\} \\
\qquad X^{e}:=\rho (X_{k}(e)) \\
\qquad \psi ^{e}:=R(\varphi _{k}[d_{k}:=e]) \\
\qquad \mathcal{E}_{k}:=\mathcal{E}_{k}(\mathcal{\sigma }_{k}X^{e}=\rho
(\psi ^{e})) \\
\qquad todo:=todo\cup \{Y(f)\in \mathsf{occ}(\psi ^{e})\ |\ Y(f)\notin done\}
\\
\text{\textbf{return }}\mathcal{E}_{1}\cdots \mathcal{E}_{n},%
\end{array}%
\end{equation*}%
where $\rho $ is extended from predicate variables to quantifier free
predicate formulae using

\begin{eqnarray*}
\rho (b) &=&b \\
\quad \rho (\varphi \oplus \psi ) &=&\rho (\varphi )\oplus \rho (\psi )
\end{eqnarray*}%
\newpage

\subsection{Alternative lazy algorithm}

Another version of the lazy algorithm named \textsc{Pbes2besAlternativeLazy}
has been implemented in the tool \textbf{pbesinst}, but for the current
implementation no specification is available. It is also used in the tool
\textbf{pbes2bool}. In 2015 Xiao Qi has made a specification and a
corresponding implementation of \textsc{Pbes2besAlternativeLazy}. However,
since then a lot of changes have been made to the implementation, without
updating the specification. Below some specifications made by Xiao Qi are
included, that are still relevant.

The subroutine \textsc{FindLoop} attempts to find a $\mu $- or $\nu $-loop
within all generated equations on the same rank of $X_{k}$. The map $\mathrm{%
visited}$, intially empty, is used both as a mechanism to ensure termination
and as a memoization table. It is passed by reference. Due to the fact that
\textsc{FindLoop} only considers generated equations, loops are only found
after all variable instantiations on the loop have been generated.

\begin{equation*}
\begin{array}{l}
\text{\textsc{FindLoop}}(\sigma ,\varphi ,X_{k}(e),\mathrm{equations},%
\mathrm{visited}) \\
\text{\textbf{if }}\varphi \in \{true,false\}\text{ \textbf{then }}\text{%
\textbf{return }}false \\
\text{\textbf{if }}\varphi =X_{l}(f)\text{ \textbf{then }} \\
\qquad \text{\textbf{if }}X_{l}(f)=X_{k}(e)\text{ \textbf{then }}\text{%
\textbf{return }}true \\
\qquad \text{\textbf{if }}\mathrm{rank}(X_{l})\neq \mathrm{rank}(X_{k})\text{
\textbf{then }}\text{\textbf{return }}false \\
\qquad \text{\textbf{if }}X_{l}(f)\in keys(\mathrm{visited})\text{ \textbf{%
then }}\text{\textbf{return }}\mathrm{visited}(X_{l}(f)) \\
\qquad \text{\textbf{if }}X_{l}(f)\in keys(\mathrm{equations})\text{ \textbf{%
then }} \\
\qquad \qquad \psi _{f}:=\mathrm{equations}(X_{l}(f)) \\
\qquad \qquad \mathrm{visited}(X_{l}(f)):=false \\
\qquad \qquad b:=\text{\textsc{FindLoop}}(\sigma ,\psi _{f},Y(f),\mathrm{%
equations},\mathrm{visited}) \\
\qquad \qquad \mathrm{visited}(X_{l}(f)):=b \\
\qquad \qquad \text{\textbf{return }}b \\
\qquad \text{\textbf{return }}false \\
\text{\textbf{if }}\sigma =\mu \text{ \textbf{then }} \\
\qquad \text{\textbf{if }}\varphi =\psi \wedge \chi \text{ \textbf{then }}
\\
\qquad \qquad \text{\textbf{return }}\text{\textsc{FindLoop}}(\sigma ,\psi
,X_{k}(e),\mathrm{equations},\mathrm{visited}))\vee \text{\textsc{FindLoop}}%
(\sigma ,\chi ,X_{k}(e),\mathrm{equations},\mathrm{visited})) \\
\qquad \text{\textbf{if }}\varphi =\psi \vee \chi \text{ \textbf{then }} \\
\qquad \qquad \text{\textbf{return }}\text{\textsc{FindLoop}}(\sigma ,\psi
,X_{k}(e),\mathrm{equations},\mathrm{visited}))\wedge \text{\textsc{FindLoop}%
}(\sigma ,\chi ,X_{k}(e),\mathrm{equations},\mathrm{visited})) \\
\text{\textbf{if }}\sigma =\nu \text{ \textbf{then}} \\
\qquad \text{\textbf{if }}\varphi =\psi \wedge \chi \text{ \textbf{then }}
\\
\qquad \qquad \text{\textbf{return }}\text{\textsc{FindLoop}}(\sigma ,\psi
,X_{k}(e),\mathrm{equations},\mathrm{visited}))\wedge \text{\textsc{FindLoop}%
}(\sigma ,\chi ,X_{k}(e),\mathrm{equations},\mathrm{visited})) \\
\qquad \text{\textbf{if }}\varphi =\psi \vee \chi \text{ \textbf{then }} \\
\qquad \qquad \text{\textbf{return }}\text{\textsc{FindLoop}}(\sigma ,\psi
,X_{k}(e),\mathrm{equations},\mathrm{visited}))\vee \text{\textsc{FindLoop}}%
(\sigma ,\chi ,X_{k}(e),\mathrm{equations},\mathrm{visited}))%
\end{array}%
\end{equation*}

The subroutine \textsc{BackSubstitute} subsitutes the value of $X_{k}(e)$,
which is either $true$ or $false$, to all its occurences, and do this
recursively for all new variable instantiations that are found trivial. The
argument $\mathrm{equations}$ is passed by reference:%
\begin{equation*}
\begin{array}{l}
\text{\textsc{BackSubstitute}}(X_{k}(e),\mathrm{equations})\mathrm{%
tosubstitute}:=\{X_{k}(e)\} \\
\text{\textbf{while }}\mathrm{tosubstitute}\neq \emptyset \text{ \textbf{do }%
} \\
\qquad \text{\textbf{choose }}X_{l}(f)\in \mathrm{tosubstitute} \\
\qquad \mathrm{tosubstitute}:=\mathrm{tosubstitute}\backslash \{X_{l}(f)\}
\\
\qquad \text{\textbf{foreach }}(Z(g),\varphi )\in \mathrm{equations} \\
\qquad \qquad \text{\textbf{if }}X_{l}(f)\in \mathrm{occ}(\varphi )\text{
\textbf{then }} \\
\qquad \qquad \qquad \varphi :=\text{\textsc{Simplify}}(\varphi \lbrack
X_{l}(f):=\mathrm{equations}(X_{l}(f))]) \\
\qquad \qquad \qquad \mathrm{equations}(Z(g)):=\varphi \\
\qquad \qquad \qquad \text{\textbf{if }}\varphi \in \{true,false\}\text{
\textbf{then }} \\
\qquad \qquad \qquad \qquad \mathrm{tosubstitute}:=\mathrm{tosubstitute}\cup
Z(g)%
\end{array}%
\end{equation*}

The subroutine \textsc{RegenerateStates} re-explores the reachable state
space. The arguments $\mathrm{todo}$ and $\mathrm{reachable}$ are passed by
reference:

\begin{equation*}
\begin{array}{l}
\text{\textsc{RegenerateStates}}(\mathrm{todo},\mathrm{reachable},\mathrm{%
done},\mathrm{equations}) \\
\mathrm{todo}:=\emptyset \\
\mathrm{reachable}:=\emptyset \\
\mathrm{grey}:=\{Xinit(einit)\} \\
\text{\textbf{while }}grey\neq \emptyset \\
\qquad \text{\textbf{choose }}X_{k}(e)\in \mathrm{grey} \\
\qquad \mathrm{reachable}:=\mathrm{reachable}\cup X_{k}(e) \\
\qquad \text{\textbf{if }}X_{k}(e)\in \mathrm{done}\text{ \textbf{then }} \\
\qquad \qquad \mathrm{grey}:=\mathrm{grey}\cup \mathrm{occ}(\mathrm{equations%
}(X_{k}(e))) \\
\qquad \text{\textbf{else }} \\
\qquad \qquad \mathrm{todo}:=\mathrm{todo}\cup X_{k}(e)%
\end{array}%
\end{equation*}%
\newpage

\subsection{Lazy algorithm (structure graph)}

There is a third version of the lazy algorithm named \textsc{%
Pbes2besStructureGraph} that produces a structure graph (parity game)
instead of a BES. It is implemented in the tool \textbf{pbessolve}.

\subsubsection{Structure graph definition}

A structure graph is a graph $(V,E)$ with $V$ a set of BES variables. On
this graph a partial function $r:V\rightarrow \mathbb{N}$ is defined that
assigns a rank to each node, and a partial function $d:V\rightarrow
\{\blacktriangle ,\blacktriangledown ,\top ,\bot \}$ is defined that assigns
a decoration to each node. A structure graph is formally defined using the
following SOS rules:%
\begin{equation*}
\frac{X\in bnd(\mathcal{E})}{\left\langle X,\mathcal{E}\right\rangle
\pitchfork rank_{\mathcal{E}}(X)}
\end{equation*}

\begin{equation*}
\frac{{}}{\left\langle true,\mathcal{E}\right\rangle \top }\qquad \frac{{}}{%
\left\langle false,\mathcal{E}\right\rangle \bot }
\end{equation*}%
\begin{equation*}
\frac{{}}{\left\langle f\wedge f^{\prime },\mathcal{E}\right\rangle
\blacktriangle }\qquad \frac{{}}{\left\langle f\vee f^{\prime },\mathcal{E}%
\right\rangle \blacktriangledown }
\end{equation*}%
\begin{equation*}
\frac{\left\langle f,\mathcal{E}\right\rangle \blacktriangle \quad \lnot
\left\langle f,\mathcal{E}\right\rangle \pitchfork \quad \left\langle f,%
\mathcal{E}\right\rangle \rightarrow \left\langle g,\mathcal{E}\right\rangle
}{\left\langle f\wedge f^{\prime },\mathcal{E}\right\rangle \rightarrow
\left\langle g,\mathcal{E}\right\rangle }\qquad \frac{\left\langle f^{\prime
},\mathcal{E}\right\rangle \blacktriangle \quad \lnot \left\langle f^{\prime
},\mathcal{E}\right\rangle \pitchfork \quad \left\langle f^{\prime },%
\mathcal{E}\right\rangle \rightarrow \left\langle g^{\prime },\mathcal{E}%
\right\rangle }{\left\langle f\wedge f^{\prime },\mathcal{E}\right\rangle
\rightarrow \left\langle g^{\prime },\mathcal{E}\right\rangle }
\end{equation*}%
\begin{equation*}
\frac{\left\langle f,\mathcal{E}\right\rangle \blacktriangledown \quad \lnot
\left\langle f,\mathcal{E}\right\rangle \pitchfork \quad \left\langle f,%
\mathcal{E}\right\rangle \rightarrow \left\langle g,\mathcal{E}\right\rangle
}{\left\langle f\vee f^{\prime },\mathcal{E}\right\rangle \rightarrow
\left\langle g,\mathcal{E}\right\rangle }\qquad \frac{\left\langle f^{\prime
},\mathcal{E}\right\rangle \blacktriangledown \quad \lnot \left\langle
f^{\prime },\mathcal{E}\right\rangle \pitchfork \quad \left\langle f^{\prime
},\mathcal{E}\right\rangle \rightarrow \left\langle g^{\prime },\mathcal{E}%
\right\rangle }{\left\langle f\vee f^{\prime },\mathcal{E}\right\rangle
\rightarrow \left\langle g^{\prime },\mathcal{E}\right\rangle }
\end{equation*}%
\begin{equation*}
\frac{\lnot \left\langle f,\mathcal{E}\right\rangle \blacktriangle }{%
\left\langle f\wedge f^{\prime },\mathcal{E}\right\rangle \rightarrow
\left\langle f,\mathcal{E}\right\rangle }\qquad \frac{\lnot \left\langle
f^{\prime },\mathcal{E}\right\rangle \blacktriangle }{\left\langle f\wedge
f^{\prime },\mathcal{E}\right\rangle \rightarrow \left\langle f^{\prime },%
\mathcal{E}\right\rangle }
\end{equation*}%
\begin{equation*}
\frac{\lnot \left\langle f,\mathcal{E}\right\rangle \blacktriangledown }{%
\left\langle f\vee f^{\prime },\mathcal{E}\right\rangle \rightarrow
\left\langle f,\mathcal{E}\right\rangle }\qquad \frac{\lnot \left\langle
f^{\prime },\mathcal{E}\right\rangle \blacktriangledown }{\left\langle f\vee
f^{\prime },\mathcal{E}\right\rangle \rightarrow \left\langle f^{\prime },%
\mathcal{E}\right\rangle }
\end{equation*}%
\begin{equation*}
\frac{\left\langle f,\mathcal{E}\right\rangle \pitchfork n}{\left\langle
f\wedge f^{\prime },\mathcal{E}\right\rangle \rightarrow \left\langle f,%
\mathcal{E}\right\rangle }\qquad \frac{\left\langle f^{\prime },\mathcal{E}%
\right\rangle \pitchfork n}{\left\langle f\wedge f^{\prime },\mathcal{E}%
\right\rangle \rightarrow \left\langle f,\mathcal{E}\right\rangle }
\end{equation*}%
\begin{equation*}
\frac{\left\langle f,\mathcal{E}\right\rangle \pitchfork n}{\left\langle
f\vee f^{\prime },\mathcal{E}\right\rangle \rightarrow \left\langle f,%
\mathcal{E}\right\rangle }\qquad \frac{\left\langle f^{\prime },\mathcal{E}%
\right\rangle \pitchfork n}{\left\langle f\vee f^{\prime },\mathcal{E}%
\right\rangle \rightarrow \left\langle f,\mathcal{E}\right\rangle }
\end{equation*}%
\begin{equation*}
\frac{\sigma X=f\in \mathcal{E}\quad \left\langle f,\mathcal{E}\right\rangle
\blacktriangle \quad \lnot \left\langle f,\mathcal{E}\right\rangle
\pitchfork }{\left\langle X,\mathcal{E}\right\rangle \blacktriangle }\qquad
\frac{\sigma X=f\in \mathcal{E}\quad \left\langle f,\mathcal{E}\right\rangle
\blacktriangledown \quad \lnot \left\langle f,\mathcal{E}\right\rangle
\pitchfork }{\left\langle X,\mathcal{E}\right\rangle \blacktriangledown }
\end{equation*}%
\begin{equation*}
\frac{\sigma X=f\in \mathcal{E}\quad \lnot \left\langle f,\mathcal{E}%
\right\rangle \blacktriangledown \quad \lnot \left\langle f,\mathcal{E}%
\right\rangle \blacktriangle }{\left\langle X,\mathcal{E}\right\rangle
\rightarrow \left\langle f,\mathcal{E}\right\rangle }\qquad \frac{\sigma
X=f\in \mathcal{E}\quad \left\langle f,\mathcal{E}\right\rangle \pitchfork }{%
\left\langle X,\mathcal{E}\right\rangle \rightarrow \left\langle f,\mathcal{E%
}\right\rangle }
\end{equation*}%
\begin{equation*}
\frac{\sigma X=f\in \mathcal{E}\quad \left\langle f,\mathcal{E}\right\rangle
\rightarrow \left\langle g,\mathcal{E}\right\rangle \quad \left\langle f,%
\mathcal{E}\right\rangle \blacktriangle \quad \lnot \left\langle f,\mathcal{E%
}\right\rangle \pitchfork }{\left\langle X,\mathcal{E}\right\rangle
\rightarrow \left\langle g,\mathcal{E}\right\rangle }
\end{equation*}%
\begin{equation*}
\frac{\sigma X=f\in \mathcal{E}\quad \left\langle f,\mathcal{E}\right\rangle
\rightarrow \left\langle g,\mathcal{E}\right\rangle \quad \left\langle f,%
\mathcal{E}\right\rangle \blacktriangledown \quad \lnot \left\langle f,%
\mathcal{E}\right\rangle \pitchfork }{\left\langle X,\mathcal{E}%
\right\rangle \rightarrow \left\langle g,\mathcal{E}\right\rangle }
\end{equation*}%
First we rewrite that into a more explicit form, and remove the $%
\left\langle ,\mathcal{E}\right\rangle $ annotation:%
\begin{equation*}
\frac{X\in bnd(\mathcal{E})}{X\pitchfork rank_{\mathcal{E}}(X)}
\end{equation*}

\begin{equation*}
\frac{{}}{true\top }\qquad \frac{{}}{false\bot }
\end{equation*}%
\begin{equation*}
\frac{{}}{\left( f\wedge f^{\prime }\right) \blacktriangle }\qquad \frac{{}}{%
\left( f\vee f^{\prime }\right) \blacktriangledown }
\end{equation*}%
\begin{equation*}
\frac{f\rightarrow g}{(f\wedge f^{\prime })\rightarrow g}\qquad \frac{%
f\rightarrow g}{(f^{\prime }\wedge f)\rightarrow g}
\end{equation*}%
\begin{equation*}
\frac{f\rightarrow g}{(f\vee f^{\prime })\rightarrow g}\qquad \frac{%
f\rightarrow g}{(f^{\prime }\vee f)\rightarrow g}
\end{equation*}%
\begin{equation*}
\frac{\lnot f\blacktriangle }{f\wedge f^{\prime }\rightarrow f}\qquad \frac{%
\lnot f^{\prime }\blacktriangle }{f\wedge f^{\prime }\rightarrow f^{\prime }}
\end{equation*}%
\begin{equation*}
\frac{\lnot f\blacktriangledown }{f\vee f^{\prime }\rightarrow f}\qquad
\frac{\lnot f^{\prime }\blacktriangledown }{f\vee f^{\prime }\rightarrow
f^{\prime }}
\end{equation*}%
\begin{equation*}
\frac{{}}{X\wedge f\rightarrow X}\qquad \frac{{}}{f\wedge X\rightarrow X}
\end{equation*}%
\begin{equation*}
\frac{{}}{X\vee f\rightarrow X}\qquad \frac{{}}{f\vee X\rightarrow X}
\end{equation*}%
\begin{equation*}
\frac{\sigma X=f\wedge f^{\prime }\in \mathcal{E}}{X\blacktriangle }\qquad
\frac{\sigma X=f\vee f^{\prime }\in \mathcal{E}}{X\blacktriangledown }
\end{equation*}%
\begin{equation*}
\frac{\sigma X=Y\in \mathcal{E}}{X\rightarrow Y}\qquad \frac{\sigma X=\top
\in \mathcal{E}}{X\rightarrow \top }\qquad \frac{\sigma X=\bot \in \mathcal{E%
}}{X\rightarrow \bot }
\end{equation*}%
\begin{equation*}
\frac{\sigma X=f\wedge f^{\prime }\in \mathcal{E}\quad f\wedge f^{\prime
}\rightarrow g}{X\rightarrow g}
\end{equation*}%
\begin{equation*}
\frac{\sigma X=f\vee f^{\prime }\in \mathcal{E}\quad f\vee f^{\prime
}\rightarrow g}{X\rightarrow g}
\end{equation*}%
Note that in this definition separate nodes are created for the left hand
side $X$ and the right hand side $f$ of each equation $\sigma X=f$ . This is
undesirable, hence in the implementation below the nodes $X$ and $f$ are
merged into one node labeled with $X$.

\subsubsection{Generic lazy PBES instantiation algorithms}

In this section two generic variants of lazy PBES instantiation are described that that report all discovered BES
equations using a callback function \textsc{ReportEquation}. These algorithms contain a number of optimizations
with respect to the original version. The first version \textsc{Pbes2besLazy2} stores todo elements in a set,
and is a straightforward generalization of \textsc{Pbes2besLazy}.

\begin{equation*}
\begin{array}{l}
\text{\textsc{Pbes2besLazy2(}}\mathcal{E}\text{, }X_{init}(e_{init})\text{, }%
R\text{\textsc{)}} \\
\mathrm{init}:=R(X_{init}(e_{init})) \\
\mathrm{todo}:=\{\mathrm{init}\} \\
\mathrm{done}:=\emptyset  \\
\mathrm{equation}:=\emptyset  \\
\text{\textbf{while }}\mathrm{todo}\neq \emptyset \text{ \textbf{do}} \\
\qquad \text{\textbf{choose }}X_{k}(e)\in todo \\
\qquad \mathrm{todo}:=\mathrm{todo}\ \backslash \ \{X_{k}(e)\} \\
\qquad \mathrm{done}:=\mathrm{done}\cup \{X_{k}(e)\} \\
\qquad \psi ^{e}:=R(\varphi _{k}[d_{k}:=e]) \\
\qquad \psi ^{e}:=\text{\textsc{ForwardSubstitute}}(\psi ^{e},\mathrm{%
equation}) \\
\qquad \psi ^{e}:=\text{\textsc{SimplifyLoop}}(\psi ^{e},X_{k}(e)) \\
\qquad \mathrm{equation}(X_{k}(e)):=\psi ^{e} \\
\qquad \text{\textsc{ReportEquation}}(X_{k}(e),\psi ^{e}) \\
\qquad \mathrm{todo}:=\mathrm{todo}\cup (\mathrm{occ}(\psi ^{e})\setminus
\mathrm{done}) \\
\qquad \mathrm{equation}:=\text{\textsc{BackwardSubstitute}}(\psi
^{e},X_{k}(e),\mathrm{equation}) \\
\qquad \mathrm{todo}:=\text{\textsc{ResetTodo}}(\mathrm{init},\mathrm{todo},%
\mathrm{done},\mathrm{equation})%
\end{array}%
\end{equation*}%
The calls to \textsc{ForwardSubstitute}, \textsc{SimplifyLoop}, \textsc{%
BackwardSubstitute}, and \textsc{Reset} are optimizations that can be
omitted.

The second version \textsc{Pbes2besLazy3} stores todo elements in a double ended queue. This is done to support
breadth first and depth first search. This is achieved by choosing the first respectively the last element of the todo
queue. The choice for a double ended queue introduces a complication: the todo queue may contain duplicate
elements. This problem is solved by immediatiely adding new todo elements to the set done. This changes the
role of the set done, hence it is renamed to discovered.

\begin{equation*}
\begin{array}{l}
\text{\textsc{Pbes2besLazy3(}}\mathcal{E}\text{, }X_{init}(e_{init})\text{, }%
R\text{\textsc{)}} \\
\mathrm{init}:=R(X_{init}(e_{init})) \\
\mathrm{todo}:=[\mathrm{init}] \\
\mathrm{discovered}:=\{ \mathrm{init} \} \\
\mathrm{equation}:=\emptyset  \\
\text{\textbf{while }}\mathrm{todo}\neq \emptyset \text{ \textbf{do}} \\
\qquad \text{\textbf{choose }}X_{k}(e)\in todo \\
\qquad \mathrm{todo}:=\mathrm{todo}\ +\!\!+ \ [X_{k}(e)] \\
\qquad \mathrm{discovered}:=\mathrm{discovered}\cup \{X_{k}(e)\} \\
\qquad \psi ^{e}:=R(\varphi _{k}[d_{k}:=e]) \\
\qquad \psi ^{e}:=\text{\textsc{ForwardSubstitute}}(\psi ^{e},\mathrm{%
equation}) \\
\qquad \psi ^{e}:=\text{\textsc{SimplifyLoop}}(\psi ^{e},X_{k}(e)) \\
\qquad \mathrm{equation}(X_{k}(e)):=\psi ^{e} \\
\qquad \text{\textsc{ReportEquation}}(X_{k}(e),\psi ^{e}) \\

\qquad \text{\textbf{for }} Y_{l}(f) \in \mathrm{occ}(\psi ^{e})\setminus \mathrm{discovered} \ \text{\textbf{do }}\\
\qquad \qquad \mathrm{todo}:=\mathrm{todo}\ +\!\!+ \ [Y_{l}(f)] \\
\qquad \qquad \mathrm{discovered}:=\mathrm{discovered}\ \cup \{Y_{l}(f)\} \\

\qquad \mathrm{equation}:=\text{\textsc{BackwardSubstitute}}(\psi
^{e},X_{k}(e),\mathrm{equation}) \\
\qquad \mathrm{todo}:=\text{\textsc{ResetTodo}}(\mathrm{init},\mathrm{todo},%
\mathrm{discovered},\mathrm{equation})%
\end{array}%
\end{equation*}

\subsubsection{Pbes2besStructureGraph}

For the \textsc{Pbes2besStructureGraph} algorithm a concrete implementation of \textsc{ReportEquation}
needs to be defined. We assume there is a structure graph $G$. The callback function
\textsc{ReportEquation} extends $G$ as follows:%
\begin{equation*}
\begin{array}{l}
\text{\textsc{ReportEquation}}(X_{k}(e),\psi ^{e}) \\
G:=G\cup SG^{0}(X_{k}(e),\psi ^{e})%
\end{array}%
\end{equation*}

where $SG^{0}$ and $SG^{1}$ are defined as%
\begin{equation*}
\begin{array}{|c|c|}
\hline
\psi & SG^{0}(\varphi ,\psi ) \\ \hline\hline
true & (\{\varphi \},\emptyset ) \\ \hline
false & (\{\varphi \},\emptyset ) \\ \hline
Y & (\{\varphi ,\psi \},\{(\varphi ,\psi )\}) \\ \hline
\psi _{1}\wedge \cdots \wedge \psi _{n} & \left( \{\varphi ,\psi _{1},\cdots
,\psi _{n}\},\{(\varphi ,\psi _{1}),\cdots ,(\varphi ,\psi _{n})\}\right)
\cup \bigcup\limits_{i=1}^{n}SG^{1}(\psi _{i}) \\ \hline
\psi _{1}\vee \cdots \vee \psi _{n} & \left( \{\varphi ,\psi _{1},\cdots
,\psi _{n}\},\{(\varphi ,\psi _{1}),\cdots ,(\varphi ,\psi _{n})\}\right)
\cup \bigcup\limits_{i=1}^{n}SG^{1}(\psi _{i}) \\ \hline
\end{array}%
\end{equation*}%
\begin{equation*}
\begin{array}{|c|c|}
\hline
\psi & SG^{1}(\varphi ) \\ \hline\hline
true & (\{\psi \},\emptyset ) \\ \hline
false & (\{\psi \},\emptyset ) \\ \hline
Y & (\{\psi \},\emptyset ) \\ \hline
\psi _{1}\wedge \cdots \wedge \psi _{n} & \left( \{\psi ,\psi _{1},\cdots
,\psi _{n}\},\{(\psi ,\psi _{1}),\cdots ,(\psi ,\psi _{n})\}\right) \cup
\bigcup\limits_{i=1}^{n}SG^{1}(\psi _{i}) \\ \hline
\psi _{1}\vee \cdots \vee \psi _{n} & \left( \{\psi ,\psi _{1},\cdots ,\psi
_{n}\},\{(\psi ,\psi _{1}),\cdots ,(\psi ,\psi _{n})\}\right) \cup
\bigcup\limits_{i=1}^{n}SG^{1}(\psi _{i}), \\ \hline
\end{array}%
\end{equation*}%
where we assume that in $\psi _{1}\wedge \cdots \wedge \psi _{n}$ none of
the $\psi _{i}$ is a conjunction, and in $\psi _{1}\vee \cdots \vee \psi
_{n} $ none of the $\psi _{i}$ is a disjunction.
Note that both $SG^{0}(\varphi ,\psi)$ and $SG^{1}(\varphi ,\psi )$ are defined as a pair ($V$, $E$) of
nodes and edges.
\newpage
