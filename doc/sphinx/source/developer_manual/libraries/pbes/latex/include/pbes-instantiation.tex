%TCIDATA{Version=5.50.0.2890}
%TCIDATA{LaTeXparent=1,1,pbes-implementation-notes.tex}
                      

\section{PBES instantiation}

In this section we describe two instantiation algorithms for PBESs.

\subsection{Lazy algorithm}

In this section we describe an implementation of the lazy instantiation
algorithm \textsc{Pbes2besLazy} that uses instantiation to compute a BES. It
takes two extra parameters, an injective function $\rho $ that renames
proposition variables to predicate variables, and a rewriter $R$ that
eliminates quantifiers from predicate formulae. Let $\mathcal{E=(\sigma }%
_{1}X_{1}(d_{1}:D_{1})=\varphi _{1})\ldots \mathcal{(\sigma }%
_{n}X_{n}(d_{n}:D_{n})=\varphi _{n})$ be a PBES, and $X_{init}(e_{init})$ an
initial state.%
\begin{equation*}
\begin{array}{l}
\text{\textsc{Pbes2besLazy(}}\mathcal{E}\text{, }X_{init}(e_{init})\text{, }R%
\text{, }\rho \text{\textsc{)}} \\ 
\text{\textbf{for }}i:=1\cdots n\text{ \textbf{do }}\mathcal{E}_{i}:=\epsilon
\\ 
todo:=\{R(X_{init}(e_{init}))\} \\ 
done:=\emptyset \\ 
\text{\textbf{while }}todo\neq \emptyset \text{ \textbf{do}} \\ 
\qquad \text{\textbf{choose }}X_{k}(e)\in todo \\ 
\qquad todo:=todo\ \backslash \ \{X_{k}(e)\} \\ 
\qquad done:=done\cup \{X_{k}(e)\} \\ 
\qquad X^{e}:=\rho (X_{k}(e)) \\ 
\qquad \psi ^{e}:=R(\varphi _{k}[d_{k}:=e]) \\ 
\qquad \mathcal{E}_{k}:=\mathcal{E}_{k}(\mathcal{\sigma }_{k}X^{e}=\rho
(\psi ^{e})) \\ 
\qquad todo:=todo\cup \{Y(f)\in \mathsf{occ}(\psi ^{e})\ |\ Y(f)\notin done\}
\\ 
\text{\textbf{return }}\mathcal{E}_{1}\cdots \mathcal{E}_{n},%
\end{array}%
\end{equation*}%
where $\rho $ is extended from predicate variables to quantifier free
predicate formulae using%
\begin{equation*}
\begin{array}{cc}
\rho (b)=_{def}b & \quad \rho (\varphi \oplus \psi )=_{def}\rho (\varphi
)\oplus \rho (\psi )%
\end{array}%
\end{equation*}%
\pagebreak

\subsection{Lazy algorithm (structure graph)}

A structure graph is a graph $(V,E)$ with $V$ a set of BES variables. On
this graph a partial function $r:V\rightarrow \mathbb{N}$ is defined that
assigns a rank to each node, and a partial function $d:V\rightarrow
\{\blacktriangle ,\blacktriangledown ,\top ,\bot \}$ is defined that assigns
a decoration to each node. A structure graph is formally defined using the
following SOS rules:%
\begin{equation*}
\frac{X\in bnd(\mathcal{E})}{\left\langle X,\mathcal{E}\right\rangle
\pitchfork rank_{\mathcal{E}}(X)}
\end{equation*}

\begin{equation*}
\frac{{}}{\left\langle true,\mathcal{E}\right\rangle \top }\qquad \frac{{}}{%
\left\langle false,\mathcal{E}\right\rangle \bot }
\end{equation*}%
\begin{equation*}
\frac{{}}{\left\langle f\wedge f^{\prime },\mathcal{E}\right\rangle
\blacktriangle }\qquad \frac{{}}{\left\langle f\vee f^{\prime },\mathcal{E}%
\right\rangle \blacktriangledown }
\end{equation*}%
\begin{equation*}
\frac{\left\langle f,\mathcal{E}\right\rangle \blacktriangle \quad \lnot
\left\langle f,\mathcal{E}\right\rangle \pitchfork \quad \left\langle f,%
\mathcal{E}\right\rangle \rightarrow \left\langle g,\mathcal{E}\right\rangle 
}{\left\langle f\wedge f^{\prime },\mathcal{E}\right\rangle \rightarrow
\left\langle g,\mathcal{E}\right\rangle }\qquad \frac{\left\langle f^{\prime
},\mathcal{E}\right\rangle \blacktriangle \quad \lnot \left\langle f^{\prime
},\mathcal{E}\right\rangle \pitchfork \quad \left\langle f^{\prime },%
\mathcal{E}\right\rangle \rightarrow \left\langle g^{\prime },\mathcal{E}%
\right\rangle }{\left\langle f\wedge f^{\prime },\mathcal{E}\right\rangle
\rightarrow \left\langle g^{\prime },\mathcal{E}\right\rangle }
\end{equation*}%
\begin{equation*}
\frac{\left\langle f,\mathcal{E}\right\rangle \blacktriangledown \quad \lnot
\left\langle f,\mathcal{E}\right\rangle \pitchfork \quad \left\langle f,%
\mathcal{E}\right\rangle \rightarrow \left\langle g,\mathcal{E}\right\rangle 
}{\left\langle f\vee f^{\prime },\mathcal{E}\right\rangle \rightarrow
\left\langle g,\mathcal{E}\right\rangle }\qquad \frac{\left\langle f^{\prime
},\mathcal{E}\right\rangle \blacktriangledown \quad \lnot \left\langle
f^{\prime },\mathcal{E}\right\rangle \pitchfork \quad \left\langle f^{\prime
},\mathcal{E}\right\rangle \rightarrow \left\langle g^{\prime },\mathcal{E}%
\right\rangle }{\left\langle f\vee f^{\prime },\mathcal{E}\right\rangle
\rightarrow \left\langle g^{\prime },\mathcal{E}\right\rangle }
\end{equation*}%
\begin{equation*}
\frac{\lnot \left\langle f,\mathcal{E}\right\rangle \blacktriangle }{%
\left\langle f\wedge f^{\prime },\mathcal{E}\right\rangle \rightarrow
\left\langle f,\mathcal{E}\right\rangle }\qquad \frac{\lnot \left\langle
f^{\prime },\mathcal{E}\right\rangle \blacktriangle }{\left\langle f\wedge
f^{\prime },\mathcal{E}\right\rangle \rightarrow \left\langle f^{\prime },%
\mathcal{E}\right\rangle }
\end{equation*}%
\begin{equation*}
\frac{\lnot \left\langle f,\mathcal{E}\right\rangle \blacktriangledown }{%
\left\langle f\vee f^{\prime },\mathcal{E}\right\rangle \rightarrow
\left\langle f,\mathcal{E}\right\rangle }\qquad \frac{\lnot \left\langle
f^{\prime },\mathcal{E}\right\rangle \blacktriangledown }{\left\langle f\vee
f^{\prime },\mathcal{E}\right\rangle \rightarrow \left\langle f^{\prime },%
\mathcal{E}\right\rangle }
\end{equation*}%
\begin{equation*}
\frac{\left\langle f,\mathcal{E}\right\rangle \pitchfork n}{\left\langle
f\wedge f^{\prime },\mathcal{E}\right\rangle \rightarrow \left\langle f,%
\mathcal{E}\right\rangle }\qquad \frac{\left\langle f^{\prime },\mathcal{E}%
\right\rangle \pitchfork n}{\left\langle f\wedge f^{\prime },\mathcal{E}%
\right\rangle \rightarrow \left\langle f,\mathcal{E}\right\rangle }
\end{equation*}%
\begin{equation*}
\frac{\left\langle f,\mathcal{E}\right\rangle \pitchfork n}{\left\langle
f\vee f^{\prime },\mathcal{E}\right\rangle \rightarrow \left\langle f,%
\mathcal{E}\right\rangle }\qquad \frac{\left\langle f^{\prime },\mathcal{E}%
\right\rangle \pitchfork n}{\left\langle f\vee f^{\prime },\mathcal{E}%
\right\rangle \rightarrow \left\langle f,\mathcal{E}\right\rangle }
\end{equation*}%
\begin{equation*}
\frac{\sigma X=f\in \mathcal{E}\quad \left\langle f,\mathcal{E}\right\rangle
\blacktriangle \quad \lnot \left\langle f,\mathcal{E}\right\rangle
\pitchfork }{\left\langle X,\mathcal{E}\right\rangle \blacktriangle }\qquad 
\frac{\sigma X=f\in \mathcal{E}\quad \left\langle f,\mathcal{E}\right\rangle
\blacktriangledown \quad \lnot \left\langle f,\mathcal{E}\right\rangle
\pitchfork }{\left\langle X,\mathcal{E}\right\rangle \blacktriangledown }
\end{equation*}%
\begin{equation*}
\frac{\sigma X=f\in \mathcal{E}\quad \lnot \left\langle f,\mathcal{E}%
\right\rangle \blacktriangledown \quad \lnot \left\langle f,\mathcal{E}%
\right\rangle \blacktriangle }{\left\langle X,\mathcal{E}\right\rangle
\rightarrow \left\langle f,\mathcal{E}\right\rangle }\qquad \frac{\sigma
X=f\in \mathcal{E}\quad \left\langle f,\mathcal{E}\right\rangle \pitchfork }{%
\left\langle X,\mathcal{E}\right\rangle \rightarrow \left\langle f,\mathcal{E%
}\right\rangle }
\end{equation*}%
\begin{equation*}
\frac{\sigma X=f\in \mathcal{E}\quad \left\langle f,\mathcal{E}\right\rangle
\rightarrow \left\langle g,\mathcal{E}\right\rangle \quad \left\langle f,%
\mathcal{E}\right\rangle \blacktriangle \quad \lnot \left\langle f,\mathcal{E%
}\right\rangle \pitchfork }{\left\langle X,\mathcal{E}\right\rangle
\rightarrow \left\langle g,\mathcal{E}\right\rangle }
\end{equation*}%
\begin{equation*}
\frac{\sigma X=f\in \mathcal{E}\quad \left\langle f,\mathcal{E}\right\rangle
\rightarrow \left\langle g,\mathcal{E}\right\rangle \quad \left\langle f,%
\mathcal{E}\right\rangle \blacktriangledown \quad \lnot \left\langle f,%
\mathcal{E}\right\rangle \pitchfork }{\left\langle X,\mathcal{E}%
\right\rangle \rightarrow \left\langle g,\mathcal{E}\right\rangle }
\end{equation*}%
First we rewrite that into a more explicit form, and remove the $%
\left\langle ,\mathcal{E}\right\rangle $ annotation:%
\begin{equation*}
\frac{X\in bnd(\mathcal{E})}{X\pitchfork rank_{\mathcal{E}}(X)}
\end{equation*}

\begin{equation*}
\frac{{}}{true\top }\qquad \frac{{}}{false\bot }
\end{equation*}%
\begin{equation*}
\frac{{}}{\left( f\wedge f^{\prime }\right) \blacktriangle }\qquad \frac{{}}{%
\left( f\vee f^{\prime }\right) \blacktriangledown }
\end{equation*}%
\begin{equation*}
\frac{f\rightarrow g}{(f\wedge f^{\prime })\rightarrow g}\qquad \frac{%
f\rightarrow g}{(f^{\prime }\wedge f)\rightarrow g}
\end{equation*}%
\begin{equation*}
\frac{f\rightarrow g}{(f\vee f^{\prime })\rightarrow g}\qquad \frac{%
f\rightarrow g}{(f^{\prime }\vee f)\rightarrow g}
\end{equation*}%
\begin{equation*}
\frac{\lnot f\blacktriangle }{f\wedge f^{\prime }\rightarrow f}\qquad \frac{%
\lnot f^{\prime }\blacktriangle }{f\wedge f^{\prime }\rightarrow f^{\prime }}
\end{equation*}%
\begin{equation*}
\frac{\lnot f\blacktriangledown }{f\vee f^{\prime }\rightarrow f}\qquad 
\frac{\lnot f^{\prime }\blacktriangledown }{f\vee f^{\prime }\rightarrow
f^{\prime }}
\end{equation*}%
\begin{equation*}
\frac{{}}{X\wedge f\rightarrow X}\qquad \frac{{}}{f\wedge X\rightarrow X}
\end{equation*}%
\begin{equation*}
\frac{{}}{X\vee f\rightarrow X}\qquad \frac{{}}{f\vee X\rightarrow X}
\end{equation*}%
\begin{equation*}
\frac{\sigma X=f\wedge f^{\prime }\in \mathcal{E}}{X\blacktriangle }\qquad 
\frac{\sigma X=f\vee f^{\prime }\in \mathcal{E}}{X\blacktriangledown }
\end{equation*}%
\begin{equation*}
\frac{\sigma X=Y\in \mathcal{E}}{X\rightarrow Y}\qquad \frac{\sigma X=\top
\in \mathcal{E}}{X\rightarrow \top }\qquad \frac{\sigma X=\bot \in \mathcal{E%
}}{X\rightarrow \bot }
\end{equation*}%
\begin{equation*}
\frac{\sigma X=f\wedge f^{\prime }\in \mathcal{E}\quad f\wedge f^{\prime
}\rightarrow g}{X\rightarrow g}
\end{equation*}%
\begin{equation*}
\frac{\sigma X=f\vee f^{\prime }\in \mathcal{E}\quad f\vee f^{\prime
}\rightarrow g}{X\rightarrow g}
\end{equation*}%
Note that in this definition separate nodes are created for the left hand
side $X$ and the right hand side $f$ of each equation $\sigma X=f$ . This is
undesirable, hence in the implementation below only nodes for each right
hand side $f$ are created.%
\begin{equation*}
\begin{array}{l}
\text{\textsc{Pbes2besStructureGraph(}}\mathcal{E}\text{, }X_{init}(e_{init})%
\text{, }R\text{, }\rho \text{\textsc{)}} \\ 
\text{\textbf{for }}i:=1\cdots n\text{ \textbf{do }}\mathcal{E}%
_{i}:=\epsilon  \\ 
todo:=\{R(X_{init}(e_{init}))\} \\ 
done:=\emptyset  \\ 
\text{\textbf{while }}todo\neq \emptyset \text{ \textbf{do}} \\ 
\qquad G:=(\emptyset ,\emptyset ) \\ 
\qquad \text{\textbf{choose }}X_{k}(e)\in todo \\ 
\qquad todo:=todo\ \backslash \ \{X_{k}(e)\} \\ 
\qquad done:=done\cup \{X_{k}(e)\} \\ 
\qquad X^{e}:=\rho (X_{k}(e)) \\ 
\qquad \psi ^{e}:=R(\varphi _{k}[d_{k}:=e]) \\ 
\qquad G:=G\cup SG(\psi ^{e}) \\ 
\qquad todo:=todo\cup \{Y(f)\in \mathsf{occ}(\psi ^{e})\ |\ Y(f)\notin done\}
\\ 
\text{\textbf{return }}G,%
\end{array}%
\end{equation*}%
\begin{equation*}
\begin{array}{|c|c|}
\hline
\psi  & SG(\psi ) \\ \hline\hline
true & (\{\psi \},\emptyset ) \\ \hline
false & (\{\psi \},\emptyset ) \\ \hline
\psi _{1}\wedge \cdots \wedge \psi _{n} & \left( \{\psi \}\cup \left(
\bigcup\limits_{i=1}^{n}V_{i}\right) ,\left(
\bigcup\limits_{i=1}^{n}E_{i}\right) \cup \{(\psi ,v(\psi _{i}))\mid \lnot
conj(\psi _{i})\}\right)  \\ \hline
\psi _{1}\vee \cdots \vee \psi _{n} & \left( \{\psi \}\cup \left(
\bigcup\limits_{i=1}^{n}V_{i}\right) ,\left(
\bigcup\limits_{i=1}^{n}E_{i}\right) \cup \{(\psi ,v(\psi _{i}))\mid \lnot
disj(\psi _{i})\}\right)  \\ \hline
Y & (\{Y,v(Y)\},\{(Y,v(Y))\}) \\ \hline
\end{array}%
\end{equation*}%
where $(V_{i},E_{i})=SG(\psi _{i})$, where $disj$ is defined as%
\begin{eqnarray*}
disj(b) &=&false \\
disj(\varphi \wedge \psi ) &=&true \\
disj(\varphi \vee \psi ) &=&false \\
disj(X) &=&false,
\end{eqnarray*}%
where $conj$ is defined as%
\begin{eqnarray*}
conj(b) &=&false \\
conj(\varphi \wedge \psi ) &=&false \\
conj(\varphi \vee \psi ) &=&true \\
conj(X) &=&false,
\end{eqnarray*}%
and where%
\begin{equation*}
=\left\{ 
\begin{array}{ccc}
v(X)=\psi  &  & \text{if }\sigma X=\psi \text{ is an equation of the BES} \\ 
v(\psi )=\psi  &  & \text{otherwise}%
\end{array}%
\right. 
\end{equation*}

\newpage

\subsection{Finite algorithm}

Let $\mathcal{E=(\sigma }_{1}X_{1}(d_{1}:D_{1},e_{1}:E_{1})=\varphi
_{1})\cdots \mathcal{(\sigma }_{n}X_{n}(d_{n}:D_{n},e_{n}:E_{n})=\varphi
_{n})$ be a PBES. We assume that all data sorts $D_{i}$ are finite and all
data sorts $E_{i}$ are infinite. Let $r$ be a data rewriter, and let $\rho $
be an injective function that creates a unique predicate variable from a
predicate variable name and a data value according to $\rho
(X(d:D,e:E),d_{0})\rightarrow Y(e:E)$, where $D$ is finite and $E$ is
infinite and $d_{0}\in D$. Note that $D$ and $D_{i}$ may be
multi-dimensional sorts.%
\begin{equation*}
\begin{array}{l}
\text{\textsc{Pbes2besFinite(}}\mathcal{E}\text{, }r\text{, }\rho \text{%
\textsc{)}} \\ 
\text{\textbf{for }}i:=1\cdots n\text{ \textbf{do}} \\ 
\qquad \mathcal{E}_{i}:=\{\mathcal{\sigma }_{i}\rho (X_{i},d)=R(\varphi
_{k}[d_{k}:=d])\ |\ d\in D_{i}\} \\ 
\text{\textbf{return }}\mathcal{E}_{1}\cdots \mathcal{E}_{n},%
\end{array}%
\end{equation*}%
with $R$ a rewriter on pbes expressions that is defined as follows:%
\begin{eqnarray*}
R(b) &=&b \\
R(\lnot \varphi ) &=&\lnot R(\varphi ) \\
R(\varphi \oplus \psi ) &=&R(\varphi )\oplus (\psi ) \\
R(X_{i}(d,e)) &=&\left\{ 
\begin{array}{cc}
\rho (X_{i},r(d))(r(e)) & \text{if }FV(d)=\emptyset \\ 
\dbigvee\limits_{d_{i}\in D_{i}}r(d=d_{i})\wedge \rho (X_{i},d_{i})(r(e)) & 
\text{if }FV(d)\neq \emptyset%
\end{array}%
\right. \\
R(\forall _{d:D}.\varphi ) &=&\forall _{d:D}.R(\varphi ) \\
R(\exists _{d:D}.\varphi ) &=&\exists _{d:D}.R(\varphi )
\end{eqnarray*}%
where $\oplus \in \{\vee ,\wedge ,\Rightarrow \}$, $b$ a data expression and 
$\varphi $ and $\psi $ pbes expressions and $FV(d)$ is the set of free
variables appearing in $d$.\newpage
