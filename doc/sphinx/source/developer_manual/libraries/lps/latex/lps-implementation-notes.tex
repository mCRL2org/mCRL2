
\documentclass{article}
%%%%%%%%%%%%%%%%%%%%%%%%%%%%%%%%%%%%%%%%%%%%%%%%%%%%%%%%%%%%%%%%%%%%%%%%%%%%%%%%%%%%%%%%%%%%%%%%%%%%%%%%%%%%%%%%%%%%%%%%%%%%%%%%%%%%%%%%%%%%%%%%%%%%%%%%%%%%%%%%%%%%%%%%%%%%%%%%%%%%%%%%%%%%%%%%%%%%%%%%%%%%%%%%%%%%%%%%%%%%%%%%%%%%%%%%%%%%%%%%%%%%%%%%%%%%
\usepackage{makeidx}
\usepackage{amssymb}
\usepackage{amsmath}
\usepackage{eurosym}
\usepackage{fullpage}
\usepackage{amsfonts}

\setcounter{MaxMatrixCols}{10}
%TCIDATA{OutputFilter=Latex.dll}
%TCIDATA{Version=5.50.0.2890}
%TCIDATA{<META NAME="SaveForMode" CONTENT="1">}
%TCIDATA{BibliographyScheme=Manual}
%TCIDATA{LastRevised=Thursday, October 30, 2014 10:22:32}
%TCIDATA{<META NAME="GraphicsSave" CONTENT="32">}
%TCIDATA{<META NAME="DocumentShell" CONTENT="Standard LaTeX\Blank - Standard LaTeX Article">}
%TCIDATA{CSTFile=40 LaTeX article.cst}

\font \aap cmmi10
\providecommand{\at}[1]{\mbox{\aap ,} #1}
\input{include/tcilatex}

\begin{document}


\section{ The LPS-library}

This document describes data types and algorithms of the LPS-library.

\subsection{Terms}

For an arbitrary term $t$ we define $\mathcal{V}ar(t)$ as the set of data
variables that occur in $t$. The result of substituting $d^{\prime }$ for $d$
in a term $t$ is denoted as $t[d:=d^{\prime }]$. With $d\in Sub(t)$ we
denote that data variable $d$ is a subterm of data expression $t$.

\subsection{Timed Linear processes}

Let $\mathcal{D}$ be the set of all data expressions, with $\simeq $ an
equivalence relation on $\mathcal{D}$. All data expressions have implicitly
defined an associated sort. Let $\mathcal{V\subset D}$ be the set of all
data variables. Furthermore, let $\mathcal{A}$ be the set of action labels.
A timed linear process $X$ is a process equation of the following form:%
\begin{equation}
\begin{array}{l}
X(d:D)=\sum\limits_{i\in I}\sum\limits_{e:E_{i}}c_{i}(d,e)\rightarrow
a_{i}(f_{i}(d,e)) \mbox{\aap ,} t_{i}(d,e)X(g_{i}(d,e))+\sum\limits_{i\in
J}\sum\limits_{e:E_{i}}c_{i}(d,e)\rightarrow \delta \mbox{\aap ,} t_{i}(d,e),%
\end{array}
\label{eq:lps_timed}
\end{equation}%
and $I$ and $J$ are disjoint and finite index sets, and for $i\in I\cup J$:

\begin{itemize}
\item $c_{i}:D\times E_{i}\rightarrow \mathbb{B}$ is the condition function,

\item $a_{i}(d,e)$ is a multi-action $a_{i}^{1}(f_{i}^{1}(d,e))|\cdots
|a_{i}^{n_{i}}(f_{i}^{n_{i}}(d,e))$, where $f_{i}:D\times E_{i}\rightarrow 
\mathcal{D}$ is the action parameter function,

\item $t_{i}:D\times E_{i}\rightarrow \mathbb{R}$ is the time stamp function,

\item $g_{i}:D\times E_{i}\rightarrow D$ is the next state function.
\end{itemize}

The components $d_{i}$ $\in \mathcal{V}$ of vector $d=[d_{1},\cdots ,d_{m}]$
are called the \emph{process parameters} of $X$.

\subsection{Untimed Linear processes}

Similarly, an untimed linear process $X$ is a process equation of the
following form:%
\begin{equation}
\begin{array}{l}
X(d:D)=\sum\limits_{i\in I}\sum\limits_{e:E_{i}}c_{i}(d,e)\rightarrow
a_{i}(f_{i}(d,e))X(g_{i}(d,e))+\sum\limits_{i\in
J}\sum\limits_{e:E_{i}}c_{i}(d,e)\rightarrow \delta%
\end{array}%
,  \label{eq:lps_untimed}
\end{equation}%
with $c_{i}$, $a_{i}$ and $g_{i}$ defined as above.

\subsubsection{State Space}

An untimed linear process \ref{eq:lps_untimed} with initial value $\overline{%
d}\in D$ defines a labeled transition system $M=(S,\Sigma ,\rightarrow
,s_{0})$, where

\begin{itemize}
\item $S=D$ is the (possibly infinite) set of states

\item $\Sigma =\{a_{i}(f_{i}(d,e))|i\in I\wedge e:E_{i}\}$ is the (possibly
infinite) set of labels

\item $\rightarrow $ $=\{(d,a_{i}(f_{i}(d,e)),g_{i}(d,e))|i\in I\wedge
e:E_{i}\wedge c_{i}(d,e)\}$

\item $s${}$_{0}=\overline{d}$ is the initial state
\end{itemize}

For $R\subset S$ we define $next\_states(R)$ as $\{s\in S|\exists r\in
R:r\rightarrow s\}$.

\subsection{Global variables}

In the mCRL2 tool set a linear process is parameterized with a finite set $%
DC $ of so called \emph{global} variables. This means that the expressions $%
c_{i}(d,e)$, $f_{i}(d,e)$, $g_{i}(d,e)$, and $t_{i}(d,e)$ may contain
unbound variables from the set $DC$. Global variables have the implicit
property that any two processes obtained by assigning values to them are
strongly bisimulation equivalent. In many algorithms we will simply ignore
global variables. In some cases, algorithms may assign values to some of the
global variables.

\subsection{Notations}

Let $p$ be an untimed linear process as defined in \ref{eq:lps_untimed}.

\begin{itemize}
\item A \emph{constant parameter} of $p$ is a parameter $d_{i}$ that has a
constant value for all reachable states of the corresponding state space,
given an initial state $\overline{d}:D$.

\item An \emph{insignificant parameter} of $p$ is a process parameter $d_{i}$
such that for any two initial states that differ only at the value of $d_{i}$%
, the corresponding state spaces are isomorphic.

\item Let $p$ be a linear process, $x$ a data variable and $y$ a data
expression. Then $p[x:=y]$ is the linear process obtained from $p$ by
applying the substitution $x:=y$ to all terms $c_{i}(d,e)$, $f_{i}(d,e)$, $%
g_{i}(d,e)$, and $t_{i}(d,e)$ that appear in the definition of $p$.
\end{itemize}

\subsection{Linear processes}

Linear process expressions in mCRL2 are expressions built according to the
following syntax:

\begin{equation*}
\begin{array}{ccc}
\text{expression} & \text{C++ equivalent} & \text{ATerm grammar} \\ 
b(e) & \text{action(}b\text{,}e\text{)} & \text{Action} \\ 
\sum\limits_{v}c\rightarrow a\mbox{\aap ,}t\ P(g) & \text{summand(}v,c,a,g%
\text{)} & \text{LinearProcessSummand} \\ 
\sum\limits_{v}c\rightarrow \delta \mbox{\aap ,}t & \text{summand(}%
v,c,\delta \text{)} & \text{LinearProcessSummand} \\ 
P(d:=e) & \text{process\_initializer(}f,d:=e\text{)\QQfnmark{%
Here $f$ is a superset of the free variables appearing in $e$.}} & \text{%
LinearProcessInit} \\ 
\sum\limits_{i\in I}s_{i} & \text{linear\_process(}f,v,s\text{)\QQfnmark{%
Here $s=[s_{1},\cdots ,s_{n}]$ is a sequence of summands, $v$ is the
sequence of process parameters of the corresponding process, and $f$ is a
sequence of free variables appearing in $s$.}} & \text{LinearProcess}%
\end{array}%
\QQfntext{-1}{
Here $f$ is a superset of the free variables appearing in $e$.}
\QQfntext{1}{
Here $s=[s_{1},\cdots ,s_{n}]$ is a sequence of summands, $v$ is the
sequence of process parameters of the corresponding process, and $f$ is a
sequence of free variables appearing in $s$.}
\end{equation*}%
where the types of the symbols are as follows:%
\begin{equation*}
\begin{array}{cl}
a & \text{a (timed) multi-action} \\ 
b & \text{a string (action name)} \\ 
\delta & \text{a (timed) deadlock} \\ 
P & \text{a process identifier} \\ 
e,f,g & \text{a sequence of data expressions} \\ 
d,v & \text{a sequence of data variables} \\ 
t & \text{a data expression of type real} \\ 
s & \text{a sequence of summands} \\ 
c & \text{ a data expression of type bool}%
\end{array}%
\end{equation*}%
A grammar for linear processes can be found in the Process implementation
notes document.

\subsection{Well typedness constraints}

Not all linear processes that adhere to the grammar for linear processes are
considered valid. A number of restrictions apply to make them valid input
for the mCRL2 toolset. These restrictions are called \emph{well typedness
constraints}.

\subsubsection{well typedness constraints for data specifications}

\begin{itemize}
\item the domain and range sorts of constructors are declared in the data
specification

\item the domain and range sorts of mappings are declared in the data
specification
\end{itemize}

\subsubsection{well typedness constraints for a linear process}

\begin{itemize}
\item the process parameters have unique names

\item process parameters and summation variables have different names

\item the left hand sides of the assignments of summands are contained in
the process parameters

\item the summands are well typed
\end{itemize}

\subsubsection{well typedness constraints for linear process specifications}

\begin{itemize}
\item the sorts occurring in the summation variables are declared in the
data specification

\item the sorts occurring in the process parameters are declared in the data
specification

\item the sorts occurring in the global variables are declared in the data
specification

\item the sorts occurring in the action labels are declared in the data
specification

\item the action labels occurring in the linear process are declared in the
action specification

\item the data specification is well typed

\item the linear process is well typed

\item the process initializer is well typed

\item the global variables occurring in the linear process are declared in
the global variable specification

\item the global variables occurring in the initial process are declared in
the global variable specification

\item the global variables have unique names
\end{itemize}

\subsubsection{well typedness constraints for other types}

\begin{itemize}
\item the sorts of the left and right hand side of an assignment are the same

\item the time of a summand has type Real

\item the condition of a summand has type Bool

\item the set of left hand sides of assignments in an action summand or
process initializer does not contain duplicates
\end{itemize}

\newpage

\section{Algorithms}

We now define two operations on linear processes: removing (insignificant)
parameters and removing constant parameters. Let $X(d:D)$ be an linear
process as defined in \ref{eq:lps_untimed}, or \ref{eq:lps_timed} and let $%
\{d_{j_{1}},\cdots ,d_{j_{m}}\}$ be a set of insignificant parameters of $X$%
. Then we define \textsc{RemoveParameters(}$p,\{d_{j_{1}},\cdots
,d_{j_{m}}\} $\textsc{)} as a linear process obtained from $X(d:D)$ by
removing $\{d_{j_{1}},\cdots ,d_{j_{m}}\}$ from the process parameters of $X$%
, and by replacing each term $c_{i}(d,e)$, $f_{i}(d,e)$, $g_{i}(d,e)$, or $%
t_{i}(d,e)$ that appears in the definition of $X$, and that has one of the
variables $d_{j_{1}},\cdots ,d_{j_{m}}$ as a subterm by a term $%
c_{i}^{\prime }(d,e)$, $f_{i}^{\prime }(d,e)$, $g_{i}^{\prime }(d,e)$, or $%
t_{i}^{\prime }(d,e)$ that does not have one of the variables $%
d_{j_{1}},\cdots ,d_{j_{m}}$ as a subterm, and such that the remaining
process is strongly bisimulation equivalent to $X(d:D)$\footnote{%
A more formal definition of this is welcome.}.

Let $X(d:D)$ be an linear process as defined in \ref{eq:lps_untimed}, or \ref%
{eq:lps_timed} and let $\{d_{j_{1}},\cdots ,d_{j_{m}}\}$ be a set of
constant parameters of $X$, given the state $\overline{d}:D$. Then we define 
\textsc{RemoveConstantParameters}($p,\{d_{j_{1}},\cdots ,d_{j_{m}}\},%
\overline{d}$) as a linear process obtained from $X(d:D)$ by removing $%
\{d_{j_{1}},\cdots ,d_{j_{m}}\}$ from the process parameters of $X$, and by
replacing each term $c_{i}(d,e)$, $f_{i}(d,e)$, $g_{i}(d,e)$, or $t_{i}(d,e)$
that appears in the definition of $X$ by a term $c_{i}^{\prime }(d,e)$, $%
f_{i}^{\prime }(d,e)$, $g_{i}^{\prime }(d,e)$, or $t_{i}^{\prime }(d,e)$
that does not have one of the variables $d_{j_{1}},\cdots ,d_{j_{m}}$ as a
subterm, and such that the remaining process is strongly bisimulation
equivalent to $X(d:D)$\footnote{%
Or should this be restricted to the result of substituting all the constant
values, and possibly applying the rewriter to it?}.

\subsection{Parelm}

%TCIMACRO{\QSubDoc{Include parelm}{\documentclass{article}

\begin{document}

Let the linear process $p$ be defined as%
\[
X(d:D)=\sum_{i\in I}\sum_{e:E_{i}}c_{i}(d,e)\rightarrow a_{i}(d,e)\circ
t_{i}(d,e)X(g_{i}(d,e)),
\]%
with $d=[d_{1},\ldots ,d_{n}]$. For an arbitrary term $t$ we define $Var(t)$
as the set of data variables that occur as a subterm of $t$.

\bigskip

Let $G^{0}$ be defined as
\[
G^{0}=\{k\in \lbrack 1,\ldots ,n]\ |\ \exists _{i\in I}:d_{k}\in
Var(c_{i}(d,e)\rightarrow a_{i}(d,e)\circ t_{i}(d,e))\}.
\]%
We define for $m\geq 0$:%
\[
G^{m+1}=G^{m}\cup \{k\in \lbrack 1,\ldots ,n]\ |\ \exists _{l\in
G^{m}}\exists _{i\in I}:d_{k}\in Var(g_{i}(d,e)_{l}\}.
\]%
Let $M$ be the minimal value for which $G^{M}=G^{M+1}$. The parelm algorithm
eliminates all variables $d_{k}$ with $k\notin G^{M}$ from the process $p$.
The claim is that the resulting process is strongly bisimulation equivalent
to $p$.

\end{document}
}}%
%BeginExpansion
\documentclass{article}

\begin{document}

Let the linear process $p$ be defined as%
\[
X(d:D)=\sum_{i\in I}\sum_{e:E_{i}}c_{i}(d,e)\rightarrow a_{i}(d,e)\circ
t_{i}(d,e)X(g_{i}(d,e)),
\]%
with $d=[d_{1},\ldots ,d_{n}]$. For an arbitrary term $t$ we define $Var(t)$
as the set of data variables that occur as a subterm of $t$.

\bigskip

Let $G^{0}$ be defined as
\[
G^{0}=\{k\in \lbrack 1,\ldots ,n]\ |\ \exists _{i\in I}:d_{k}\in
Var(c_{i}(d,e)\rightarrow a_{i}(d,e)\circ t_{i}(d,e))\}.
\]%
We define for $m\geq 0$:%
\[
G^{m+1}=G^{m}\cup \{k\in \lbrack 1,\ldots ,n]\ |\ \exists _{l\in
G^{m}}\exists _{i\in I}:d_{k}\in Var(g_{i}(d,e)_{l}\}.
\]%
Let $M$ be the minimal value for which $G^{M}=G^{M+1}$. The parelm algorithm
eliminates all variables $d_{k}$ with $k\notin G^{M}$ from the process $p$.
The claim is that the resulting process is strongly bisimulation equivalent
to $p$.

\end{document}
%
%EndExpansion

\subsection{Constelm}

%TCIMACRO{\QSubDoc{Include constelm}{%TCIDATA{Version=5.50.0.2890}
%TCIDATA{LaTeXparent=1,1,pbes-implementation-notes.tex}
                      

Let $P$ be the stochastic linear process%
\[
\begin{array}{l}
P(d:D)=\sum\limits_{i\in I}\sum\limits_{e:E_{i}}c_{i}(d,e_{i})\rightarrow
a_{i}(f_{i}(d,e_{i}))\at t_{i}(d,e_{i}).\frac{p_{i}(d,e_{i},h_{i})}{%
h_{i}:F_{i}}P(g_{i}(d,e_{i},h_{i})),%
\end{array}%
\]%
with initial state%
\[
\begin{array}{l}
P_{init}=\frac{p(h)}{h:F}P(g(h)),%
\end{array}%
\]

with $d=[d_{1},\ldots ,d_{n}]$ the process parameters of $P$. The algorithm 
\textsc{Constelm} is used to find process parameters that have a constant
value. If needed the global variables of $P$ will be assigned constant
values. The result is a substitution $\sigma $ that assigns these constant
values to the corresponding process parameters and global variables. It is
not guaranteed that all constant parameters are detected.

\subsection{Constelm implementations}

Then we define the following implementation of \textsc{Constelm}:

\[
\begin{tabular}{|l|}
\hline
\textsc{Constelm(}$P,P_{init},V,R$\textsc{)} \\ 
$r:=R(g(h))$ \\ 
$G:=\{d_{1},\ldots ,d_{n}\}$ \\ 
$\sigma :=\epsilon $ \\ 
\textbf{for} $j:=1$ \textbf{to} $n$ \textbf{do} \\ 
$\qquad \sigma \lbrack d_{j}]:=r_{j}$ \\ 
$\qquad undo[j]:=\emptyset $ \\ 
$\text{\textbf{do}}$ \\ 
$\qquad \Delta G:=\emptyset $ \\ 
$\qquad $\textbf{for} $i\in I$ \textbf{do} \\ 
$\qquad \qquad $\textbf{if }$R(\sigma (c_{i}(d,e_{i})))\neq false$ \\ 
$\qquad \qquad \qquad $\textbf{for }$d_{j}\in G\setminus \Delta G$ \textbf{do%
} \\ 
$\qquad \qquad \qquad \qquad $\textbf{let }$z=R(g_{i}(d,e_{i},h_{i})_{j},%
\sigma )$ \\ 
$\qquad \qquad \qquad \qquad $\textbf{if }$z\neq \sigma (d_{j})$ \\ 
$\qquad \qquad \qquad \qquad \qquad $\textbf{if }$z\in V$ \\ 
$\qquad \qquad \qquad \qquad \qquad \qquad \sigma \lbrack z]:=r_{j}$ \\ 
$\qquad \qquad \qquad \qquad \qquad \qquad undo[j]:=undo[j]\cup \{z\}$ \\ 
$\qquad \qquad \qquad \qquad \qquad $\textbf{else} \\ 
$\qquad \qquad \qquad \qquad \qquad \qquad \Delta G:=\Delta G\cup \{d_{j}\}$
\\ 
$\qquad \qquad \qquad \qquad \qquad \qquad \sigma \lbrack d_{j}]:=d_{j}$ \\ 
$\qquad \qquad \qquad \qquad \qquad \qquad $\textbf{for }$w\in undo[j]$ 
\textbf{do} \\ 
$\qquad \qquad \qquad \qquad \qquad \qquad \qquad \sigma \lbrack w]:=w$ \\ 
$\qquad \qquad \qquad \qquad \qquad \qquad undo[j]:=\emptyset $ \\ 
$\qquad G:=G\setminus \Delta G$ \\ 
$\text{\textbf{while }}(\Delta G\neq \emptyset )$ \\ 
$\text{\textbf{return} }\sigma $ \\ \hline
\end{tabular}%
\]%
where $V$ is the set of global variables of the linear process $P$, where $R$
is a rewriter and $\epsilon $ is the empty substitution (i.e. the identity
function). The notation $\sigma \lbrack v]:=w$ means $\sigma =\sigma
^{\prime }$ with%
\[
\left\{ 
\begin{array}{l}
\sigma ^{\prime }(x)=\sigma ^{\prime }(x)\text{ if }x\neq v \\ 
\sigma ^{\prime }(x)=w\text{ if }x\neq v%
\end{array}%
\right. 
\]%
An alternative implementation may look like this:

\[
\begin{tabular}{|l|}
\hline
\textsc{Constelm2(}$p,\overline{d}$\textsc{)} \\ 
$R:=\overline{\{d\}}$ \\ 
$G:=\{1,\ldots ,n\}$ \\ 
$\text{\textbf{do}}$ \\ 
$\qquad \Delta R:=next\_states(p,R)\backslash R$ \\ 
$\qquad \text{\textbf{if }}\exists _{i\in I}.\exists _{j\in G}.\exists
_{r\in \Delta R}.r_{j}\neq \overline{d}_{j}$ \\ 
$\qquad \qquad G:=G\backslash \{j\}$ \\ 
$\qquad \text{\textbf{fi}}$ \\ 
$\qquad R:=R\cup \Delta R$ \\ 
$\text{\textbf{while }}(\Delta R\neq \emptyset )$ \\ 
$\text{\textbf{return} \textsc{RemoveConstantParameters(}}%
p,\{d_{j_{1}},\cdots ,d_{j_{m}}\},\overline{d}\text{\textsc{)}},$ \\ \hline
\end{tabular}%
\]%
where $\{j_{1},\cdots ,j_{m}\}=G$.\newpage
}}%
%BeginExpansion
%TCIDATA{Version=5.50.0.2890}
%TCIDATA{LaTeXparent=1,1,pbes-implementation-notes.tex}
                      

Let $P$ be the stochastic linear process%
\[
\begin{array}{l}
P(d:D)=\sum\limits_{i\in I}\sum\limits_{e:E_{i}}c_{i}(d,e_{i})\rightarrow
a_{i}(f_{i}(d,e_{i}))\at t_{i}(d,e_{i}).\frac{p_{i}(d,e_{i},h_{i})}{%
h_{i}:F_{i}}P(g_{i}(d,e_{i},h_{i})),%
\end{array}%
\]%
with initial state%
\[
\begin{array}{l}
P_{init}=\frac{p(h)}{h:F}P(g(h)),%
\end{array}%
\]

with $d=[d_{1},\ldots ,d_{n}]$ the process parameters of $P$. The algorithm 
\textsc{Constelm} is used to find process parameters that have a constant
value. If needed the global variables of $P$ will be assigned constant
values. The result is a substitution $\sigma $ that assigns these constant
values to the corresponding process parameters and global variables. It is
not guaranteed that all constant parameters are detected.

\subsection{Constelm implementations}

Then we define the following implementation of \textsc{Constelm}:

\[
\begin{tabular}{|l|}
\hline
\textsc{Constelm(}$P,P_{init},V,R$\textsc{)} \\ 
$r:=R(g(h))$ \\ 
$G:=\{d_{1},\ldots ,d_{n}\}$ \\ 
$\sigma :=\epsilon $ \\ 
\textbf{for} $j:=1$ \textbf{to} $n$ \textbf{do} \\ 
$\qquad \sigma \lbrack d_{j}]:=r_{j}$ \\ 
$\qquad undo[j]:=\emptyset $ \\ 
$\text{\textbf{do}}$ \\ 
$\qquad \Delta G:=\emptyset $ \\ 
$\qquad $\textbf{for} $i\in I$ \textbf{do} \\ 
$\qquad \qquad $\textbf{if }$R(\sigma (c_{i}(d,e_{i})))\neq false$ \\ 
$\qquad \qquad \qquad $\textbf{for }$d_{j}\in G\setminus \Delta G$ \textbf{do%
} \\ 
$\qquad \qquad \qquad \qquad $\textbf{let }$z=R(g_{i}(d,e_{i},h_{i})_{j},%
\sigma )$ \\ 
$\qquad \qquad \qquad \qquad $\textbf{if }$z\neq \sigma (d_{j})$ \\ 
$\qquad \qquad \qquad \qquad \qquad $\textbf{if }$z\in V$ \\ 
$\qquad \qquad \qquad \qquad \qquad \qquad \sigma \lbrack z]:=r_{j}$ \\ 
$\qquad \qquad \qquad \qquad \qquad \qquad undo[j]:=undo[j]\cup \{z\}$ \\ 
$\qquad \qquad \qquad \qquad \qquad $\textbf{else} \\ 
$\qquad \qquad \qquad \qquad \qquad \qquad \Delta G:=\Delta G\cup \{d_{j}\}$
\\ 
$\qquad \qquad \qquad \qquad \qquad \qquad \sigma \lbrack d_{j}]:=d_{j}$ \\ 
$\qquad \qquad \qquad \qquad \qquad \qquad $\textbf{for }$w\in undo[j]$ 
\textbf{do} \\ 
$\qquad \qquad \qquad \qquad \qquad \qquad \qquad \sigma \lbrack w]:=w$ \\ 
$\qquad \qquad \qquad \qquad \qquad \qquad undo[j]:=\emptyset $ \\ 
$\qquad G:=G\setminus \Delta G$ \\ 
$\text{\textbf{while }}(\Delta G\neq \emptyset )$ \\ 
$\text{\textbf{return} }\sigma $ \\ \hline
\end{tabular}%
\]%
where $V$ is the set of global variables of the linear process $P$, where $R$
is a rewriter and $\epsilon $ is the empty substitution (i.e. the identity
function). The notation $\sigma \lbrack v]:=w$ means $\sigma =\sigma
^{\prime }$ with%
\[
\left\{ 
\begin{array}{l}
\sigma ^{\prime }(x)=\sigma ^{\prime }(x)\text{ if }x\neq v \\ 
\sigma ^{\prime }(x)=w\text{ if }x\neq v%
\end{array}%
\right. 
\]%
An alternative implementation may look like this:

\[
\begin{tabular}{|l|}
\hline
\textsc{Constelm2(}$p,\overline{d}$\textsc{)} \\ 
$R:=\overline{\{d\}}$ \\ 
$G:=\{1,\ldots ,n\}$ \\ 
$\text{\textbf{do}}$ \\ 
$\qquad \Delta R:=next\_states(p,R)\backslash R$ \\ 
$\qquad \text{\textbf{if }}\exists _{i\in I}.\exists _{j\in G}.\exists
_{r\in \Delta R}.r_{j}\neq \overline{d}_{j}$ \\ 
$\qquad \qquad G:=G\backslash \{j\}$ \\ 
$\qquad \text{\textbf{fi}}$ \\ 
$\qquad R:=R\cup \Delta R$ \\ 
$\text{\textbf{while }}(\Delta R\neq \emptyset )$ \\ 
$\text{\textbf{return} \textsc{RemoveConstantParameters(}}%
p,\{d_{j_{1}},\cdots ,d_{j_{m}}\},\overline{d}\text{\textsc{)}},$ \\ \hline
\end{tabular}%
\]%
where $\{j_{1},\cdots ,j_{m}\}=G$.\newpage
%
%EndExpansion

\subsection{Conversion to linear process}

The function $lin$ converts a process expression to linear process format,
if it is linear.

\begin{eqnarray*}
lin(\text{if\_then}(c,\delta )) &=&\text{summand}([],c,\delta ) \\
lin(\text{if\_then}(c,\text{seq}(x,P(g))) &=&\text{summand}([],c,lin(x),g) \\
\text{add\_summand\_variables}(\text{summand}(v,c,\overrightarrow{a},g),w)
&=&\text{summand}(v+w,c,\overrightarrow{a},g) \\
\text{add\_summand\_variables}(\text{summand}(v,c,\delta ),w) &=&\text{%
summand}(v+w,c,\delta ) \\
lin(\text{sum}(v,x) &=&\text{add\_summand\_variables}(lin(x),w) \\
lin(\text{seq}(x,P)) &=&lin(x) \\
\text{set\_multi\_action\_time}(\text{multi\_action}([a_{1},\cdots
,a_{n}]),t) &=&\text{multi\_action}([a_{1},\cdots ,a_{n}],t) \\
lin(\text{at\_time}(x,t)) &=&\text{set\_multi\_action\_time}(lin(x),t) \\
\text{multi\_action}([a_{1},\cdots ,a_{m}])+\text{multi\_action}%
([a_{m+1},\cdots ,a_{n}]) &=&\text{multi\_action}([a_{1},\cdots ,a_{n}]) \\
lin(\text{sync}(x,y)) &=&lin(x)+lin(y) \\
lin(a) &=&\text{multi\_action}([a]) \\
lin(\tau ) &=&\text{multi\_action}([]) \\
lin(\text{choice}(\text{choice}(x,y),z)) &=&lin(\text{choice}(x,y))+[lin(z)]
\\
lin(\text{choice}(x,\text{choice}(y,z))) &=&[lin(x)]+lin(\text{choice}(y,z))
\\
lin(\text{choice}(x,y)) &=&[lin(x),lin(y)]
\end{eqnarray*}

\end{document}
