% at
% left merge
% \RequirePackage{stmaryrd}


\documentclass{article}
%%%%%%%%%%%%%%%%%%%%%%%%%%%%%%%%%%%%%%%%%%%%%%%%%%%%%%%%%%%%%%%%%%%%%%%%%%%%%%%%%%%%%%%%%%%%%%%%%%%%%%%%%%%%%%%%%%%%%%%%%%%%%%%%%%%%%%%%%%%%%%%%%%%%%%%%%%%%%%%%%%%%%%%%%%%%%%%%%%%%%%%%%%%%%%%%%%%%%%%%%%%%%%%%%%%%%%%%%%%%%%%%%%%%%%%%%%%%%%%%%%%%%%%%%%%%
\usepackage{amssymb}
\usepackage{geometry}
\usepackage{stmaryrd}

%TCIDATA{OutputFilter=LATEX.DLL}
%TCIDATA{Version=5.50.0.2890}
%TCIDATA{<META NAME="SaveForMode" CONTENT="1">}
%TCIDATA{BibliographyScheme=Manual}
%TCIDATA{Created=Friday, June 15, 2012 17:58:29}
%TCIDATA{LastRevised=Monday, October 17, 2016 09:42:49}
%TCIDATA{<META NAME="GraphicsSave" CONTENT="32">}
%TCIDATA{<META NAME="DocumentShell" CONTENT="Standard LaTeX\Blank - Standard LaTeX Article">}
%TCIDATA{CSTFile=40 LaTeX article.cst}

\newtheorem{theorem}{Theorem}
\newtheorem{acknowledgement}[theorem]{Acknowledgement}
\newtheorem{algorithm}[theorem]{Algorithm}
\newtheorem{axiom}[theorem]{Axiom}
\newtheorem{case}[theorem]{Case}
\newtheorem{claim}[theorem]{Claim}
\newtheorem{conclusion}[theorem]{Conclusion}
\newtheorem{condition}[theorem]{Condition}
\newtheorem{conjecture}[theorem]{Conjecture}
\newtheorem{corollary}[theorem]{Corollary}
\newtheorem{criterion}[theorem]{Criterion}
\newtheorem{definition}[theorem]{Definition}
\newtheorem{example}[theorem]{Example}
\newtheorem{exercise}[theorem]{Exercise}
\newtheorem{lemma}[theorem]{Lemma}
\newtheorem{notation}[theorem]{Notation}
\newtheorem{problem}[theorem]{Problem}
\newtheorem{proposition}[theorem]{Proposition}
\newtheorem{remark}[theorem]{Remark}
\newtheorem{solution}[theorem]{Solution}
\newtheorem{summary}[theorem]{Summary}
\newenvironment{proof}[1][Proof]{\noindent\textbf{#1.} }{\ \rule{0.5em}{0.5em}}
\geometry{left=1in,right=1in,top=1in,bottom=1in}
\input{include/tcilatex}
\font \aap cmmi10
\providecommand{\at}{\mathbin{\mbox{\aap ,}}}
\providecommand{\leftmerge}{\mathbin{\mathrel{\llfloor}}}


\begin{document}

\title{Process Library Implementation Notes}
\author{Wieger Wesselink}
\maketitle

\section{Process Library Implementation Notes}

\subsection{Processes}

Process expressions in mCRL2 are expressions built according to the
following syntax:%
\[
\begin{array}{ccc}
\text{expression} & \text{C++ equivalent} & \text{ATerm grammar} \\ 
a(e) & \text{action(}a\text{,}e\text{)} & \text{Action} \\ 
P(e) & \text{process(}P\text{,}e\text{)} & \text{Process} \\ 
P(d:=e) & \text{process\_assignment(}P\text{,}d:=e\text{)} & \text{%
ProcessAssignment} \\ 
\delta & \text{delta()} & \text{Delta} \\ 
\tau & \text{tau()} & \text{Tau} \\ 
\dsum\limits_{d}x & \text{sum(}d\text{,}x\text{)} & \text{Sum} \\ 
\partial _{B}(x) & \text{block(}B\text{,}x\text{)} & \text{Block} \\ 
\tau _{B}(x) & \text{hide(}B\text{,}x\text{)} & \text{Hide} \\ 
\rho _{R}(x) & \text{rename(}R\text{,}x\text{)} & \text{Rename} \\ 
\Gamma _{C}(x) & \text{comm(}C\text{,}x\text{)} & \text{Comm} \\ 
\bigtriangledown _{V}(x) & \text{allow(}V\text{,}x\text{)} & \text{Allow} \\ 
x\mid y & \text{sync(}x\text{,}y\text{)} & \text{Sync} \\ 
x\at t & \text{at\_time(}x\text{,}t\text{)} & \text{AtTime} \\ 
x\cdot y & \text{seq(}x\text{,}y\text{)} & \text{Seq} \\ 
c\rightarrow x & \text{if\_then(}c\text{,}x\text{)} & \text{IfThen} \\ 
c\rightarrow x\diamond y & \text{if\_then\_else(}c\text{,}x\text{,}y\text{)}
& \text{IfThenElse} \\ 
x\ll y & \text{binit(}x\text{,}y\text{)} & \text{BInit} \\ 
x\ \parallel \ y & \text{merge(}x\text{,}y\text{)} & \text{Merge} \\ 
x\ \leftmerge \ y & \text{lmerge(}x\text{,}y\text{)} & \text{LMerge} \\ 
x+y & \text{choice(}x\text{,}y\text{)} & \text{Choice}%
\end{array}%
\]

where the types of the symbols are as follows:%
\[
\begin{array}{cl}
a,b & \text{strings (action names)} \\ 
P & \text{a process identifier} \\ 
e & \text{a sequence of data expressions} \\ 
d & \text{a sequence of data variables} \\ 
B & \text{a set of strings (action names) } \\ 
R & \text{a sequence of rename expressions} \\ 
C & \text{a sequence of communication expressions} \\ 
V & \text{a sequence of multi actions} \\ 
t & \text{a data expression of type real} \\ 
x,y & \text{process expressions} \\ 
c & \text{ a data expression of type bool}%
\end{array}%
\]%
A rename expression is of the form $a\rightarrow b$, with $a$ and $b$ action
names. A multi action is of the form $a_{1}\ |\ \cdots \ |\ a_{n}$, with $%
a_{i}$ actions. A communication expression is of the form $b_{1}\ |\ \cdots
\ |\ b_{n}\rightarrow b$, with $b$ and $b_{i}$ action names.

\subsubsection{Restrictions}

A multi action is a multi set of actions. The left hand sides of the
communication expressions in $C$ must be unique. Also the left hand sides of
the rename expressions in $R$ must be unique.

\newpage

\subsubsection{Linear process expressions}

Linear process expressions are a subset of process expresions satisfying the
following grammar:
\begin{verbatim}
<linear process expression> ::= choice(<linear process expression>, <linear process expression>)
                              | <summand>
 
<summand>                   ::= sum(<variables>, <alternative>)
                              | <conditional action prefix>
                              | <conditional deadlock>
 
<conditional action prefix> ::= if_then(<condition>, <action prefix>)
                              | <action prefix>
 
<action prefix>             ::= seq(<timed multiaction>, <process reference>)
                              | <timed multiaction>
 
<timed multiaction>         ::= at_time(<multiaction>, <time stamp>)
                              | <multiaction>
 
<multiaction>               ::= tau()
                              | <action>
                              | sync(<multiaction>, <multiaction>)
 
<conditional deadlock>      ::= if_then(<condition>, <timed deadlock>)
                              | <timed deadlock>
 
<timed deadlock>            ::= delta()
                              | at_time(delta(), <time stamp>)
 
<process reference>         ::= process(<process identifier>, <data expressions>)
                              | process_assignment(<process identifier>, <data assignments>)
\end{verbatim}

\newpage

\subsection{Guarded process expressions}

We define the predicate $is\_guarded$ for process expressions as follows: $%
is\_guarded(p)=is\_guarded(p,\emptyset )$%
\[
\begin{array}{lll}
is\_guarded(a(e),W) & = & true \\ 
is\_guarded(\delta ,W) & = & true \\ 
is\_guarded(\tau ,W) & = & true \\ 
is\_guarded(P(e),W) & = & 
\begin{array}{l}
\left\{ 
\begin{array}{ll}
false & \text{if }P\in W \\ 
is\_guarded(p,W\cup \{P\}) & \text{if }P\notin W%
\end{array}%
\right. \\ 
\text{where }P(d)=p\text{ is the equation corresponding to }P(e)%
\end{array}
\\ 
is\_guarded(p+q,W) & = & is\_guarded(p,W)\wedge is\_guarded(q,W) \\ 
is\_guarded(p\cdot q,W) & = & is\_guarded(p,W) \\ 
is\_guarded(c\rightarrow p,W) & = & is\_guarded(p,W) \\ 
is\_guarded(c\rightarrow p\diamond q,W) & = & is\_guarded(p,W)\wedge
is\_guarded(q,W) \\ 
is\_guarded(\Sigma _{d:D}\ p,W) & = & is\_guarded(p,W) \\ 
is\_guarded(p\at t,W) & = & is\_guarded(p,W) \\ 
is\_guarded(p\ll q,W) & = & is\_guarded(p,W) \\ 
is\_guarded(p\parallel q,W) & = & is\_guarded(p,W)\wedge is\_guarded(q,W) \\ 
is\_guarded(p\leftmerge q,W) & = & is\_guarded(p,W) \\ 
is\_guarded(p\mid q,W) & = & is\_guarded(p,W)\wedge is\_guarded(q,W) \\ 
is\_guarded(\rho _{R}(p),W) & = & is\_guarded(p,W) \\ 
is\_guarded(\partial _{B}(p),W) & = & is\_guarded(p,W) \\ 
is\_guarded(\tau _{I}(p),W) & = & is\_guarded(p,W) \\ 
is\_guarded(\Gamma _{C}(p),W) & = & is\_guarded(p,W) \\ 
is\_guarded(\nabla _{V}(p),W) & = & is\_guarded(p,W)%
\end{array}%
\]%
\newline
N.B. This specification assumes that process names are unique. In mCRL2
process names can be overloaded, therefore in the implemenation $W$ contains 
\emph{process identifiers} (i.e. both the process name and the sorts of the
arguments) instead of process names.

\newpage

\subsection{Alphabet reduction}

Alphabet reduction is a preprocessing step for linearization. It is a
transformation on process expressions that preserves branching bisimulation.

\subsubsection{Notations}

In this text action names are represented using $a,b,\ldots $ and multi
action names using $\alpha ,\beta ,\ldots $ So in general we have $\alpha
=a_{1}\mid \ldots \mid a_{n}$. In alphabet reduction data parameters play a
minor role, therefore we choose a notation in which data parameters are
omitted. We use the abbreviation $\overline{a}=a(e_{1},\ldots ,e_{n})$ to
denote an action, and $\overline{\alpha }=\overline{a_{1}}\mid \ldots \mid 
\overline{a_{n}}$ to denote a multi action, where $e_{1},\ldots ,e_{n}$ are
data expressions.Note that a multi action is a multiset (or bag) of actions
and a multi action name is a multiset of names. We write $\alpha \beta $ as
shorthand for $\alpha \cup \beta $ and $a\beta $ for $\{a\}\cup \beta $.
Sets of multi action names are represented using $A,A_{1},A_{2},\ldots $ A
communication $C$ maps multi action names to action names, and is denoted as 
$\{\alpha _{1}\rightarrow a_{1},\ldots ,\alpha _{n}\rightarrow a_{n}\}$. A
renaming $R$ is a substitution on action names, and is denoted as $%
R=\{a_{1}\rightarrow b_{1},\ldots ,a_{n}\rightarrow b_{n}\}$. A block set $B$
is a set of action names. A hide set $I$ is a set of action names.

\subsubsection{Definitions}

We define multi actions $\overline{\alpha }$ using the following grammar:%
\[
\overline{\alpha }:=\overline{a}\shortmid \overline{\alpha }\mid \overline{a}%
, 
\]%
where $\overline{a}$ is an action, and where $\shortmid $ is used to
distinguish alternatives.

We define pCRL terms $p$ using the following grammar:%
\[
p::=\overline{a}\shortmid P\shortmid \delta \shortmid \tau \shortmid
p+p\shortmid p\cdot p\shortmid c\rightarrow p\shortmid c\rightarrow
p\diamond p\shortmid \Sigma _{d:D}p\shortmid p\at t\shortmid p\ll p, 
\]%
and parallel mCRL terms $q$ using the following grammar:%
\[
q::=p\shortmid q\parallel q\shortmid q\leftmerge q\shortmid q\mid q\shortmid
\rho _{R}(q)\shortmid \partial _{B}(q)\shortmid \tau _{I}(q)\shortmid \Gamma
_{C}(q)\shortmid \nabla _{V}(q). 
\]

\paragraph{Remark 1}

Note that there is an unfortunate overload of the $\boldsymbol{\mid }$%
-operator in both multi actions and process expressions. This has
consequences for the implementation, since it there is no clean distinction
between parallel and non-parallel operators.

\paragraph{Remark 2}

The mCRL2 language also has a construct $P(d_{i_{1}}=e_{i_{1}},\ldots
,d_{i_{k}}=e_{i_{k}})$, but this is just a shorthand notation. Therefore we
will ignore it in this text.

\subsubsection{Alphabet operations}

Let $A,A_{1}$ and $A_{2}$ be sets of multi action names. Then we define%
\[
\begin{array}{lll}
A^{\subseteq } & = & \{\alpha \mid \exists \beta .\alpha \beta \in A\} \\ 
A_{1}A_{2} & = & \{\alpha \beta \mid \alpha \in A_{1}\text{ and }\beta \in
A_{2}\} \\ 
A_{1}\leftarrowtail A_{2} & = & \{\alpha \mid \exists \beta .\alpha \beta
\in A_{1}\text{ and }\beta \in A_{2}\}%
\end{array}%
\]%
Note that $\beta $ can take the value $\tau $ in the definition of $%
A_{1}\leftarrowtail A_{2}$, which implies $A_{1}\subset A_{1}\leftarrowtail
A_{2}$. The set $A^{\subseteq }$ has an exponential size, so whenever
possible it should not be computed explicitly.

Let $C$ be a communication set, then we define%
\[
\begin{array}{lll}
C(A) & = & \cup _{\alpha \in A}\text{\textsc{Comm}(}C\text{, }\alpha \text{)}
\\ 
C^{-1}(A) & = & \cup _{\alpha \in A}\text{\textsc{CommInverse}(}C\text{, }%
\alpha \text{)} \\ 
filter_{\nabla }(C,A) & = & \{\gamma \rightarrow c\in C\mid \exists _{\alpha
\in A}.\gamma \subset \alpha \}%
\end{array}%
\]%
where \textsc{Comm} and \textsc{CommInverse} are defined using pseudo code
as follows:%
\[
\begin{array}{l}
\text{\textsc{Comm}(}C\text{, }\alpha \text{)} \\ 
R:=\{\alpha \} \\ 
\text{\textbf{for }}\gamma \rightarrow c\in C\text{ \textbf{do}} \\ 
\qquad \text{\textbf{if }}\exists \beta .\alpha =\beta \gamma \text{ \textbf{%
then} }R:=R\cup \text{\textsc{Comm}(}C\text{, }\beta c\text{)} \\ 
\text{\textbf{return }}R%
\end{array}%
\]%
\[
\begin{array}{l}
\text{\textsc{CommInverse}(}C\text{, }\alpha _{1}\text{,}\alpha _{2}\text{)}
\\ 
R:=\{\alpha _{1}\alpha _{2}\} \\ 
\text{\textbf{for }}\gamma \rightarrow c\in C\text{ \textbf{do}} \\ 
\qquad \text{\textbf{if }}\exists \beta .\alpha _{1}=\beta c\text{ \textbf{%
then} }R:=R\cup \text{\textsc{CommInverse}(}C\text{, }\beta \text{,}\alpha
_{2}\gamma \text{)} \\ 
\text{\textbf{return }}R%
\end{array}%
\]%
Note that $C^{-1}(\alpha )=$\textsc{CommInverse}($C$, $\alpha $,$\tau $).

Let $R$ be a rename set, then we define%
\[
\begin{array}{lll}
R(\alpha ) & = & \{R(\alpha _{i})\mid \alpha _{i}\in \alpha \} \\ 
R^{-1}(\alpha ) & = & \{\beta \mid R(\beta )=\alpha \} \\ 
R(A) & = & \{R(\alpha )\mid \alpha \in A\} \\ 
R^{-1}(A) & = & \{R^{-1}(\alpha )\mid \alpha \in A\}%
\end{array}%
\]%
Let $I$ be a hide set, then we define%
\[
\begin{array}{ccc}
\tau _{I}(A) & = & \{\beta \mid \exists _{\alpha \in A,\gamma \in I^{\ast
}}.\alpha =\beta \gamma \wedge \beta \cap I=\emptyset \} \\ 
\tau _{I}^{-1}(A) & = & \partial _{I}(A)I^{\ast }%
\end{array}%
\]%
Let $B$ be a block set, then we define%
\[
\partial _{B}(A)=\{\alpha \in A\mid \alpha \cap B=\emptyset \} 
\]%
We define a mapping $act$ that extracts the individual action names of a set
of multi action names:%
\[
\begin{array}{lll}
act\left( a_{1}\mid \ldots \mid a_{n}\right) & = & \left\{ a_{1}\mid \ldots
\mid a_{n}\right\} \\ 
act\left( A\right) & = & \bigcup_{\alpha \in A}act\left( \alpha \right)%
\end{array}%
\]

\subsubsection{The mapping $\protect\alpha $}

We define the mapping $\alpha $ as follows. The value $\alpha (p,\emptyset )$
is an over approximation of the alphabet of process expression $p$.%
\[
\begin{array}{lll}
\alpha (\overline{a},W) & = & \{a\} \\ 
\alpha (P,W) & = & 
\begin{array}{l}
\left\{ 
\begin{array}{ll}
\emptyset & \text{if }P\in W \\ 
\alpha (p,W\cup \{P\}) & \text{if }P\notin W,%
\end{array}%
\right. \\ 
\text{ where }P=p\text{ is the equation of }P%
\end{array}
\\ 
\alpha (\delta ,W) & = & \emptyset \\ 
\alpha (\tau ,W) & = & \{\tau \} \\ 
\alpha (p+q,W) & = & \alpha (p,W)\cup \alpha (q,W) \\ 
\alpha (p\cdot q,W) & = & \alpha (p,W)\cup \alpha (q,W) \\ 
\alpha (c\rightarrow p,W) & = & \alpha (p,W) \\ 
\alpha (c\rightarrow p\diamond q,W) & = & \alpha (p,W)\cup \alpha (q,W) \\ 
\alpha (\Sigma _{d:D}p,W) & = & \alpha (p,W) \\ 
\alpha (p\at t,W) & = & \alpha (p,W) \\ 
\alpha (p\ll q,W) & = & \alpha (p,W)\cup \alpha (q,W) \\ 
\alpha (p\parallel q,W) & = & \alpha (p,W)\cup \alpha (q,W)\cup \alpha
(p,W)\alpha (q,W) \\ 
\alpha (p\leftmerge q,W) & = & \alpha (p,W)\cup \alpha (q,W)\cup \alpha
(p,W)\alpha (q,W) \\ 
\alpha (p\mid q,W) & = & \alpha (p,W)\alpha (q,W) \\ 
\alpha (\rho _{R}(p),W) & = & R(\alpha (p,W)) \\ 
\alpha (\partial _{B}(p),W) & = & \partial _{B}(\alpha (p,W)) \\ 
\alpha (\tau _{I}(p),W) & = & \tau _{I}(\alpha (p,W)) \\ 
\alpha (\Gamma _{C}(p),W) & = & C(\alpha (p,W)) \\ 
\alpha (\nabla _{V}(p),W) & = & \alpha (p,W)\cap (V\cup \{\tau \})%
\end{array}%
\]%
Example 1

If $C=\{a\mid b\rightarrow c\}$, then $\alpha (\Gamma _{C}(a(1)\mid
b(2)))=\{a,b,c,a\mid b\}$. Note that the action $c$ does not occur in the
transition system of this process expression.

\paragraph{Example 2}

In the computation of $\left\{ a_{1},a_{2},\ldots ,a_{20}\right\} \cap
\alpha \left( a_{1}\parallel a_{2}\parallel \ldots \parallel a_{20}\right) $
the above mentioned optimization is really needed.

\subsubsection{Computation of the alphabet}

When computing $A\cap \alpha (p,W)$ for some multi action name set $A$, it
may be beneficial to apply an optimization. This is done to keep
intermediate expressions small. We introduce $\alpha (p,W,A)$ $=A\cap \alpha
(p,W)$, and define it as follows:%
\[
\begin{array}{lll}
\alpha (\overline{a},W,A) & = & \left\{ 
\begin{array}{ll}
\{a\} & \text{if }a\in A \\ 
\emptyset & \text{if }a\notin A%
\end{array}%
\right. \\ 
\alpha (P,W,A) & = & 
\begin{array}{l}
\left\{ 
\begin{array}{ll}
\emptyset & \text{if }P\in W \\ 
\alpha (p,W\cup \{P\},A) & \text{if }P\notin W,%
\end{array}%
\right. \\ 
\text{ where }P=p\text{ is the equation of }P%
\end{array}
\\ 
\alpha (p+q,W,A) & = & \alpha (p,W,A)\cup \alpha (q,W,A) \\ 
\alpha (p\cdot q,W,A) & = & \alpha (p,W,A)\cup \alpha (q,W,A) \\ 
\alpha (c\rightarrow p,W,A) & = & \alpha (p,W,A) \\ 
\alpha (c\rightarrow p\diamond q,W,A) & = & \alpha (p,W,A)\cup \alpha (q,W,A)
\\ 
\alpha (\Sigma _{d:D}p,W,A) & = & \alpha (p,W,A) \\ 
\alpha (p\at t,W,A) & = & \alpha (p,W,A) \\ 
\alpha (p\ll q,W,A) & = & \alpha (p,W,A)\cup \alpha (q,W,A) \\ 
\alpha (p\parallel q,W,A) & = & \alpha (p,W,A)\cup \alpha (q,W,A)\cup \alpha
(p,W,A^{\subseteq })\alpha (q,W,A^{\subseteq }) \\ 
\alpha (p\leftmerge q,W,A) & = & \alpha (p,W,A)\cup \alpha (q,W,A)\cup
\alpha (p,W,A^{\subseteq })\alpha (q,W,A^{\subseteq }) \\ 
\alpha (p\mid q,W,A) & = & \alpha (p,W,A^{\subseteq })\alpha
(q,W,A^{\subseteq })%
\end{array}%
\]

\subsubsection{More efficient computation of the alphabet}

The computation of $\alpha (p,W,A)$ can be done more efficiently. We define
the function $proc(p,W)$ as follows:%
\[
\begin{array}{lll}
proc(\overline{a},W) & = & \emptyset \\ 
proc(P,W) & = & \left\{ 
\begin{array}{ll}
\emptyset & \text{if }P\in W \\ 
\{P\}\cup proc(p,W) & \text{if }P\notin W%
\end{array}%
\right. \\ 
proc(p+q,W) & = & proc(p,W)\cup proc(q,W) \\ 
proc(p\cdot q,W) & = & proc(p,W)\cup proc(q,W) \\ 
proc(c\rightarrow p,W) & = & proc(p,W) \\ 
proc(c\rightarrow p\diamond q,W) & = & proc(p,W)\cup proc(q,W) \\ 
proc(\Sigma _{d:D}p,W) & = & proc(p,W) \\ 
proc(p\at t,W) & = & proc(p,W)%
\end{array}%
\]%
Using this function we can change the computation of $\alpha (p,W,A)$ at
three places:%
\[
\begin{array}{lll}
\alpha (p+q,W,A) & = & \alpha (p,W,A)\cup \alpha (q,W\cup proc(p,W),A) \\ 
\alpha (p\cdot q,W,A) & = & \alpha (p,W,A)\cup \alpha (q,W\cup proc(p,W),A)
\\ 
\alpha (c\rightarrow p\diamond q,W,A) & = & \alpha (p,W,A)\cup \alpha
(q,W\cup proc(p,W),A)%
\end{array}%
\]%
Note that the value $proc(p,W)$ can be computed on the fly during the
computation of $\alpha (p,W,A)$.

\subsubsection{Bounded alphabet}

In practice one often wants to compute $\alpha (p,A)=\alpha (\nabla _{A}(p))$%
. This can be computed more efficiently as follows:

\[
\begin{array}{lll}
\alpha (\overline{a},A) & = & \left\{ 
\begin{array}{ll}
\{a\} & \text{if }a\in A \\ 
\emptyset & \text{if }a\notin A%
\end{array}%
\right. \\ 
\alpha (P,A) & = & \alpha (p,A)\text{, where }P=p\text{ is the equation of }P
\\ 
\alpha (p+q,A) & = & \alpha (p,A)\cup \alpha (q,A) \\ 
\alpha (p\cdot q,A) & = & \alpha (p,A)\cup \alpha (q,A) \\ 
\alpha (c\rightarrow p,A) & = & \alpha (p,A) \\ 
\alpha (c\rightarrow p\diamond q,A) & = & \alpha (p,A)\cup \alpha (q,A) \\ 
\alpha (\Sigma _{d:D}p,A) & = & \alpha (p,A) \\ 
\alpha (p\at t,A) & = & \alpha (p,A) \\ 
\alpha (p\ll q,A) & = & \alpha (p,A)\cup \alpha (q,A) \\ 
\alpha (p\parallel q,A) & = & \alpha (p,A)\cup \alpha (q,A)\cup \alpha
(p,A^{\subseteq })\alpha (q,A\leftarrowtail \alpha (p,A^{\subseteq })) \\ 
\alpha (p\leftmerge q,A) & = & \alpha (p,A)\cup \alpha (q,A)\cup \alpha
(p,A^{\subseteq })\alpha (q,A\leftarrowtail \alpha (p,A^{\subseteq })) \\ 
\alpha (p\mid q,A) & = & \alpha (p,A^{\subseteq })\alpha (q,A\leftarrowtail
\alpha (p,A^{\subseteq })) \\ 
\alpha (\rho _{R}(p),A) & = & R(\alpha (p,R^{-1}(A))) \\ 
\alpha (\partial _{B}(p),A) & = & \alpha (p,\partial _{B}(A)) \\ 
\alpha (\tau _{I}(p),A) & = & \tau _{I}(\alpha (p,\tau _{I}^{-1}(A))) \\ 
\alpha (\Gamma _{C}(p),A) & = & C(\alpha (p,C^{-1}(A))) \\ 
\alpha (\nabla _{V}(p),A) & = & \alpha (p,A\cap V))%
\end{array}%
\]

\subsubsection{The mappings $push$, $push_{\protect\nabla }$ and $%
push_{\partial }$}

We define mappings $push$, $push_{\nabla }$ and $push_{\partial }$ such that 
$push(p)$ is bisimulation equivalent to $p$, $push_{\nabla }(A,p)$ is
bisimulation equivalent to $\nabla _{A}(p)$, and $push_{\partial }(B,p)$ is
bisimulation equivalent to $\partial _{B}(p)$. The goal of these mappings is
to push allow and block expressions deeply inside process expressions. It is
important to know that an allow set $A$ in the expression $\nabla _{A}(p)$
implicitly contains the empty multi action $\tau $. Let $\mathcal{E}$ $%
=\{P_{1}(d)=p_{1},\ldots ,P_{n}(d)=p_{n}\}$ be a sequence of process
equations.%
\[
\begin{tabular}{lll}
$push(p)$ & $=$ & $p$ if $p$ is a pCRL expression \\ 
$push(p\parallel q)$ & $=$ & $push\left( p\right) \parallel push\left(
q\right) $ \\ 
$push(p\leftmerge q)$ & $=$ & $push\left( p\right) \leftmerge push\left(
q\right) $ \\ 
$push(p\mid q)$ & $=$ & $push\left( p\right) \mid push\left( q\right) $ \\ 
$push(\rho _{R}(p))$ & $=$ & $\rho _{R}(push\left( p\right) )$ \\ 
$push(\partial _{B}(p))$ & $=$ & $push_{\partial }(B,p)$ \\ 
$push(\tau _{I}(p))$ & $=$ & $\tau _{I}(push\left( p\right) )$ \\ 
$push(\Gamma _{C}(p))$ & $=$ & $\Gamma _{C}\left( push\left( p\right)
\right) $ \\ 
$push(\nabla _{V}(p))$ & $=$ & $push_{\nabla }(V,p)$%
\end{tabular}%
\]%
We assume that $P_{A,e}^{\nabla }$ is a unique name for every $P\in
\{P_{1},\ldots ,P_{n}\}$, multi action name set $A$ and sequence of data
expressions $e$.%
\[
\begin{tabular}[t]{lll}
$push_{\nabla }\left( A,\overline{a}\right) $ & $=$ & $\left\{ 
\begin{array}{ll}
\overline{a} & \text{if }N(\overline{a})\in A \\ 
\delta  & \text{otherwise}%
\end{array}%
\right. $ \\ 
$push_{\nabla }\left( A,P\left( e\right) \right) $ & $=$ & $%
\begin{array}{l}
P_{A}^{\nabla }(e)\text{, where }P(d)=p\text{ is the equation of }P\text{,
and} \\ 
\text{where }P_{A}^{\nabla }(d)=push_{\nabla }\left( A,p\right) \text{ is a
new equation}%
\end{array}%
$ \\ 
$push_{\nabla }\left( A,\delta \right) $ & $=$ & $\delta $ \\ 
$push_{\nabla }\left( A,\tau \right) $ & $=$ & $\tau $ \\ 
$push_{\nabla }\left( A,p+q\right) $ & $=$ & $\nabla _{A}(p+q)$ \\ 
$push_{\nabla }\left( A,p\cdot q\right) $ & $=$ & $\nabla _{A}(p\cdot q)$ \\ 
$push_{\nabla }\left( A,c\rightarrow p\right) $ & $=$ & $\nabla _{A}\left(
c\rightarrow p\right) $ \\ 
$push_{\nabla }\left( A,c\rightarrow p\diamond q\right) $ & $=$ & $\nabla
_{A}\left( c\rightarrow p\diamond q\right) $ \\ 
$push_{\nabla }\left( A,\Sigma _{d:D}p\right) $ & $=$ & $\nabla _{A}\left(
\Sigma _{d:D}p\right) $ \\ 
$push_{\nabla }\left( A,p\at t\right) $ & $=$ & $\nabla _{A}\left( p\at %
t\right) $ \\ 
$push_{\nabla }\left( A,p\ll q\right) $ & $=$ & $\nabla _{A}\left( p\ll
q\right) $ \\ 
$push_{\nabla }(A,p\parallel q)$ & $=$ & $\mathsf{\nabla _{A}}(A,p^{\prime
}\parallel q^{\prime })\text{ where}\left\{ \text{ }%
\begin{array}{lll}
p^{\prime } & = & push_{\nabla }(A^{\subseteq },p) \\ 
q^{\prime } & = & push_{\nabla }(A\leftarrowtail \alpha (p^{\prime }),q)%
\end{array}%
\right. $ \\ 
$push_{\nabla }(A,p\leftmerge q)$ & $=$ & $\mathsf{\nabla _{A}}(A,p^{\prime }%
\leftmerge q^{\prime })\text{ where}\left\{ \text{ }%
\begin{array}{lll}
p^{\prime } & = & push_{\nabla }(A^{\subseteq },p) \\ 
q^{\prime } & = & push_{\nabla }(A\leftarrowtail \alpha (p^{\prime }),q)%
\end{array}%
\right. $ \\ 
$push_{\nabla }(A,p\mid q)$ & $=$ & $\mathsf{\nabla _{A}}(A,p^{\prime }\mid
q^{\prime })$ where$\left\{ \text{ }%
\begin{array}{lll}
p^{\prime } & = & push_{\nabla }(A^{\subseteq },p) \\ 
q^{\prime } & = & push_{\nabla }(A\leftarrowtail \alpha (p^{\prime }),q)%
\end{array}%
\right. $ \\ 
$push_{\nabla }(A,\rho _{R}(p))$ & $=$ & $\rho _{R}(p^{\prime })$ where $%
p^{\prime }=$ $push_{\nabla }(R^{-1}(A),p)$ \\ 
$push_{\nabla }(A,\partial _{B}(p))$ & $=$ & $push_{\nabla }(\partial
_{B}(A),p)$ \\ 
$push_{\nabla }(A,\tau _{I}(p))$ & $=$ & $\tau _{I}(p^{\prime })$ where $%
p^{\prime }=push_{\nabla }(\tau _{I}^{-1}(A),p)$ \\ 
$push_{\nabla }(A,\Gamma _{C}(p))$ & $=$ & $\mathsf{allow}(A,\Gamma
_{C}(p^{\prime }))$ where $p^{\prime }=push_{\nabla }(C^{-1}(A),p)$ \\ 
$push_{\nabla }(A,\nabla _{V}(p))$ & $=$ & $push_{\nabla }(A\cap V,p),$%
\end{tabular}%
\]

\paragraph{Optimizations}

During the computation of $push_{\nabla }$ the following optimizations are
applied in the right hand side of each equation:%
\[
\begin{array}{lll}
\nabla _{A}(p) & = & \left\{ 
\begin{array}{ll}
p & \text{if }(A\cup \{\tau \})\cap \alpha (p)=\alpha (p) \\ 
\nabla _{A\cap \alpha (p)}(p) & \text{otherwise}%
\end{array}%
\right.  \\ 
\nabla _{\emptyset }(p) & = & \left\{ 
\begin{array}{ll}
\tau  & \text{if }p=\tau  \\ 
\delta  & \text{otherwise}%
\end{array}%
\right.  \\ 
\Gamma _{C}(p) & = & \Gamma _{filter_{\nabla }(C,\alpha (p))}(p) \\ 
\delta \mid \delta  & = & \delta  \\ 
\delta \parallel \delta  & = & \delta 
\end{array}%
\]%
For non pCRL expression the alphabet $\alpha (p)$ is computed on the fly
during the computation of $push_{\nabla }\left( A,p\right) $.

\paragraph{Example 1}

Let $P=(a+b)\cdot P$. Then $push_{\nabla }\left( \{a\},P,\emptyset \right)
=P^{\prime }$, with $P^{\prime }=push_{\nabla }\left( \{a\},(a+b)\cdot
P,\{(P,\{a\},P^{\prime })\}\right) =push_{\nabla }\left(
\{a\},(a+b),\{(P,\{a\},P^{\prime })\}\right) \cdot push_{\nabla }\left(
\{a\},P,\{(P,\{a\},P^{\prime })\}\right) =\cdots =a\cdot P^{\prime }$.

\paragraph{Example 2}

Let $P=a\cdot \nabla _{\{a\}}(P)$. Then $push_{\nabla }\left(
\{a\},P,\emptyset \right) =P^{\prime }$, with $P^{\prime }=push_{\nabla
}\left( \{a\},a\cdot \nabla _{\{a\}}(P),\{(P,\{a\},P^{\prime })\}\right)
=push_{\nabla }\left( \{a\},a,\{(P,\{a\},P^{\prime })\}\right) \cdot
push_{\nabla }\left( \{a\},\nabla _{\{a\}}(P),\{(P,\{a\},P^{\prime
})\}\right) =\cdots =a\cdot P^{\prime }$.

We assume that $P_{A,e}^{\partial }$ is a unique name for every $P\in
\{P_{1},\ldots ,P_{n}\}$, multi action name set $A$ and sequence of data
expressions $e$.%
\[
\begin{tabular}{lll}
$push_{\partial }(B,\overline{a})$ & $=$ & $\left\{ \text{ }%
\begin{array}{lll}
\overline{a} &  & \text{if }N(\overline{a})\cap B=\emptyset \\ 
\delta &  & \text{otherwise}%
\end{array}%
\right. $ \\ 
$push_{\partial }(B,P(e))$ & $=$ & $%
\begin{array}{l}
P_{B,e}^{\partial }(e) \\ 
\text{where }P(d)=p\text{ is the equation of }P\text{, and} \\ 
\text{where }P_{B,e}^{\partial }(d)=push_{\partial }\left( B,p\right) \text{
is a new equation}%
\end{array}%
$ \\ 
$push_{\partial }(B,\delta )$ & $=$ & $\delta $ \\ 
$push_{\partial }(B,\tau )$ & $=$ & $\tau $ \\ 
$push_{\partial }(B,p+q)$ & $=$ & $push_{\partial }(B,p)+push_{\partial
}(B,q)$ \\ 
$push_{\partial }(B,p\cdot q)$ & $=$ & $push_{\partial }(B,p)\cdot
push_{\partial }(B,q)$ \\ 
$push_{\partial }(B,c\rightarrow p)$ & $=$ & $c\rightarrow push_{\partial
}(B,p)$ \\ 
$push_{\partial }(B,c\rightarrow p\diamond q)$ & $=$ & $c\rightarrow
push_{\partial }(B,p)\diamond push_{\partial }(B,q)$ \\ 
$push_{\partial }(B,\Sigma _{d:D}p)$ & $=$ & $\Sigma _{d:D}push_{\partial
}(B,p)$ \\ 
$push_{\partial }(B,p\at t)$ & $=$ & $push_{\partial }(B,p)\at t$ \\ 
$push_{\partial }(B,p\ll q)$ & $=$ & $push_{\partial }(B,p)\ll
push_{\partial }(B,q)$ \\ 
$push_{\partial }(B,p\parallel q)$ & $=$ & $push_{\partial }\left(
B,p\right) \parallel push_{\partial }\left( B,q\right) $ \\ 
$push_{\partial }(B,p\leftmerge q)$ & $=$ & $push_{\partial }\left(
B,p\right) \leftmerge push_{\partial }\left( B,q\right) $ \\ 
$push_{\partial }(B,p\mid q)$ & $=$ & $push_{\partial }\left( B,p\right)
\mid push_{\partial }\left( B,q\right) $ \\ 
$push_{\partial }(B,\rho _{R}(p))$ & $=$ & $\rho _{R}\left( R^{-1}\left(
B\right) ,p\right) $ \\ 
$push_{\partial }(B,\partial _{B_{1}}(p))$ & $=$ & $push_{\partial }(B\cup
B_{1},p)$ \\ 
$push_{\partial }(B,\tau _{I}(p))$ & $=$ & $\tau _{I}\left( push_{\partial
}\left( B\setminus I,p\right) \right) $ \\ 
$push_{\partial }(B,\Gamma _{C}(p))$ & $=$ & $\mathsf{block}(B,\Gamma
_{C}\left( push_{\partial }\left( B^{\prime },p\right) \right) $ where $%
B^{\prime }=B\setminus \left\{ b\in B\mid \exists _{\gamma \rightarrow c\in
C}.b\in \gamma \wedge c\notin B\right\} $ \\ 
$push_{\partial }(B,\nabla _{V}(p))$ & $=$ & $push_{\nabla }(\partial
_{B}(A),p,\emptyset ),$%
\end{tabular}%
\]%
where%
\[
\begin{array}{lll}
\mathsf{block}(B,p) & = & \left\{ 
\begin{array}{ll}
p & \text{if }B=\emptyset \\ 
\partial _{B}(p) & \text{otherwise}%
\end{array}%
\right.%
\end{array}%
\]

\paragraph{Example 3}

The presence of $R^{-1}(\partial _{B}(A))$ instead of just $R^{-1}(A)$ in
the right hand side of the rename operator is explained by the example $%
push_{\nabla }(\{b\},\rho _{\{b\rightarrow c\}}b)$. We see that $\rho
_{\{b\rightarrow c\}}push_{\nabla }(R^{-1}(A),p)=\rho _{\{b\rightarrow
c\}}push_{\nabla }(\{b\},b)=\rho _{\{b\rightarrow c\}}b=c$, which is clearly
the wrong answer.

\subsubsection{Allow sets}

There are two rules in the definition of $push_{\nabla }$ where the allow
set can/should not be computed explicitly. The computation of $push_{\nabla
}(A,p\parallel q)$ involves computation of $push_{\nabla }(p,A^{\subseteq }%
\dot{)}$. We want to avoid the computation of $A^{\subseteq }$, since it can
become very large. The computation of $push_{\nabla }(A,\tau _{I}(p))$
involves computation of $push_{\nabla }(p,\tau _{I}^{-1}(A)\dot{)}$. The set 
$\tau _{I}^{-1}(A)=AI^{\ast }$ is infinite.

In the implementation we use allow sets of the form $A^{\subseteq }I^{\ast
}, $ where $A$ is a set of multi action names and $I$ is a set of action
names. The $^{\subseteq }$ is optional and $I$ may be empty. Such an allow
set is stored as two sets $A$ and $I$, together with an attribute that tells
if $^{\subseteq }$ is appicable. We need to show that allow sets are closed
under the operations in $push_{\nabla }$.%
\[
\begin{array}{lll}
\partial _{B}(A^{\subseteq }I^{\ast }) & = & \tau _{B}(A)^{\subseteq }\tau
_{B}(I)^{\ast } \\ 
\tau _{I_{1}}^{-1}\left( A^{\subseteq }I^{\ast }\right) & = & \partial
_{I_{1}}(A^{\subseteq })\left( I\cup I_{1}\right) ^{\ast } \\ 
\left( A^{\subseteq }I^{\ast }\right) \cap V & = & \{\beta \in V\mid \exists
_{\alpha \in A}.\tau _{I}(\beta )\sqsubseteq \alpha \} \\ 
R^{-1}\left( A^{\subseteq }I^{\ast }\right) & = & R^{-1}\left( A^{\subseteq
}\right) R^{-1}\left( I\right) ^{\ast } \\ 
C^{-1}\left( A^{\subseteq }I^{\ast }\right) & \subseteq & C^{-1}\left(
A\right) ^{\subseteq }act\left( C^{-1}\left( I\right) \right) ^{\ast } \\ 
\left( A^{\subseteq }I^{\ast }\right) \leftarrowtail A_{1} & = & 
A^{\subseteq }I^{\ast } \\ 
\left( A^{\subseteq }I^{\ast }\right) ^{\subseteq } & = & A^{\subseteq
}I^{\ast } \\ 
\partial _{B}(AI^{\ast }) & = & \partial _{B}(A)\tau _{B}(I)^{\ast } \\ 
\tau _{I_{1}}^{-1}\left( AI^{\ast }\right) & = & \partial _{I_{1}}(A)\left(
I\cup I_{1}\right) ^{\ast } \\ 
\left( AI^{\ast }\right) \cap V & = & \{\beta \in V\mid \exists _{\alpha \in
A}.\tau _{I}(\beta )=\alpha \} \\ 
R^{-1}\left( AI^{\ast }\right) & = & R^{-1}\left( A\right) R^{-1}\left(
I\right) ^{\ast } \\ 
C^{-1}\left( AI^{\ast }\right) & \subseteq & C^{-1}\left( A\right) act\left(
C^{-1}\left( I\right) \right) ^{\ast } \\ 
\left( AI^{\ast }\right) ^{\subseteq } & = & A^{\subseteq }I^{\ast }%
\end{array}%
\]%
where we used the following properties:%
\[
\begin{array}{lll}
\partial _{B}\left( A_{1}A_{2}\right) & = & \partial _{B}\left( A_{1}\right)
\partial _{B}\left( A_{1}\right) \\ 
\partial _{B}\left( A^{\subseteq }\right) & = & \tau _{B}(A)^{\subseteq } \\ 
R^{-1}\left( A_{1}A_{2}\right) & = & R^{-1}\left( A_{1}\right) R^{-1}\left(
A_{2}\right) \\ 
R^{-1}\left( A^{\ast }\right) & = & R^{-1}\left( A\right) ^{\ast } \\ 
C^{-1}\left( A^{\subseteq }\right) & \subseteq & C^{-1}\left( A\right)
^{\subseteq } \\ 
C^{-1}\left( A_{1}A_{2}\right) & = & C^{-1}\left( A_{1}\right) C^{-1}\left(
A_{2}\right) \\ 
C^{-1}\left( A^{\ast }\right) & = & C^{-1}\left( A\right) ^{\ast } \\ 
A^{\subseteq }\leftarrowtail A_{1} & = & A^{\subseteq }%
\end{array}%
\]%
Note that in case of the communication we only have an inclusion relation
instead of equality. This is done to stay within the format $A^{\subseteq
}I^{\ast }$. As a consequence the implementation uses an over-approximation
of $C^{-1}\left( A^{\subseteq }I^{\ast }\right) $ and $C^{-1}\left( AI^{\ast
}\right) $. Furthermore note that the property $R^{-1}\left( A^{\subseteq
}\right) =R^{-1}\left( A\right) ^{\subseteq }$ does not hold. A counter
example is $R=\{b\rightarrow a\}$ and $A=\{a,b\mid c\}$. In that case we
have $R^{-1}\left( A^{\subseteq }\right) =\{a,b,c\}^{\subseteq }$ and $%
R^{-1}\left( A\right) ^{\subseteq }=\{a,b\}^{\subseteq }$. Another property
that was initially assumed, but that does not hold is $\left( AI^{\ast
}\right) \leftarrowtail A_{1}=\left( A\leftarrowtail \tau _{I}(A_{1})\right)
I^{\ast }$.

\newpage

\subsection{Optimization for $push_{\protect\nabla }$}

In some cases the $push_{\nabla }$ operator produces expressions that are
too large. This section proposes an optimization for the case $push_{\nabla
}(A,\Gamma _{C}(p))$ that can help to prevent this problem for certain
practical cases.

\[
\begin{tabular}[t]{lll}
$push_{\nabla }(A,\Gamma _{C}(p))$ & $=$ & $\left\{ 
\begin{array}{ll}
\mathsf{allow}(A,\Gamma _{C\setminus C^{\prime }}(push_{\nabla \Gamma
}(A^{\prime },C^{\prime },p))) & \text{if }C\neq C^{\prime } \\ 
push_{\nabla \Gamma }(A,C,p)) & \text{otherwise,}%
\end{array}%
\right. $%
\end{tabular}%
\]%
with $C^{\prime }=\{\beta \rightarrow b\in C\mid b\notin
\bigcup\limits_{\beta ^{\prime }\rightarrow b^{\prime }\in C}\beta ^{\prime
}\}$ and $A^{\prime }=((C\setminus C^{\prime })(A))^{\subseteq }$ and%
\[
\begin{tabular}[t]{lll}
$push_{\nabla \Gamma }(A,C,p\parallel q)$ & $=$ & $\mathsf{allow}\left(
A,\Gamma _{C}\left( \mathsf{allow}\left( C^{-1}(A),p^{\prime }\parallel
q^{\prime }\right) \right) \right) \text{ where}\left\{ \text{ }%
\begin{array}{lll}
p^{\prime } & = & push_{\nabla \Gamma }(A^{\prime },C,p) \\ 
q^{\prime } & = & push_{\nabla \Gamma }(A^{\prime \prime },C,q) \\ 
A^{\prime } & = & C^{-1}(A)^{\subseteq }\setminus \left( C^{-1}(A)\setminus
A\right) \\ 
A^{\prime \prime } & = & \left( C^{-1}(A)\leftarrowtail \alpha (p^{\prime
})\right) \setminus \left( C^{-1}(A)\setminus A\right)%
\end{array}%
\right. $ \\ 
$push_{\nabla \Gamma }(A,C,p\leftmerge q)$ & $=$ & $\mathsf{allow}\left(
A,\Gamma _{C}\left( \mathsf{allow}\left( C^{-1}(A),p^{\prime }\leftmerge %
q^{\prime }\right) \right) \right) \text{ where}\left\{ \text{ }%
\begin{array}{lll}
p^{\prime } & = & push_{\nabla \Gamma }(A^{\prime },C,p) \\ 
q^{\prime } & = & push_{\nabla \Gamma }(A^{\prime \prime },C,q) \\ 
A^{\prime } & = & C^{-1}(A)^{\subseteq }\setminus \left( C^{-1}(A)\setminus
A\right) \\ 
A^{\prime \prime } & = & \left( C^{-1}(A)\leftarrowtail \alpha (p^{\prime
})\right) \setminus \left( C^{-1}(A)\setminus A\right)%
\end{array}%
\right. $ \\ 
$push_{\nabla \Gamma }(A,C,p\mid q)$ & $=$ & $\mathsf{allow}\left( A,\Gamma
_{C}\left( \mathsf{allow}\left( C^{-1}(A),p^{\prime }\mid q^{\prime }\right)
\right) \right) \text{ where}\left\{ \text{ }%
\begin{array}{lll}
p^{\prime } & = & push_{\nabla \Gamma }(A^{\prime },C,p) \\ 
q^{\prime } & = & push_{\nabla \Gamma }(A^{\prime \prime },C,q) \\ 
A^{\prime } & = & C^{-1}(A)^{\subseteq }\setminus \left( C^{-1}(A)\setminus
A\right) \\ 
A^{\prime \prime } & = & \left( C^{-1}(A)\leftarrowtail \alpha (p^{\prime
})\right) \setminus \left( C^{-1}(A)\setminus A\right)%
\end{array}%
\right. $ \\ 
$push_{\nabla \Gamma }(A,C,\partial _{B}(p))$ & $=$ & $push_{\nabla \Gamma
}(\partial _{B}(A),C,p)$ \\ 
$push_{\nabla \Gamma }(A,C,\nabla _{V}(p))$ & $=$ & $push_{\nabla \Gamma
}(A\cap V,C,p)$ \\ 
$push_{\nabla \Gamma }(A,C,p)$ & $=$ & $\mathsf{allow}(A,\Gamma
_{C}(p^{\prime }))$ where $p^{\prime }=push_{\nabla }(C^{-1}(A),p)$ for all
other cases of $p$%
\end{tabular}%
\]%
Note that in this case the allow set $A$ has the general shape $\left(
A_{1}^{\subseteq }\setminus A_{2}^{\subseteq }\right) I^{\ast }$ (?), with
the subset operator $\subseteq $ optional, and with $I$ possibly empty. To
implement this optimization, it needs to be investigated if such a set $A$
is closed under the operations $\partial _{B}(A)$, $\tau _{I_{1}}^{-1}(A)$, $%
A\cap V$, $R^{-1}(A)$, $C^{-1}(A)$, $A\leftarrowtail A_{1}$, $A^{\subseteq }$
and $C(A)$.

\end{document}
