% at
% left merge
% \RequirePackage{stmaryrd}
%%%%%%%%%%\input{tcilatex}
%%%%%%%%%\input{tcilatex}
%%%%%%%%\input{tcilatex}
%%%%%%%\input{tcilatex}
%%%%%%\input{tcilatex}
%%%%%\input{tcilatex}
%%%%\input{tcilatex}
%%%\input{tcilatex}
%%\input{tcilatex}


\documentclass{article}
%%%%%%%%%%%%%%%%%%%%%%%%%%%%%%%%%%%%%%%%%%%%%%%%%%%%%%%%%%%%%%%%%%%%%%%%%%%%%%%%%%%%%%%%%%%%%%%%%%%%%%%%%%%%%%%%%%%%%%%%%%%%%%%%%%%%%%%%%%%%%%%%%%%%%%%%%%%%%%%%%%%%%%%%%%%%%%%%%%%%%%%%%%%%%%%%%%%%%%%%%%%%%%%%%%%%%%%%%%%%%%%%%%%%%%%%%%%%%%%%%%%%%%%%%%%%
\usepackage{amssymb}
\usepackage{geometry}
\usepackage{stmaryrd}

%TCIDATA{OutputFilter=LATEX.DLL}
%TCIDATA{Version=5.50.0.2890}
%TCIDATA{<META NAME="SaveForMode" CONTENT="1">}
%TCIDATA{BibliographyScheme=Manual}
%TCIDATA{Created=Friday, June 15, 2012 17:58:29}
%TCIDATA{LastRevised=Tuesday, November 20, 2012 14:51:27}
%TCIDATA{<META NAME="GraphicsSave" CONTENT="32">}
%TCIDATA{<META NAME="DocumentShell" CONTENT="Standard LaTeX\Blank - Standard LaTeX Article">}
%TCIDATA{CSTFile=40 LaTeX article.cst}

\newtheorem{theorem}{Theorem}
\newtheorem{acknowledgement}[theorem]{Acknowledgement}
\newtheorem{algorithm}[theorem]{Algorithm}
\newtheorem{axiom}[theorem]{Axiom}
\newtheorem{case}[theorem]{Case}
\newtheorem{claim}[theorem]{Claim}
\newtheorem{conclusion}[theorem]{Conclusion}
\newtheorem{condition}[theorem]{Condition}
\newtheorem{conjecture}[theorem]{Conjecture}
\newtheorem{corollary}[theorem]{Corollary}
\newtheorem{criterion}[theorem]{Criterion}
\newtheorem{definition}[theorem]{Definition}
\newtheorem{example}[theorem]{Example}
\newtheorem{exercise}[theorem]{Exercise}
\newtheorem{lemma}[theorem]{Lemma}
\newtheorem{notation}[theorem]{Notation}
\newtheorem{problem}[theorem]{Problem}
\newtheorem{proposition}[theorem]{Proposition}
\newtheorem{remark}[theorem]{Remark}
\newtheorem{solution}[theorem]{Solution}
\newtheorem{summary}[theorem]{Summary}
\newenvironment{proof}[1][Proof]{\noindent\textbf{#1.} }{\ \rule{0.5em}{0.5em}}
\geometry{left=1in,right=1in,top=1in,bottom=1in}
\input{include/tcilatex}
\font \aap cmmi10
\newcommand{\at}{\mathbin{\mbox{\aap ,}}}
\newcommand{\leftm}{\mathbin{\mathrel{\llfloor}}}
%\input{tcilatex}

\begin{document}


\section{Process Library Implementation Notes}

\subsection{Processes}

Process expressions in mCRL2 are expressions built according to the
following syntax:%
\[
\begin{array}{ccc}
\text{expression} & \text{C++ equivalent} & \text{ATerm grammar} \\ 
a(e) & \text{action(}a\text{,}e\text{)} & \text{Action} \\ 
P(e) & \text{process(}P\text{,}e\text{)} & \text{Process} \\ 
P(d:=e) & \text{process\_assignment(}P\text{,}d:=e\text{)} & \text{%
ProcessAssignment} \\ 
\delta & \text{delta()} & \text{Delta} \\ 
\tau & \text{tau()} & \text{Tau} \\ 
\dsum\limits_{d}x & \text{sum(}d\text{,}x\text{)} & \text{Sum} \\ 
\partial _{B}(x) & \text{block(}B\text{,}x\text{)} & \text{Block} \\ 
\tau _{B}(x) & \text{hide(}B\text{,}x\text{)} & \text{Hide} \\ 
\rho _{R}(x) & \text{rename(}R\text{,}x\text{)} & \text{Rename} \\ 
\Gamma _{C}(x) & \text{comm(}C\text{,}x\text{)} & \text{Comm} \\ 
\bigtriangledown _{V}(x) & \text{allow(}V\text{,}x\text{)} & \text{Allow} \\ 
x\mid y & \text{sync(}x\text{,}y\text{)} & \text{Sync} \\ 
x\mbox{\aap ,}t & \text{at\_time(}x\text{,}t\text{)} & \text{AtTime} \\ 
x\cdot y & \text{seq(}x\text{,}y\text{)} & \text{Seq} \\ 
c\rightarrow x & \text{if\_then(}c\text{,}x\text{)} & \text{IfThen} \\ 
c\rightarrow x\diamond y & \text{if\_then\_else(}c\text{,}x\text{,}y\text{)}
& \text{IfThenElse} \\ 
x\ll y & \text{binit(}x\text{,}y\text{)} & \text{BInit} \\ 
x\ \parallel \ y & \text{merge(}x\text{,}y\text{)} & \text{Merge} \\ 
x\ \mathbin{\mathrel{\llfloor}}\ y & \text{lmerge(}x\text{,}y\text{)} & 
\text{LMerge} \\ 
x+y & \text{choice(}x\text{,}y\text{)} & \text{Choice}%
\end{array}%
\]

where the types of the symbols are as follows:%
\[
\begin{array}{cl}
a,b & \text{strings (action names)} \\ 
P & \text{a process identifier} \\ 
e & \text{a sequence of data expressions} \\ 
d & \text{a sequence of data variables} \\ 
B & \text{a set of strings (action names) } \\ 
R & \text{a sequence of rename expressions} \\ 
C & \text{a sequence of communication expressions} \\ 
V & \text{a sequence of multi actions} \\ 
t & \text{a data expression of type real} \\ 
x,y & \text{process expressions} \\ 
c & \text{ a data expression of type bool}%
\end{array}%
\]%
A rename expression is of the form $a\rightarrow b$, with $a$ and $b$ action
names. A multi action is of the form $a_{1}\ |\ \cdots \ |\ a_{n}$, with $%
a_{i}$ actions. A communication expression is of the form $b_{1}\ |\ \cdots
\ |\ b_{n}\rightarrow b$, with $b$ and $b_{i}$ action names.

\subsubsection{Restrictions}

A multi action is a multi set of actions. The left hand sides of the
communication expressions in $C$ must be unique. Also the left hand sides of
the rename expressions in $R$ must be unique.

\newpage

\subsubsection{Linear process expressions}

Linear process expressions are a subset of process expresions satisfying the
following grammar:
\begin{verbatim}
<linear process expression> ::= choice(<linear process expression>, <linear process expression>)
                              | <summand>
 
<summand>                   ::= sum(<variables>, <alternative>)
                              | <conditional action prefix>
                              | <conditional deadlock>
 
<conditional action prefix> ::= if_then(<condition>, <action prefix>)
                              | <action prefix>
 
<action prefix>             ::= seq(<timed multiaction>, <process reference>)
                              | <timed multiaction>
 
<timed multiaction>         ::= at_time(<multiaction>, <time stamp>)
                              | <multiaction>
 
<multiaction>               ::= tau()
                              | <action>
                              | sync(<multiaction>, <multiaction>)
 
<conditional deadlock>      ::= if_then(<condition>, <timed deadlock>)
                              | <timed deadlock>
 
<timed deadlock>            ::= delta()
                              | at_time(delta(), <time stamp>)
 
<process reference>         ::= process(<process identifier>, <data expressions>)
                              | process_assignment(<process identifier>, <data assignments>)
\end{verbatim}

\newpage

\subsection{Guarded process expressions}

We define the predicate $is\_guarded$ for process expressions as follows: $%
is\_guarded(p)=is\_guarded(p,\emptyset )$%
\[
\begin{array}{lll}
is\_guarded(a(e),W) & = & true \\ 
is\_guarded(\delta ,W) & = & true \\ 
is\_guarded(\tau ,W) & = & true \\ 
is\_guarded(P(e),W) & = & 
\begin{array}{l}
\left\{ 
\begin{array}{ll}
false & \text{if }P\in W \\ 
is\_guarded(p,W\cup \{P\}) & \text{if }P\notin W%
\end{array}%
\right. \\ 
\text{where }P(d)=p\text{ is the equation corresponding to }P(e)%
\end{array}
\\ 
is\_guarded(p+q,W) & = & is\_guarded(p,W)\wedge is\_guarded(q,W) \\ 
is\_guarded(p\cdot q,W) & = & is\_guarded(p,W) \\ 
is\_guarded(c\rightarrow p,W) & = & is\_guarded(p,W) \\ 
is\_guarded(c\rightarrow p\diamond q,W) & = & is\_guarded(p,W)\wedge
is\_guarded(q,W) \\ 
is\_guarded(\Sigma _{d:D}\ p,W) & = & is\_guarded(p,W) \\ 
is\_guarded(p^{\mathbin{\mbox{\aap ,}}}t,W) & = & is\_guarded(p,W) \\ 
is\_guarded(p\ll q,W) & = & is\_guarded(p,W) \\ 
is\_guarded(p\parallel q,W) & = & is\_guarded(p,W)\wedge is\_guarded(q,W) \\ 
is\_guarded(p\mathbin{\mathrel{\llfloor}}q,W) & = & is\_guarded(p,W) \\ 
is\_guarded(p\mid q,W) & = & is\_guarded(p,W)\wedge is\_guarded(q,W) \\ 
is\_guarded(\rho _{R}(p),W) & = & is\_guarded(p,W) \\ 
is\_guarded(\partial _{B}(p),W) & = & is\_guarded(p,W) \\ 
is\_guarded(\tau _{I}(p),W) & = & is\_guarded(p,W) \\ 
is\_guarded(\Gamma _{C}(p),W) & = & is\_guarded(p,W) \\ 
is\_guarded(\nabla _{V}(p),W) & = & is\_guarded(p,W)%
\end{array}%
\]%
\newline
N.B. This specification assumes that process names are unique. In mCRL2
process names can be overloaded, therefore in the implemenation $W$ contains 
\emph{process identifiers} (i.e. both the process name and the sorts of the
arguments) instead of process names.

\newpage

\subsection{Alphabet reduction}

Alphabet reduction is a preprocessing step for linearization. It is a
transformation on process expressions that preserves branching bisimulation.

\subsubsection{Notations}

In this text action names are represented using $a,b,\ldots $ and multi
action names using $\alpha ,\beta ,\ldots $ So in general we have $\alpha
=a_{1}\mid \ldots \mid a_{n}$. In alphabet reduction data parameters play a
minor role, therefore we choose a notation in which data parameters are
omitted. We use the abbreviation $\overline{a}=a(e_{1},\ldots ,e_{n})$ to
denote an action, and $\overline{\alpha }=\overline{a_{1}}\mid \ldots \mid 
\overline{a_{n}}$ to denote a multi action, where $e_{1},\ldots ,e_{n}$ are
data expressions.Note that a multi action is a multiset (or bag) of actions
and a multi action name is a multiset of names. We write $\alpha \beta $ as
shorthand for $\alpha \cup \beta $ and $a\beta $ for $\{a\}\cup \beta $.
Sets of multi action names are represented using $A,A_{1},A_{2},\ldots $ A
communication $C$ maps multi action names to action names, and is denoted as 
$\{\alpha _{1}\rightarrow a_{1},\ldots ,\alpha _{n}\rightarrow a_{n}\}$. A
renaming $R$ is a substitution on action names, and is denoted as $%
R=\{a_{1}\rightarrow b_{1},\ldots ,a_{n}\rightarrow b_{n}\}$. A block set $B$
is a set of action names. A hide set $I$ is a set of action names.

\subsubsection{Definitions}

We define multi actions $\overline{\alpha }$ using the following grammar:%
\[
\overline{\alpha }:=\overline{a}\shortmid \overline{\alpha }\mid \overline{a}%
, 
\]%
where $\overline{a}$ is an action, and where $\shortmid $ is used to
distinguish alternatives.

We define pCRL terms $p$ using the following grammar:%
\[
p::=\overline{a}\shortmid P\shortmid \delta \shortmid \tau \shortmid
p+p\shortmid p\cdot p\shortmid c\rightarrow p\shortmid c\rightarrow
p\diamond p\shortmid \Sigma _{d:D}p\shortmid p^{\mathbin{\mbox{\aap
,}}}t\shortmid p\ll p, 
\]%
and parallel mCRL terms $q$ using the following grammar:%
\[
q::=p\shortmid q\parallel q\shortmid q\mathbin{\mathrel{\llfloor}}q\shortmid
q\mid q\shortmid \rho _{R}(q)\shortmid \partial _{B}(q)\shortmid \tau
_{I}(q)\shortmid \Gamma _{C}(q)\shortmid \nabla _{V}(q). 
\]

\paragraph{Remark 1}

Note that there is an unfortunate overload of the $\boldsymbol{\mid }$%
-operator in both multi actions and process expressions. This has
consequences for the implementation, since it there is no clean distinction
between parallel and non-parallel operators.

\paragraph{Remark 2}

The mCRL2 language also has a construct $P(d_{i_{1}}=e_{i_{1}},\ldots
,d_{i_{k}}=e_{i_{k}})$, but this is just a shorthand notation. Therefore we
will ignore it in this text.

\subsubsection{Alphabet operations}

Let $A,A_{1}$ and $A_{2}$ be sets of multi action names. Then we define%
\[
\begin{array}{lll}
A^{\subseteq } & = & \{\alpha \mid \exists \beta .\alpha \beta \in A\} \\ 
A_{1}\mid A_{2} & = & \{\alpha \beta \mid \alpha \in A_{1}\text{ and }\beta
\in A_{2}\} \\ 
A_{1}\leftarrowtail A_{2} & = & \{\alpha \mid \exists \beta .\alpha \beta
\in A_{1}\text{ and }\beta \in A_{2}\}%
\end{array}%
\]%
Note that $\beta $ can take the value $\tau $ in the definition of $%
A_{1}\leftarrowtail A_{2}$, which implies $A_{1}\subset A_{1}\leftarrowtail
A_{2}$. The set $A^{\subseteq }$ has an exponential size, so whenever
possible it should not be computed explicitly.

Let $C$ be a communication set, then we define%
\[
\begin{array}{lll}
C(A) & = & \cup _{\alpha \in A}\text{\textsc{Comm}(}C\text{, }\alpha \text{)}
\\ 
C^{-1}(A) & = & \cup _{\alpha \in A}\text{\textsc{CommInverse}(}C\text{, }%
\alpha \text{)} \\ 
filter_{\nabla }(C,A) & = & \{\gamma \rightarrow c\in C\mid \exists _{\alpha
\in A}.\gamma \subset \alpha \}%
\end{array}%
\]%
where \textsc{Comm} and \textsc{CommInverse} are defined using pseudo code
as follows:%
\[
\begin{array}{l}
\text{\textsc{Comm}(}C\text{, }\alpha \text{)} \\ 
R:=\{\alpha \} \\ 
\text{\textbf{for }}\gamma \rightarrow c\in C\text{ \textbf{do}} \\ 
\qquad \text{\textbf{if }}\exists \beta .\alpha =\beta \gamma \text{ \textbf{%
then} }R:=R\cup \text{\textsc{Comm}(}C\text{, }\beta c\text{)} \\ 
\text{\textbf{return }}R%
\end{array}%
\]%
\[
\begin{array}{l}
\text{\textsc{CommInverse}(}C\text{, }\alpha \text{)} \\ 
R:=\{\alpha \} \\ 
\text{\textbf{for }}\gamma \rightarrow c\in C\text{ \textbf{do}} \\ 
\qquad \text{\textbf{if }}\exists \beta .\alpha =\beta c\text{ \textbf{then} 
}R:=R\cup \text{\textsc{CommInverse}(}C\text{, }\beta \gamma \text{)} \\ 
\text{\textbf{return }}R%
\end{array}%
\]

Let $R$ be a rename set, then we define%
\[
\begin{array}{lll}
R(\alpha ) & = & \{R(\alpha _{i})\mid \alpha _{i}\in \alpha \} \\ 
R^{-1}(\alpha ) & = & \{\beta \mid R(\beta )=\alpha \} \\ 
R(A) & = & \{R(\alpha )\mid \alpha \in A\} \\ 
R^{-1}(A) & = & \{R^{-1}(\alpha )\mid \alpha \in A\}%
\end{array}%
\]%
Let $I$ be a hide set, then we define%
\[
\begin{array}{ccc}
\tau _{I}(A) & = & \{\beta \mid \exists _{\alpha \in A,\gamma \in I^{\ast
}}.\alpha =\beta \gamma \wedge \beta \cap I=\emptyset \} \\ 
\tau _{I}^{-1}(A) & = & A\mid I^{\ast }%
\end{array}%
\]%
Let $B$ be a block set, then we define%
\[
\partial _{B}(A)=\{\alpha \in A\mid \alpha \cap B=\emptyset \} 
\]

\subsubsection{The mapping $\protect\alpha $}

We define the mapping $\alpha $ as follows. The value $\alpha (p,\emptyset )$
is an over approximation of the alphabet of process expression $p$.%
\[
\begin{array}{lll}
\alpha (\overline{a},W) & = & \{a\} \\ 
\alpha (P,W) & = & 
\begin{array}{l}
\left\{ 
\begin{array}{ll}
\emptyset  & \text{if }P\in W \\ 
\alpha (p,W\cup \{P\}) & \text{if }P\notin W,%
\end{array}%
\right.  \\ 
\text{ where }P=p\text{ is the equation of }P%
\end{array}
\\ 
\alpha (\delta ,W) & = & \emptyset  \\ 
\alpha (\tau ,W) & = & \{\tau \} \\ 
\alpha (p+q,W) & = & \alpha (p,W)\cup \alpha (q,W) \\ 
\alpha (p\cdot q,W) & = & \alpha (p,W)\cup \alpha (q,W) \\ 
\alpha (c\rightarrow p,W) & = & \alpha (p,W) \\ 
\alpha (c\rightarrow p\diamond q,W) & = & \alpha (p,W)\cup \alpha (q,W) \\ 
\alpha (\Sigma _{d:D}p,W) & = & \alpha (p,W) \\ 
\alpha (p^{\mathbin{\mbox{\aap
,}}}t,W) & = & \alpha (p,W) \\ 
\alpha (p\ll q,W) & = & \alpha (p,W)\cup \alpha (q,W) \\ 
\alpha (p\parallel q,W) & = & \alpha (p,W)\cup \alpha (q,W)\cup \alpha
(p,W)\mid \alpha (q,W) \\ 
\alpha (p\mathbin{\mathrel{\llfloor}}q,W) & = & \alpha (p,W)\cup \alpha
(q,W)\cup \alpha (p,W)\mid \alpha (q,W) \\ 
\alpha (p\mid q,W) & = & \alpha (p,W)\mid \alpha (q,W) \\ 
\alpha (\rho _{R}(p),W) & = & R(\alpha (p,W)) \\ 
\alpha (\partial _{B}(p),W) & = & \partial _{B}(\alpha (p,W)) \\ 
\alpha (\tau _{I}(p),W) & = & \tau _{I}(\alpha (p,W)) \\ 
\alpha (\Gamma _{C}(p),W) & = & C(\alpha (p,W)) \\ 
\alpha (\nabla _{V}(p),W) & = & \alpha (p,W)\cap (V\cup \{\tau \})%
\end{array}%
\]%
When computing $A\cap \alpha (p,W)$ for some multi action name set $A$, we
can use the following optimization. This is done to keep intermediate
expressions small.%
\[
\begin{array}{lll}
A\cap \alpha (P,W) & = & 
\begin{array}{l}
\left\{ 
\begin{array}{ll}
\emptyset  & \text{if }P\in W \\ 
A^{\subseteq }\cap \alpha (p,W\cup \{P\}) & \text{if }P\notin W,%
\end{array}%
\right.  \\ 
\text{ where }P=p\text{ is the equation of }P%
\end{array}
\\ 
A\cap \alpha (p+q,W) & = & \left( A^{\subseteq }\cap \alpha (p,W)\right)
\cup \left( A^{\subseteq }\cap \alpha (q,W)\right)  \\ 
A\cap \alpha (p\cdot q,W) & = & \left( A^{\subseteq }\cap \alpha
(p,W)\right) \cup \left( A^{\subseteq }\cap \alpha (q,W)\right)  \\ 
A\cap \alpha (c\rightarrow p,W) & = & A^{\subseteq }\cap \alpha (p,W) \\ 
A\cap \alpha (c\rightarrow p\diamond q,W) & = & \left( A^{\subseteq }\cap
\alpha (p,W)\right) \cup \left( A^{\subseteq }\cap \alpha (q,W)\right)  \\ 
A\cap \alpha (\Sigma _{d:D}p,W) & = & A^{\subseteq }\cap \alpha (p,W) \\ 
A\cap \alpha (p^{\mathbin{\mbox{\aap
,}}}t,W) & = & A^{\subseteq }\cap \alpha (p,W) \\ 
A\cap \alpha (p\ll q,W) & = & \left( A^{\subseteq }\cap \alpha (p,W)\right)
\cup \left( A^{\subseteq }\cap \alpha (q,W)\right)  \\ 
A\cap \alpha (p\parallel q,W) & = & \left( A^{\subseteq }\cap \alpha
(p,W)\right) \cup \left( A^{\subseteq }\cap \alpha (q,W)\right) \cup \left(
A^{\subseteq }\cap \alpha (p,W)\right) \mid \left( A^{\subseteq }\cap \alpha
(q,W)\right)  \\ 
\alpha (p\mathbin{\mathrel{\llfloor}}q,W) & = & \left( A^{\subseteq }\cap
\alpha (p,W)\right) \cup \left( A^{\subseteq }\cap \alpha (q,W)\right) \cup
\left( A^{\subseteq }\cap \alpha (p,W)\right) \mid \left( A^{\subseteq }\cap
\alpha (q,W)\right)  \\ 
A\cap \alpha (p\mid q,W) & = & \left( A^{\subseteq }\cap \alpha (p,W)\right)
\mid \left( A^{\subseteq }\cap \alpha (q,W)\right) 
\end{array}%
\]

\paragraph{Example 1}

If $C=\{a\mid b\rightarrow c\}$, then $\alpha (\Gamma _{C}(a(1)\mid
b(2)))=\{a,b,c,a\mid b\}$. Note that the action $c$ does not occur in the
transition system of this process expression.

\paragraph{Example 2}

In the computation of $\left\{ a_{1},a_{2},\ldots ,a_{20}\right\} \cap
\alpha \left( a_{1}\parallel a_{2}\parallel \ldots \parallel a_{20}\right) $
the above mentioned optimization is really needed.

\subsubsection{The mapping $push_{\protect\nabla }$}

We define a mapping $push_{\nabla }$ such that $push_{\nabla }(A,p)$ is
bisimulation equivalent to $\nabla _{A}(p)$. This mapping is used as a
preprocession step to linearization, and it has the goal to push allow
expressions deeply inside process expressions. It is important to know that
an allow set $A$ in the expression $\nabla _{A}(p)$ implicitly contains the
empty multi action $\tau $.%
\[
\begin{tabular}{lll}
$push_{\nabla }(A,p)$ & $=$ & $\nabla _{A}(p)$ if $p$ is a pCRL expression
\\ 
$push_{\nabla }(A,p\parallel q)$ & $=$ & $\nabla _{A}(p^{\prime }\parallel
q^{\prime })\text{ where}\left\{ \text{ }%
\begin{array}{lll}
p^{\prime } & = & push_{\nabla }(A^{\subseteq },p) \\ 
q^{\prime } & = & push_{\nabla }(A\leftarrowtail \alpha (p^{\prime }),q)%
\end{array}%
\right. $ \\ 
$push_{\nabla }(A,p\mathbin{\mathrel{\llfloor}}q)$ & $=$ & $\nabla
_{A}(p^{\prime }\mathbin{\mathrel{\llfloor}}q^{\prime })\text{ where}\left\{ 
\text{ }%
\begin{array}{lll}
p^{\prime } & = & push_{\nabla }(A^{\subseteq },p) \\ 
q^{\prime } & = & push_{\nabla }(A\leftarrowtail \alpha (p^{\prime }),q)%
\end{array}%
\right. $ \\ 
$push_{\nabla }(A,p\mid q)$ & $=$ & $\nabla _{A}(p^{\prime }\mid q^{\prime
}) $ where$\left\{ \text{ }%
\begin{array}{lll}
p^{\prime } & = & push_{\nabla }(A^{\subseteq },p) \\ 
q^{\prime } & = & push_{\nabla }(A\leftarrowtail \alpha (p^{\prime }),q)%
\end{array}%
\right. $ \\ 
$push_{\nabla }(A,\rho _{R}(p))$ & $=$ & $\rho _{R}(p^{\prime })$ where $%
p^{\prime }=$ $push_{\nabla }(R^{-1}(\partial _{B}(A)),p)$ where $B=\{a\mid
\exists b.a\rightarrow b\in R\}$ \\ 
$push_{\nabla }(A,\partial _{B}(p))$ & $=$ & $push_{\nabla }(\partial
_{B}(A),p)$ \\ 
$push_{\nabla }(A,\tau _{I}(p))$ & $=$ & $\tau _{I}(p^{\prime })$ where $%
p^{\prime }=push_{\nabla }(\tau _{I}^{-1}(A),p)$ \\ 
$push_{\nabla }(A,\Gamma _{C}(p))$ & $=$ & $\nabla _{A}(\Gamma
_{C}(p^{\prime }))$ where $p^{\prime }=push_{\nabla }(C^{-1}(A),p)$ \\ 
$push_{\nabla }(A,\nabla _{V}(p))$ & $=$ & $push_{\nabla }(A\cap V,p)$%
\end{tabular}%
\]

\paragraph{Example 3}

The presence of $R^{-1}(\partial _{B}(A))$ instead of just $R^{-1}(A)$ in
the right hand side of the rename operator is explained by the example $%
push_{\nabla }(\{b\},\rho _{\{b\rightarrow c\}}b)$. We see that $\rho
_{\{b\rightarrow c\}}push_{\nabla }(R^{-1}(A),p)=\rho _{\{b\rightarrow
c\}}push_{\nabla }(\{b\},b)=\rho _{\{b\rightarrow c\}}b=c$, which is clearly
the wrong answer.

\paragraph{Optimizations}

During the computation of $push_{\bigtriangledown }$ the following
optimizations are applied in the right hand side of each equation:%
\[
\begin{array}{lll}
\nabla _{A}(p) & = & \left\{ 
\begin{array}{ll}
p & \text{if }(A\cup \{\tau \})\cap \alpha (p)=\alpha (p) \\ 
\nabla _{A\cap \alpha (p)}(p) & \text{otherwise}%
\end{array}%
\right. \\ 
\nabla _{\emptyset }(p) & = & \left\{ 
\begin{array}{ll}
\tau & \text{if }p=\tau \\ 
\delta & \text{otherwise}%
\end{array}%
\right. \\ 
\Gamma _{C}(p) & = & \Gamma _{filter_{\nabla }(C,\alpha (p))}(p) \\ 
\delta \mid \delta & = & \delta \\ 
\delta \parallel \delta & = & \delta%
\end{array}%
\]

\subsubsection{Allow sets}

There are two rules in the definition of $push_{\nabla }$ where the allow
set can/should not be computed explicitly. The computation of $push_{\nabla
}(A,p\parallel q)$ involves computation of $push_{\nabla }(p,A^{\subseteq }%
\dot{)}$. We want to avoid the computation of $A^{\subseteq }$, since it can
become very large. The computation of $push_{\nabla }(A,\tau _{I}(p))$
involves computation of $push_{\nabla }(p,\tau _{I}^{-1}(A)\dot{)}$. The set 
$\tau _{I}^{-1}(A)=A\mid I^{\ast }$ is infinite.

In the implementation we use allow sets of the form $\left( A\mid I^{\ast
}\right) ^{\subseteq },$ where $I$ may be empty, and where $^{\subseteq }$
is optional. Such an allow set is stored as two sets $A$ and $I$, together
with an attribute that tells if $^{\subseteq }$ is appicable. We need to
show that allow sets are closed under the operations in $push_{\nabla }$.%
\[
\begin{array}{lll}
\partial _{B}(\left( A\mid I^{\ast }\right) ^{\subseteq }) & ?= & \left(
\partial _{B}(A)\mid \partial _{B}(I)^{\ast }\right) ^{\subseteq } \\ 
\tau _{I_{1}}^{-1}\left( \left( A\mid I^{\ast }\right) ^{\subseteq }\right)
& ?= & \left( A\mid \left( I\cup I_{1}\right) ^{\ast }\right) ^{\subseteq }
\\ 
\left( A\mid I^{\ast }\right) ^{\subseteq }\cap V & = & \{\beta \in V\mid
\exists _{\alpha \in A}.\tau _{I}(\beta )\sqsubseteq \alpha \} \\ 
R^{-1}\left( \left( A\mid I^{\ast }\right) ^{\subseteq }\right) & = & ? \\ 
C^{-1}\left( \left( A\mid I^{\ast }\right) ^{\subseteq }\right) & = & ? \\ 
\left( A\mid I^{\ast }\right) ^{\subseteq }\leftarrowtail A_{1} & = & ? \\ 
\left( \left( A\mid I^{\ast }\right) ^{\subseteq }\right) ^{\subseteq } & =
& \left( A\mid I^{\ast }\right) ^{\subseteq }%
\end{array}%
\]

\end{document}
