% at
% left merge
% \RequirePackage{stmaryrd}


\documentclass{article}
%%%%%%%%%%%%%%%%%%%%%%%%%%%%%%%%%%%%%%%%%%%%%%%%%%%%%%%%%%%%%%%%%%%%%%%%%%%%%%%%%%%%%%%%%%%%%%%%%%%%%%%%%%%%%%%%%%%%%%%%%%%%%%%%%%%%%%%%%%%%%%%%%%%%%%%%%%%%%%%%%%%%%%%%%%%%%%%%%%%%%%%%%%%%%%%%%%%%%%%%%%%%%%%%%%%%%%%%%%%%%%%%%%%%%%%%%%%%%%%%%%%%%%%%%%%%
\usepackage{amssymb}
\usepackage{geometry}
\usepackage{stmaryrd}

%TCIDATA{OutputFilter=LATEX.DLL}
%TCIDATA{Version=5.50.0.2890}
%TCIDATA{<META NAME="SaveForMode" CONTENT="1">}
%TCIDATA{BibliographyScheme=Manual}
%TCIDATA{Created=Friday, June 15, 2012 17:58:29}
%TCIDATA{LastRevised=Tuesday, September 25, 2012 10:14:15}
%TCIDATA{<META NAME="GraphicsSave" CONTENT="32">}
%TCIDATA{<META NAME="DocumentShell" CONTENT="Standard LaTeX\Blank - Standard LaTeX Article">}
%TCIDATA{CSTFile=40 LaTeX article.cst}

\newtheorem{theorem}{Theorem}
\newtheorem{acknowledgement}[theorem]{Acknowledgement}
\newtheorem{algorithm}[theorem]{Algorithm}
\newtheorem{axiom}[theorem]{Axiom}
\newtheorem{case}[theorem]{Case}
\newtheorem{claim}[theorem]{Claim}
\newtheorem{conclusion}[theorem]{Conclusion}
\newtheorem{condition}[theorem]{Condition}
\newtheorem{conjecture}[theorem]{Conjecture}
\newtheorem{corollary}[theorem]{Corollary}
\newtheorem{criterion}[theorem]{Criterion}
\newtheorem{definition}[theorem]{Definition}
\newtheorem{example}[theorem]{Example}
\newtheorem{exercise}[theorem]{Exercise}
\newtheorem{lemma}[theorem]{Lemma}
\newtheorem{notation}[theorem]{Notation}
\newtheorem{problem}[theorem]{Problem}
\newtheorem{proposition}[theorem]{Proposition}
\newtheorem{remark}[theorem]{Remark}
\newtheorem{solution}[theorem]{Solution}
\newtheorem{summary}[theorem]{Summary}
\newenvironment{proof}[1][Proof]{\noindent\textbf{#1.} }{\ \rule{0.5em}{0.5em}}
\geometry{left=1in,right=1in,top=1in,bottom=1in}
\input{include/tcilatex}
\font \aap cmmi10
\newcommand{\at}{\mathbin{\mbox{\aap ,}}}
\newcommand{\leftm}{\mathbin{\mathrel{\llfloor}}}

\begin{document}


\section{Process Library Implementation Notes}

\subsection{Guarded process expressions}

We define the predicate $is\_guarded$ for process expressions as follows:%
\[
\begin{array}{lll}
is\_guarded(a(e)) & = & true \\ 
is\_guarded(\delta ) & = & true \\ 
is\_guarded(\tau ) & = & true \\ 
is\_guarded(P(e)) & = & false \\ 
is\_guarded(P(d:=e)) & = & false \\ 
is\_guarded(p+q) & = & is\_guarded(p)\wedge is\_guarded(q) \\ 
is\_guarded(p\cdot q) & = & is\_guarded(p) \\ 
is\_guarded(c\rightarrow p) & = & is\_guarded(p) \\ 
is\_guarded(c\rightarrow p\diamond q) & = & is\_guarded(p)\wedge
is\_guarded(q) \\ 
is\_guarded(\Sigma _{d:D}\ p) & = & is\_guarded(p) \\ 
is\_guarded(p^{\mathbin{\mbox{\aap ,}}}t) & = & is\_guarded(p) \\ 
is\_guarded(p\ll q) & = & is\_guarded(p) \\ 
is\_guarded(p\parallel q) & = & is\_guarded(p)\wedge is\_guarded(q) \\ 
is\_guarded(p\mathbin{\mathrel{\llfloor}}q) & = & is\_guarded(p)\wedge
is\_guarded(q) \\ 
is\_guarded(p\mid q) & = & is\_guarded(p)\wedge is\_guarded(q) \\ 
is\_guarded(\rho _{R}(p)) & = & is\_guarded(p) \\ 
is\_guarded(\partial _{B}(p)) & = & is\_guarded(p) \\ 
is\_guarded(\tau _{I}(p)) & = & is\_guarded(p) \\ 
is\_guarded(\Gamma _{C}(p)) & = & is\_guarded(p) \\ 
is\_guarded(\nabla _{V}(p)) & = & is\_guarded(p)%
\end{array}%
\]%
\newpage 

\subsection{Alphabet reduction}

\subsubsection{Definitions}

We define pCRL terms $p$ using the following grammar:%
\[
p::=a\ \mathbf{|}\ P(e)\mid P(e:=d)\mid \delta \mid \tau \mid p+p\mid p\cdot
p\mid c\rightarrow p\mid c\rightarrow p\diamond p\mid \Sigma _{d:D}p\mid p^{%
\mathbin{\mbox{\aap ,}}}t\mid p\ll p 
\]%
and extended pCRL terms $q$ using the following grammar:%
\[
q::=p\mid q\parallel q\mid q\mathbin{\mathrel{\llfloor}}q\mid q\shortmid
q\mid \rho _{R}(q)\mid \partial _{B}(q)\mid \tau _{I}(q)\mid \Gamma
_{C}(q)\mid \nabla _{V}(q) 
\]

Let $A,A_{1}$ and $A_{2}$ be sets of multi action names. Then we define%
\begin{eqnarray*}
A^{\subseteq } &=&\{\alpha \mid \exists \beta .\alpha \beta \in A\} \\
A_{1}\times A_{2} &=&\{\alpha \beta \mid \alpha \in A_{1}\text{ and }\beta
\in A_{2}\} \\
A_{1}\leftarrowtail A_{2} &=&\{\alpha \mid \exists \beta .\alpha \beta \in
A_{1}\text{ and }\beta \in A_{2}\}
\end{eqnarray*}

Let $C$ be a communication set, then we define%
\begin{eqnarray*}
C(A) &=&A\cup \{a\beta \mid \exists a.\alpha \rightarrow a\in C\text{ and }%
\alpha \beta \in A\} \\
C^{-1}(A) &=&\{\alpha \beta \mid \exists a.\alpha \rightarrow a\in C\text{
and }a\beta \in A\} \\
\overline{C}(A) &=&\{\alpha \in A\mid \exists \beta .\alpha \beta
\rightarrow a\in C\} \\
C\setminus A &=&\{\alpha \rightarrow a\in C\mid a\notin A\}
\end{eqnarray*}%
Let $R$ be a rename set, then we define%
\begin{eqnarray*}
R(\alpha ) &=&\{R(a)\mid a\in \alpha \} \\
R^{-1}(a) &=&\{\beta \mid R(\beta )=\alpha \} \\
R(A) &=&\{R(\alpha )\mid \alpha \in A\} \\
R^{-1}(A) &=&\cup _{\alpha \in A}R^{-1}(\alpha )
\end{eqnarray*}%
Let $B$ be a block set, then we define%
\[
\partial _{B}(A)=\{\alpha \in A\mid \text{no action in }B\text{ occurs in }%
\alpha \} 
\]%
Let $I$ be a hide set, then we define%
\[
\tau _{I}(A)=\{\alpha \mid \exists \beta \in I^{\ast }.\alpha \beta \in A%
\text{ and no action in }I\text{ occurs in }\alpha \}\text{,} 
\]%
where $I^{\ast }$ is the transitive reflexive closure of $I$.\newpage

\subsubsection{The mapping $push_{\odot }$}

We define the mapping $push_{\odot }$, with $\odot \in \{\bigtriangledown
,\subseteq ,\partial \}$ as follows, where $\left\langle p^{\prime
},A_{p}^{\prime }\right\rangle =$ $push_{\odot }(p,A)$ and $\left\langle
q^{\prime },A_{q}^{\prime }\right\rangle =push_{\odot }(q,A)$, unless told
otherwise. Note that this specification is only valid for pCRL expressions.%
\[
\begin{tabular}{ll}
$x$ & \multicolumn{1}{|l}{$push_{\odot }(x,A)$} \\ \cline{1-2}
$\delta $ & \multicolumn{1}{|l}{$\left\langle \delta ,\emptyset
\right\rangle $} \\ 
$P(e_{1},\cdots ,e_{n})$ & \multicolumn{1}{|l}{$%
\begin{array}{l}
\left\langle Q(e_{1},\cdots ,e_{n}),A_{p}^{\prime }\right\rangle \\ 
\text{where }\left\{ \text{ }%
\begin{array}{lll}
p & = & \text{the body of process }P \\ 
Q(e_{1},\cdots ,e_{n})=p^{\prime } & = & \text{a possibly new process
equation}%
\end{array}%
\right.%
\end{array}%
$} \\ 
$P(d_{i_{1}}=e_{i_{1}},\cdots ,d_{i_{k}}=e_{i_{k}})$ & \multicolumn{1}{|l}{$%
\begin{array}{l}
\left\langle Q(d_{i_{1}}=e_{i_{1}},\cdots
,d_{i_{k}}=e_{i_{k}}),A_{p}^{\prime }\right\rangle \\ 
\text{where }\left\{ \text{ }%
\begin{array}{lll}
p & = & \text{the body of process }P \\ 
Q(e_{1},\cdots ,e_{n})=p^{\prime } & = & \text{a possibly new process
equation}%
\end{array}%
\right.%
\end{array}%
$} \\ 
$p+q$ & \multicolumn{1}{|l}{$\left\langle f_{\odot }(x,A),A_{p}^{\prime
}\cup A_{q}^{\prime }\right\rangle $} \\ 
$p\cdot q$ & \multicolumn{1}{|l}{$\left\langle f_{\odot }(x,A),A_{p}^{\prime
}\cup A_{q}^{\prime }\right\rangle $} \\ 
$c\rightarrow p$ & \multicolumn{1}{|l}{$\left\langle f_{\odot
}(x,A),A_{p}^{\prime }\right\rangle $} \\ 
$c\rightarrow p\diamond q$ & \multicolumn{1}{|l}{$\left\langle f_{\odot
}(x,A),A_{p}^{\prime }\cup A_{q}^{\prime }\right\rangle $} \\ 
$\Sigma _{d:D}p$ & \multicolumn{1}{|l}{$\left\langle f_{\odot
}(x,A),A_{p}^{\prime }\right\rangle $} \\ 
$p^{\mathbin{\mbox{\aap ,}}}t$ & \multicolumn{1}{|l}{$\left\langle f_{\odot
}(x,A),A_{p}^{\prime }\right\rangle $} \\ 
$p\ll q$ & \multicolumn{1}{|l}{$push_{\odot }(p,A)$} \\ 
$\rho _{R}(p)$ & \multicolumn{1}{|l}{$push_{\odot }(p,A^{\prime })\text{
where }A^{\prime }=\{R(\alpha )\mid \alpha \in R^{-1}(A)\},$}%
\end{tabular}%
\]%
where%
\begin{eqnarray*}
f_{\bigtriangledown }(p,A) &=&\nabla _{A}(x) \\
f_{\subseteq }(p,A) &=&\nabla _{A\setminus \{\tau \}}(x) \\
f_{\partial }(p,A) &=&\partial _{A}(x)
\end{eqnarray*}

\newpage

\subsubsection{The mapping $push_{\bigtriangledown }$}

We define the mapping $push_{\bigtriangledown }$ as follows, where $%
\left\langle p^{\prime },A_{p}^{\prime }\right\rangle =$ $%
push_{\bigtriangledown }(p,A)$ and $\left\langle q^{\prime },A_{q}^{\prime
}\right\rangle =push_{\bigtriangledown }(q,A)$, unless told otherwise.%
\[
\begin{tabular}{l|l}
$x$ & $push_{\bigtriangledown }(x,A)$ \\ \hline
$a(e_{1},\cdots ,e_{n})$ & $\left\langle \nabla _{A^{\prime }}(x),A^{\prime
}\right\rangle \text{ where }A^{\prime }=A\cap \{a\}$ \\ 
$\tau $ & $\left\langle \tau ,A^{\prime }\right\rangle \text{ where }%
A^{\prime }=A\cap \{\tau \}$ \\ 
$p\parallel q$ & $\left\langle \nabla _{A^{\prime }\setminus \{\tau
\}}(p^{\prime }\parallel q^{\prime }),A^{\prime }\right\rangle \text{ where}%
\left\{ \text{ }%
\begin{array}{lll}
\left\langle p^{\prime },A_{p}^{\prime }\right\rangle & = & push_{\subseteq
}(p,A) \\ 
\left\langle q^{\prime },A_{q}^{\prime }\right\rangle & = & 
push_{\bigtriangledown }(q,A\cup A\leftarrowtail A_{p}^{\prime }) \\ 
A^{\prime } & = & A\cap (A_{p}^{\prime }\cup A_{q}^{\prime }\cup
A_{p}^{\prime }\times A_{q}^{\prime })%
\end{array}%
\right. $ \\ 
$p\mathbin{\mathrel{\llfloor}}q$ & $\left\langle \nabla _{A^{\prime
}\setminus \{\tau \}}(p^{\prime }\mathbin{\mathrel{\llfloor}}q^{\prime
}),A^{\prime }\right\rangle \text{ where}\left\{ \text{ }%
\begin{array}{lll}
\left\langle p^{\prime },A_{p}^{\prime }\right\rangle & = & push_{\subseteq
}(p,A) \\ 
\left\langle q^{\prime },A_{q}^{\prime }\right\rangle & = & 
push_{\bigtriangledown }(q,A\cup A\leftarrowtail A_{p}^{\prime }) \\ 
A^{\prime } & = & A\cap (A_{p}^{\prime }\cup A_{q}^{\prime }\cup
A_{p}^{\prime }\times A_{q}^{\prime })%
\end{array}%
\right. $ \\ 
$p\mid q$ & $\left\langle \nabla _{A^{\prime }\setminus \{\tau \}}(p^{\prime
}\mid q^{\prime }),A^{\prime }\right\rangle \text{ where}\left\{ \text{ }%
\begin{array}{lll}
\left\langle p^{\prime },A_{p}^{\prime }\right\rangle & = & push_{\subseteq
}(p,A) \\ 
\left\langle q^{\prime },A_{q}^{\prime }\right\rangle & = & 
push_{\bigtriangledown }(q,A\cup A\leftarrowtail A_{p}^{\prime }) \\ 
A^{\prime } & = & A\cap (A_{p}^{\prime }\times A_{q}^{\prime })%
\end{array}%
\right. $ \\ 
$\rho _{R}(p)$ & $push_{\bigtriangledown }(p,A^{\prime })\text{ where }%
A^{\prime }=R(R^{-1}(A))$ \\ 
$\partial _{B}(p)$ & $push_{\bigtriangledown }(p,\partial _{B}(A))$ \\ 
$\tau _{I}(p)$ & $\left\langle \nabla _{A\setminus \{\tau \}}(\tau
_{I}(p^{\prime })),A^{\prime }\right\rangle \text{ where}\left\{ \text{ }%
\begin{array}{lll}
\left\langle p^{\prime },A_{p}^{\prime }\right\rangle & = & 
push_{\bigtriangledown }(p,Act) \\ 
A^{\prime } & = & A\cap \tau _{I}(A_{p}^{\prime })%
\end{array}%
\right. $ \\ 
$\Gamma _{C}(p)$ & $\left\langle \nabla _{A\setminus \{\tau \}}(\Gamma
_{C}(p^{\prime })),A^{\prime }\right\rangle \text{ where}\left\{ \text{ }%
\begin{array}{lll}
\left\langle p^{\prime },A_{p}^{\prime }\right\rangle & = & 
push_{\bigtriangledown }(p,A\cup C^{-1}(A)) \\ 
A^{\prime } & = & A\cap C(A_{p}^{\prime })%
\end{array}%
\right. $ \\ 
$\nabla _{V}(p)$ & $push_{\bigtriangledown }(p,A\cap V)$%
\end{tabular}%
\]%
\[
\begin{tabular}{l|l}
$x$ & $push_{\bigtriangledown }(x,Act)$ \\ \hline
$p\parallel q$ & $\left\langle \nabla _{A^{\prime }\setminus \{\tau
\}}(p^{\prime }\mathbin{\mathrel{\llfloor}}q^{\prime }),A^{\prime
}\right\rangle \text{ where }A^{\prime }=A_{p}^{\prime }\cup A_{q}^{\prime
}\cup A_{p}^{\prime }\times A_{q}^{\prime }$ \\ 
$p\mathbin{\mathrel{\llfloor}}q$ & $\left\langle \nabla _{A^{\prime
}\setminus \{\tau \}}(p^{\prime }\mathbin{\mathrel{\llfloor}}q^{\prime
}),A^{\prime }\right\rangle \text{ where }A^{\prime }=A_{p}^{\prime }\cup
A_{q}^{\prime }\cup A_{p}^{\prime }\times A_{q}^{\prime }$ \\ 
$p\mid q$ & $\left\langle \nabla _{A^{\prime }\setminus \{\tau \}}(p^{\prime
}\parallel q^{\prime }),A^{\prime }\right\rangle \text{ where }A^{\prime
}=A_{p}^{\prime }\times A_{q}^{\prime }$ \\ 
$\partial _{B}(p)$ & $push_{\partial }(p,B)$%
\end{tabular}%
\]

\newpage

\subsubsection{The mapping $push_{\subseteq }$}

We define the mapping $push_{\subseteq }$ as follows, where $\left\langle
p^{\prime },A_{p}^{\prime }\right\rangle =$ $push_{\subseteq }(p,A)$ and $%
\left\langle q^{\prime },A_{q}^{\prime }\right\rangle =push_{\subseteq
}(q,A) $, unless told otherwise.%
\[
\begin{tabular}{l|l}
$x$ & $push_{\subseteq }(x,A)$ \\ \hline
$a(e_{1},\cdots ,e_{n})$ & $\left\langle \nabla _{A^{\prime
}}(a(e_{1},\cdots ,e_{n})),A^{\prime }\right\rangle \text{ where }A^{\prime
}=A^{\subseteq }\cap \{a\}$ \\ 
$\tau $ & $\left\langle \tau ,A^{\prime }\right\rangle \text{ where }%
A^{\prime }=A\cap \{\tau \}$ \\ 
$p\parallel q$ & $\left\langle \nabla _{A^{\prime }\setminus \{\tau
\}}(p^{\prime }\parallel q^{\prime }),A^{\prime }\right\rangle \text{ where }%
A^{\prime }=A^{\subseteq }\cap (A_{p}^{\prime }\cup A_{q}^{\prime }\cup
A_{p}^{\prime }\times A_{q}^{\prime })$ \\ 
$p\mathbin{\mathrel{\llfloor}}q$ & $\left\langle \nabla _{A^{\prime
}\setminus \{\tau \}}(p^{\prime }\mathbin{\mathrel{\llfloor}}q^{\prime
}),A^{\prime }\right\rangle \text{ where }A^{\prime }=A^{\subseteq }\cap
(A_{p}^{\prime }\cup A_{q}^{\prime }\cup A_{p}^{\prime }\times A_{q}^{\prime
})$ \\ 
$p\mid q$ & $\left\langle \nabla _{A^{\prime }\setminus \{\tau \}}(p^{\prime
}\mid q^{\prime }),A^{\prime }\right\rangle \text{ where }A^{\prime
}=A^{\subseteq }\cap (A_{p}^{\prime }\times A_{q}^{\prime })$ \\ 
$\rho _{R}(p)$ & $push_{\subseteq }(p,A^{\prime })\text{ where }A^{\prime
}=\{R(\alpha )\mid \alpha \in R^{-1}(A)\}$ \\ 
$\partial _{B}(p)$ & $push_{\subseteq }(p,A^{\prime })\text{ where }%
A^{\prime }=\partial _{B}(A)$ \\ 
$\tau _{I}(p)$ & $\left\langle \nabla _{A^{\prime }\setminus \{\tau \}}(\tau
_{I}(p^{\prime })),A^{\prime }\right\rangle \text{ where }A^{\prime }=A\cap
\tau _{I}(A_{p}^{\prime })\text{ }$ \\ 
$\Gamma _{C}(p)$ & $\left\langle \nabla _{A^{\prime }}(\Gamma _{C}(p^{\prime
})),A^{\prime \prime }\right\rangle \text{ where}\left\{ \text{ }%
\begin{array}{lll}
\left\langle p^{\prime },A_{p}^{\prime }\right\rangle & = & push_{\subseteq
}(p,A\cup C^{-1}(A)) \\ 
A^{\prime } & = & A\cap C(A_{p}^{\prime }) \\ 
A^{\prime \prime } & = & A^{\subseteq }\cap C(A_{p}^{\prime })%
\end{array}%
\right. $ \\ 
$\nabla _{V}(p)$ & $push_{\subseteq }(p,A^{\prime })\text{ where }A^{\prime
}=A^{\subseteq }\cap V$%
\end{tabular}%
\]

\subsubsection{The mapping $push_{\partial }$}

We define the mapping $push_{\partial }$ as follows, where $\left\langle
p^{\prime },A_{p}^{\prime }\right\rangle =$ $push_{\partial }(p,A)$ and $%
\left\langle q^{\prime },A_{q}^{\prime }\right\rangle =push_{\partial }(q,A)$%
, unless told otherwise.%
\[
\begin{tabular}{l|l}
$x$ & $push_{\partial }(x,A)$ \\ \hline
$a(e_{1},\cdots ,e_{n})$ & $\left\langle \partial _{A}(a(e_{1},\cdots
,e_{n})),A^{\prime }\right\rangle \text{ where }A^{\prime }=A\cap \{a\}$ \\ 
$\tau $ & $\left\langle \tau ,A^{\prime }\right\rangle \text{ where }%
A^{\prime }=\{\tau \}$ \\ 
$p\parallel q$ & $\left\langle p^{\prime }\parallel q^{\prime },A^{\prime
}\right\rangle \text{ where }A^{\prime }=A_{p}^{\prime }\cup A_{q}^{\prime
}\cup A_{p}^{\prime }\times A_{q}^{\prime }$ \\ 
$p\mathbin{\mathrel{\llfloor}}q$ & $\left\langle p^{\prime }%
\mathbin{\mathrel{\llfloor}}q^{\prime },A^{\prime }\right\rangle \text{
where }A^{\prime }=A_{p}^{\prime }\cup A_{q}^{\prime }\cup A_{p}^{\prime
}\times A_{q}^{\prime }$ \\ 
$p\mid q$ & $\left\langle p^{\prime }\mid q^{\prime },A^{\prime
}\right\rangle \text{ where }A^{\prime }=A_{p}^{\prime }\times A_{q}^{\prime
}$ \\ 
$\rho _{R}(p)$ & $\left\langle \rho _{R}(p^{\prime }),A^{\prime
}\right\rangle \text{ where }\left\{ \text{ }%
\begin{array}{lll}
\left\langle p^{\prime },A_{p}^{\prime }\right\rangle & = & push_{\partial
}(p,R^{-1}(A)) \\ 
A^{\prime } & = & R(A_{p}^{\prime })%
\end{array}%
\right. $ \\ 
$\partial _{B}(p)$ & $push_{\partial }(p,A^{\prime })\text{ where }A^{\prime
}=A\cup B$ \\ 
$\tau _{I}(p)$ & $\left\langle \tau _{I}(p^{\prime }),A^{\prime
}\right\rangle \text{ where }\left\{ \text{ }%
\begin{array}{lll}
\left\langle p^{\prime },A_{p}^{\prime }\right\rangle & = & push_{\partial
}(p,A\setminus I) \\ 
A^{\prime } & = & \tau _{I}(A)%
\end{array}%
\right. $ \\ 
$\Gamma _{C}(p)$ & $\left\langle \partial _{A^{\prime \prime }}(\Gamma
_{C^{\prime }}(p^{\prime })),A^{\prime }\right\rangle \text{ where}\left\{ 
\text{ }%
\begin{array}{lll}
A^{\prime \prime } & = & \overline{C}(A) \\ 
\left\langle p^{\prime },A_{p}^{\prime }\right\rangle & = & push_{\partial
}(p,A\setminus A^{\prime \prime }) \\ 
A^{\prime } & = & \partial _{A}(C(A_{p}^{\prime })) \\ 
C^{\prime } & = & C\setminus A%
\end{array}%
\right. $ \\ 
$\nabla _{V}(p)$ & $push_{\subseteq }(p,A^{\prime })\text{ where }A^{\prime
}=A^{\subseteq }\cap V$%
\end{tabular}%
\]

\end{document}
