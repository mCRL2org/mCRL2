
\documentclass{article}
\usepackage{amsfonts}

%%%%%%%%%%%%%%%%%%%%%%%%%%%%%%%%%%%%%%%%%%%%%%%%%%%%%%%%%%%%%%%%%%%%%%%%%%%%%%%%%%%%%%%%%%%%%%%%%%%%%%%%%%%%%%%%%%%%%%%%%%%%%%%%%%%%%%%%%%%%%%%%%%%%%%%%%%%%%%%%%%%%%%%%%%%%%%%%%%%%%%%%%%%%%%%%%%%%%%%%%%%%%%%%%%%%%%%%%%%%%%%%%%%
%TCIDATA{OutputFilter=LATEX.DLL}
%TCIDATA{Version=5.50.0.2890}
%TCIDATA{<META NAME="SaveForMode" CONTENT="1">}
%TCIDATA{BibliographyScheme=Manual}
%TCIDATA{Created=Monday, January 21, 2013 17:27:12}
%TCIDATA{LastRevised=Monday, January 28, 2013 13:24:20}
%TCIDATA{<META NAME="GraphicsSave" CONTENT="32">}
%TCIDATA{<META NAME="DocumentShell" CONTENT="Standard LaTeX\Blank - Standard LaTeX Article">}
%TCIDATA{CSTFile=40 LaTeX article.cst}

\newtheorem{theorem}{Theorem}
\newtheorem{acknowledgement}[theorem]{Acknowledgement}
\newtheorem{algorithm}[theorem]{Algorithm}
\newtheorem{axiom}[theorem]{Axiom}
\newtheorem{case}[theorem]{Case}
\newtheorem{claim}[theorem]{Claim}
\newtheorem{conclusion}[theorem]{Conclusion}
\newtheorem{condition}[theorem]{Condition}
\newtheorem{conjecture}[theorem]{Conjecture}
\newtheorem{corollary}[theorem]{Corollary}
\newtheorem{criterion}[theorem]{Criterion}
\newtheorem{definition}[theorem]{Definition}
\newtheorem{example}[theorem]{Example}
\newtheorem{exercise}[theorem]{Exercise}
\newtheorem{lemma}[theorem]{Lemma}
\newtheorem{notation}[theorem]{Notation}
\newtheorem{problem}[theorem]{Problem}
\newtheorem{proposition}[theorem]{Proposition}
\newtheorem{remark}[theorem]{Remark}
\newtheorem{solution}[theorem]{Solution}
\newtheorem{summary}[theorem]{Summary}
\newenvironment{proof}[1][Proof]{\noindent\textbf{#1.} }{\ \rule{0.5em}{0.5em}}
\input{include/tcilatex}

\begin{document}


\section{Capture avoiding substitutions}

This document specifies how capture avoiding substitutions are currently
implemented in mCRL2.

\subsection{Data expressions}

mCRL2 data expressions $x$ are characterized by the following grammar:%
\[
x::=v\mid f\mid x(x)\mid x\text{ whr }v=x\mid \forall _{v}.x\mid \exists
_{v}.x\mid \lambda _{v}.x, 
\]%
where $v$ is a variable and where $f$ is a function symbol\footnote{%
For simplicity we use only single arguments in function applications, and
single variables in binding expressions. It is straightforward to generalize
this to multiple arguments and multiple variables.}.

\subsection{Substitutions}

A substitution $\sigma $ is a function that maps variables to expressions.
It is assumed that $\sigma $ has finite support, in other words there is a
finite number of variables $v$ for which $\sigma (v)\neq v$. We define the
substitution update $\sigma \lbrack v:=v^{\prime }]$ as follows:%
\[
\sigma \lbrack v:=v^{\prime }](w)=\left\{ 
\begin{array}{l}
v^{\prime }\text{ if }w=v \\ 
\sigma (w)\text{ otherwise}%
\end{array}%
\right. 
\]

\subsection{Capture avoiding substitutions}

Let $\sigma $ be a substitution that maps variables to data expressions, and
let $x$ be an arbitrary data expression. Let $FV(x)$ be the free variables
in $x$, and let $FV(\sigma )$ be the free variables in the right hand side
of $\sigma $. More precisely,%
\[
FV(\sigma )=\bigcup_{v\in domain(\sigma )}FV(\sigma (v))\setminus \{v\}.
\]%
We define a function $C$ that computes the capture avoiding substitution $%
\sigma (x)$ using $C(x,\sigma ,FV(x)\cup FV(\sigma ))$. The function $C$ is
recursively defined as follows:%
\[
\begin{array}{lll}
C(v,\sigma ,V) & = & \sigma (v) \\ 
C(f,\sigma ,V) & = & f \\ 
C(x(x_{1}),\sigma ,V) & = & C(x,\sigma ,V)(C(x_{1},\sigma ,V)) \\ 
&  &  \\ 
C(x\text{ whr }v=x_{1},\sigma ,V) & = & \left\{ 
\begin{array}{l}
C(x,\sigma ,V\cup \{v\})\text{ whr }v=C(x_{1},\sigma ,V\cup \{v\})\text{ if }%
\sigma (v)=v\text{ and }v\notin V \\ 
C(x,\sigma ^{\prime },V\cup \{v^{\prime }\})\text{ whr }v^{\prime
}=C(x_{1},\sigma ^{\prime },V\cup \{v^{\prime }\})\text{ otherwise}%
\end{array}%
\right.  \\ 
&  &  \\ 
C(\Lambda v.x,\sigma ,V) & = & \left\{ 
\begin{array}{l}
\Lambda v.C(x,\sigma ,V\cup \{v\})\text{ if }\sigma (v)=v\text{ and }v\notin
V \\ 
\Lambda v^{\prime }.C(x,\sigma ^{\prime },V\cup \{v^{\prime }\})\text{
otherwise,}%
\end{array}%
\right. 
\end{array}%
\]%
where $\Lambda \in \{\forall ,\exists ,\lambda \}$, where $v^{\prime }$ is
an arbitrary variable such that $\sigma (v^{\prime })=v^{\prime }$ and $%
v^{\prime }\notin V$, and where $\sigma ^{\prime }=\sigma \lbrack
v:=v^{\prime }]$. The function $C$ can be extended to assignments as follows:%
\[
\begin{array}{lll}
C(v=x,\sigma ,V) & = & \left\{ 
\begin{array}{l}
v=C(x,\sigma ,V\cup \{v\})\text{ if }\sigma (v)=v\text{ and }v\notin V \\ 
v^{\prime }=C(x,\sigma ^{\prime },V\cup \{v^{\prime }\})\text{ otherwise}%
\end{array}%
\right. 
\end{array}%
\]

\paragraph{Example}

Let $x=\forall b:\mathbb{B}.b\Rightarrow \forall c:\mathbb{B}.c\Rightarrow d$
and let $\sigma =[d:=b]$. Then $C(x,\sigma ,FV(x)\cup FV(\sigma ))=\forall
b^{\prime }:\mathbb{B}.b^{\prime }\Rightarrow \forall c:\mathbb{B}%
.c\Rightarrow b$.

\end{document}
