
\documentclass{article}
\usepackage{amsfonts}

%%%%%%%%%%%%%%%%%%%%%%%%%%%%%%%%%%%%%%%%%%%%%%%%%%%%%%%%%%%%%%%%%%%%%%%%%%%%%%%%%%%%%%%%%%%%%%%%%%%%%%%%%%%%%%%%%%%%%%%%%%%%%%%%%%%%%%%%%%%%%%%%%%%%%%%%%%%%%%%%%%%%%%%%%%%%%%%%%%%%%%%%%%%%%%%%%%%%%%%%%%%%%%%%%%%%%%%%%%%%%%%%%%%
%TCIDATA{OutputFilter=LATEX.DLL}
%TCIDATA{Version=5.50.0.2890}
%TCIDATA{<META NAME="SaveForMode" CONTENT="1">}
%TCIDATA{BibliographyScheme=Manual}
%TCIDATA{Created=Monday, January 21, 2013 17:27:12}
%TCIDATA{LastRevised=Monday, January 28, 2013 13:24:20}
%TCIDATA{<META NAME="GraphicsSave" CONTENT="32">}
%TCIDATA{<META NAME="DocumentShell" CONTENT="Standard LaTeX\Blank - Standard LaTeX Article">}
%TCIDATA{CSTFile=40 LaTeX article.cst}

\newtheorem{theorem}{Theorem}
\newtheorem{acknowledgement}[theorem]{Acknowledgement}
\newtheorem{algorithm}[theorem]{Algorithm}
\newtheorem{axiom}[theorem]{Axiom}
\newtheorem{case}[theorem]{Case}
\newtheorem{claim}[theorem]{Claim}
\newtheorem{conclusion}[theorem]{Conclusion}
\newtheorem{condition}[theorem]{Condition}
\newtheorem{conjecture}[theorem]{Conjecture}
\newtheorem{corollary}[theorem]{Corollary}
\newtheorem{criterion}[theorem]{Criterion}
\newtheorem{definition}[theorem]{Definition}
\newtheorem{example}[theorem]{Example}
\newtheorem{exercise}[theorem]{Exercise}
\newtheorem{lemma}[theorem]{Lemma}
\newtheorem{notation}[theorem]{Notation}
\newtheorem{problem}[theorem]{Problem}
\newtheorem{proposition}[theorem]{Proposition}
\newtheorem{remark}[theorem]{Remark}
\newtheorem{solution}[theorem]{Solution}
\newtheorem{summary}[theorem]{Summary}
\newenvironment{proof}[1][Proof]{\noindent\textbf{#1.} }{\ \rule{0.5em}{0.5em}}
\input{include/tcilatex}
\newcommand{\Bool}{\mathbb{B}}


\begin{document}


\section{Capture avoiding substitutions}

This document specifies how capture avoiding substitutions are currently
implemented in mCRL2.

\subsection{Data expressions}

mCRL2 data expressions $x$ are characterized by the following grammar:%
\[
x::=v\mid f\mid x(x)\mid x\text{ whr }v=x\mid \forall _{v}.x\mid \exists
_{v}.x\mid \lambda _{v}.x,
\]%
where $v$ is a variable and where $f$ is a function symbol\footnote{%
For simplicity we use only single arguments in function applications, and
single variables in binding expressions. It is straightforward to generalize
this to multiple arguments and multiple variables.}.

\subsection{Substitutions}

A substitution $\sigma $ is a function that maps variables to expressions.
It is assumed that $\sigma $ has finite support, in other words there is a
finite number of variables $v$ for which $\sigma (v)\neq v$. We define the
substitution update $\sigma \lbrack v:=v^{\prime }]$ as follows:%
\[
\sigma \lbrack v:=v^{\prime }](w)=\left\{
\begin{array}{l}
v^{\prime }\text{ if }w=v \\
\sigma (w)\text{ otherwise}%
\end{array}%
\right.
\]

\subsection{Capture avoiding substitutions}
\label{captavoid}

Let $\sigma $ be a substitution that maps variables to data expressions, and
let $x$ be an arbitrary data expression. Let $FV(x)$ be the free variables
in $x$, and let $FV(\sigma )$ be the free variables in the right hand side
of $\sigma $. More precisely,%
\[
FV(\sigma )=\bigcup_{v\in \mathit{domain}(\sigma )}FV(\sigma (v))\setminus \{v\}.
\]%
We define a function $C$ that computes the capture avoiding substitution $%
\sigma (x)$ using $C(x,\sigma ,FV(x)\cup FV(\sigma ))$. The function $C$ is
recursively defined as follows:%
\[
\begin{array}{lll}
C(v,\sigma ,V) & = & \sigma (v) \\
C(f,\sigma ,V) & = & f \\
C(x(x_{1}),\sigma ,V) & = & C(x,\sigma ,V)(C(x_{1},\sigma ,V)) \\
&  &  \\
C(x\text{ whr }v=x_{1},\sigma ,V) & = & \left\{
\begin{array}{l}
C(x,\sigma ,V\cup \{v\})\text{ whr }v=C(x_{1},\sigma ,V\cup \{v\})\text{ if }%
\sigma (v)=v\text{ and }v\notin V \\
C(x,\sigma ^{\prime },V\cup \{v^{\prime }\})\text{ whr }v^{\prime
}=C(x_{1},\sigma ^{\prime },V\cup \{v^{\prime }\})\text{ otherwise}%
\end{array}%
\right.  \\
&  &  \\
C(\Lambda v.x,\sigma ,V) & = & \left\{
\begin{array}{l}
\Lambda v.C(x,\sigma ,V\cup \{v\})\text{ if }\sigma (v)=v\text{ and }v\notin
V \\
\Lambda v^{\prime }.C(x,\sigma ^{\prime },V\cup \{v^{\prime }\})\text{
otherwise,}%
\end{array}%
\right.
\end{array}%
\]%
where $\Lambda \in \{\forall ,\exists ,\lambda \}$, where $v^{\prime }$ is
an arbitrary variable such that $\sigma (v^{\prime })=v^{\prime }$ and $%
v^{\prime }\notin V$, and where $\sigma ^{\prime }=\sigma \lbrack
v:=v^{\prime }]$. The function $C$ can be extended to assignments as follows\footnote{The 
definition of $C$ to assignments is not correct and not how they have been implemented.}:
%
\[
\begin{array}{lll}
C(v=x,\sigma ,V) & = & \left\{
\begin{array}{l}
v=C(x,\sigma ,V\cup \{v\})\text{ if }\sigma (v)=v\text{ and }v\notin V \\
v^{\prime }=C(x,\sigma ^{\prime },V\cup \{v^{\prime }\})\text{ otherwise}%
\end{array}%
\right.
\end{array}%
\]

\paragraph{Example}

Let $x=\forall b{:}\mathbb{B}. b\Rightarrow \forall c{:}\mathbb{B}.c\Rightarrow d$
and let $\sigma =[d:=b]$. Then $C(x,\sigma ,FV(x)\cup FV(\sigma ))=\forall
b^{\prime }{:}\mathbb{B}.b^{\prime }\Rightarrow \forall c{:}\mathbb{B}%
.c\Rightarrow b$.

\subsubsection{Capture avoiding substitutions with an identifier generator}
Let $\sigma$ be a subsitution that maps variables to data expressions.
In this section a substitution is defined that is more efficient than the capture avoiding substitution
of section \ref{captavoid} because it does not require the calculation of a set $V$ of variables.

It does require that $\sigma$ can indicate efficiently whether a variable occurs in the $\sigma(y)$
(with $\sigma(y)\not=y$) for some variable $y$. Furthermore, it requires a identifier generator, that
can generate variable names that are guaranteed to be fresh in the sense that they do not occur in any
term. 

This substitution has been implemented as \texttt{replace\_variables\_capture\_avoiding\_with\_an\_identifier\_generator}.
We use $\mathit{FV}(x)$, $\mathit{FV}(\sigma)$ and $\sigma[v:=v']$ as defined in the previous section. 
\vspace{2ex}\\
The substitution is defined as $\hat{C}$ that calculates $\sigma(x)$ using $\hat{C}(x,\sigma)$ recursively 
as follows:
\[
\begin{array}{lll}
\hat{C}(v,\sigma) & = & \sigma (v) \\
\hat{C}(f,\sigma) & = & f \\
\hat{C}(x(x_{1}),\sigma) & = & \hat{C}(x,\sigma)(\hat{C}(x_{1},\sigma)) \\
&  &  \\
\hat{C}(x\text{ whr }v=x_{1},\sigma) & = & \left\{
\begin{array}{ll}
\hat{C}(x,\sigma[v:=v])\text{ whr }v=\hat{C}(x_{1},\sigma)&\text{if }%
v\notin \mathit{FV}(\sigma), \\
\hat{C}(x,\sigma[v:=v'] )\text{ whr }v'=\hat{C}(x_{1},\sigma'\})&\text{otherwise}.%
\end{array}%
\right.  \\
&  &  \\
\hat{C}(\Lambda v.x,\sigma ,V) & = & \left\{
\begin{array}{ll}
\Lambda v.\hat{C}(x,\sigma[v:=v] )&\text{if }v\notin \mathit{FV}(\sigma), \\
\Lambda v'.\hat{C}(x,\sigma',V\cup \{v'\})&\text{otherwise,}%
\end{array}%
\right.
\end{array}%
\]%
where $\Lambda \in \{\forall ,\exists ,\lambda \}$, where $v'$ is
a fresh variable such that $\sigma (v')=v'$ and $%
v'\notin \mathit{FV}(\sigma)\cup\mathit{FV}(x)$. The identifier generator is
used to generate the name for $v'$.

In the examples below $[]$ is the substitution mapping each variable onto itself and $[w:=v']$ is the subsitution
mapping all variables onto itself, except that $w$ is mapped to $v'$.

\paragraph{Example}
Let $x=\forall b{:} \Bool.b\Rightarrow \forall c{:}\Bool.c\Rightarrow d$ and let $\sigma=[d:=b]$. Then
$\hat{C}(x,\sigma)=\forall b'{:}\Bool.b'\Rightarrow\forall c{:}\Bool.c\Rightarrow b$ where $b'$ is a fresh 
variable.

\paragraph{Example}
It is necessary that $v'$ above is chosen such that $v'\notin \mathit{FV}(\sigma)\cup\mathit{FV}(x)$.
We provide two examples to show what goes wrong if this condition is not satisfied. 
\begin{enumerate}
\item
If $v'\notin \mathit{FV}(\sigma)$ is not required, the following is possible: $\hat{C}(\forall v.w,[w:=v'])=\forall
v'.\hat{C}(w,[w:=v'])=\forall v'.v'$.
\item
If $v'\notin \mathit{FV}(x)$ is not required, it is possible that: $\hat{C}(\forall v,v',[])=\forall v'.\hat{C}(v',[])=\forall v'.v$. 
\end{enumerate}
\paragraph{Example}
In a where clause the substitutions applied to the equations after the where can remain unchanged.
E.g., $\hat{C}(f(u,v)\text{ whr }v=v,[u:=v])=\hat{C}(f(u,v),[u:=v,v:=v'])\text{ whr }v'=\hat{C}(v,[u:=v])=f(v,v')\text{ whr }
v'=v$. In an expression $f(u,v)\text{ whr }v=v$ the variable $v$ at the lhs of the `$=$' is a local variable,
whereas the $v$ at the rhs is globally bound. 
\end{document}
