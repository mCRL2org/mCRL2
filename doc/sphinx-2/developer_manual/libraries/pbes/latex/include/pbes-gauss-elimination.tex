%TCIDATA{Version=5.50.0.2890}
%TCIDATA{LaTeXparent=1,1,pbes-implementation-notes.tex}
                      

\section{Gau\ss\ Elimination}

A predicate formula $\varphi $ is defined by the following grammar:%
\begin{equation*}
\varphi ::=b|X(e)|\lnot \varphi |\varphi \wedge \varphi |\varphi \vee
\varphi |\varphi \rightarrow \varphi |\forall d:D.\varphi |\exists
d:D.\varphi |true |false
\end{equation*}%
where $b$ is a data term of sort $\mathbb{B}$, $X$ is a predicate variable, $%
d$ is a data variable of sort $D$, $e$ is a data term, $true $ represents $%
true$, and $false $ represents $false$.

\begin{definition}
(Predicate Variable Substitution) Let $\varphi ,\psi $ be predicate formulae
and $X$ a predicate variable. Then we define $\psi \lbrack \varphi /X]$ as
the result of applying the substitution $X:=\varphi $ to the formula $\psi $%
. To make this more precise: suppose $X$ is declared as $X(d:D)$, then any
occurrence $X(\overline{d})$ in $\psi $ is replaced by $\varphi \lbrack d:=%
\overline{d}]$.
\end{definition}

\begin{lemma}
(Substitution) Let $\mathcal{E}$ be an equation system for which $X,Y\notin
bnd(\mathcal{E})$, then:%
\begin{equation*}
(\sigma X(d:D)=\varphi )\mathcal{E}(\sigma ^{\prime }Y(e:E)=\psi )\equiv
(\sigma X(d:D)=\varphi )[\psi /Y]\mathcal{E}(\sigma ^{\prime }Y(e:E)=\psi )
\end{equation*}
\end{lemma}

\begin{definition}
(Approximation) Let $\varphi ,\psi $ be predicate formulae and $X$ a
predicate variable. We inductively define $\psi \lbrack \varphi /X]^{k}$ as
follows:%
\begin{eqnarray*}
&&\psi \lbrack \varphi /X]^{0}\overset{def}{=}\varphi \\
&&\psi \lbrack \varphi /X]^{k+1}\overset{def}{=}\psi \lbrack \varphi /X]^{k}
\end{eqnarray*}
\end{definition}

Thus, $\psi \lbrack \varphi /X]^{k}$ represents the result of recursively
substituting $\varphi $ for $X$ in $\psi $.

\begin{lemma}
(Approximants as Solutions) Let $\varphi $ be a predicate formula and $k\in 
\mathbb{N}$ be a natural number. Then%
\begin{eqnarray*}
(\mu X(d &:&D)=\varphi \lbrack false /X]^{k})\Rrightarrow (\mu
X(d:D)=\varphi ) \\
(\nu X(d &:&D)=\varphi )\Rrightarrow (\nu X(d:D)=\varphi \lbrack true
/X]^{k})
\end{eqnarray*}
\end{lemma}

\begin{lemma}
(Stable Approximants as Solutions) Let $\varphi $ be a predicate formula and 
$k\in \mathbb{N}$ be a natural number. Then%
\begin{eqnarray*}
\text{if }\varphi \lbrack false/X]^{k} &\longleftrightarrow &\varphi \lbrack
false/X]^{k+1}\text{ then }(\mu X(d:D)=\varphi \lbrack false/X]^{k})\equiv
(\mu X(d:D)=\varphi ) \\
\text{if }\varphi \lbrack true/X]^{k} &\longleftrightarrow &\varphi \lbrack
true/X]^{k+1}\text{ then }(\nu X(d:D)=\varphi \lbrack true/X]^{k})\equiv
(\nu X(d:D)=\varphi )
\end{eqnarray*}
\end{lemma}

\subsection{Gau\ss\ Elimination Algorithm}

Let $\mathcal{E}$ be an equation system of the form%
\begin{equation*}
\mathcal{E=(}\sigma _{1}X_{1}(d_{1}:D_{1})=\varphi _{1})\cdots (\sigma
_{n}X_{n}(d_{n}:D_{n})=\varphi _{n}),
\end{equation*}%
and let $r$ be a rewrite function that maps a pbes expression $\varphi $ to
an equivalent expression $\varphi ^{\prime }$.

Then we define%
\begin{equation*}
\begin{array}{l}
\text{\textsc{Gau\textsc{\ss\ }Elimination(}}\mathcal{E},r,\text{\textsc{%
solve)}} \\ 
\mathcal{E}^{\prime }:=\varepsilon \\ 
i:=n \\ 
\text{\textbf{while} \textbf{not} }i=0 \\ 
\text{\textbf{do}} \\ 
\qquad \varphi _{i}:=\text{\textsc{solve(}}\sigma _{i}X_{i}=\varphi _{i}%
\text{\textsc{)}} \\ 
\qquad \mathcal{E}^{\prime }:=\mathcal{E}^{\prime }(\sigma _{i}X_{i}=\varphi
_{i}) \\ 
\qquad \text{\textbf{for }}k=1\text{ \textbf{to }}i-1\text{ \textbf{do }}%
\varphi _{k}:=r(\varphi _{k}[\varphi _{i}/X_{i}])\text{ \textbf{od}} \\ 
\qquad i:=i-1 \\ 
\text{\textbf{od}} \\ 
\text{\textbf{return }}\mathcal{E}^{^{\prime }}%
\end{array}%
\end{equation*}%
Here \textsc{SolveEquation} is an algorithm that solves a pbes equation,
such that the resulting equation has no reference to the predicate variable
in its right hand side. An example of such a solve equation algorithm is 
\textsc{Approximate}.%
\begin{equation*}
\begin{array}{l}
\text{\textsc{Approximate(}}\sigma X=\varphi ,\text{\textsc{compare)}} \\ 
j:=0 \\ 
\text{\textbf{if }}\sigma =\nu \text{ \textbf{then} }\psi _{0}:=true\text{ 
\textbf{else} }\psi _{0}:=false \\ 
\text{\textbf{repeat}} \\ 
\qquad \psi _{j+1}:=\varphi \lbrack \psi _{j}/X] \\ 
\qquad j:=j+1 \\ 
\text{\textbf{until }\textsc{compare}}(\psi _{j},\psi _{j+1}) \\ 
\text{\textbf{return }}\sigma X=\psi _{j}%
\end{array}%
\end{equation*}

Also pattern matching algorithms exist for this. The \textsc{Gau\textsc{\ss\ 
}Elimination} algorithm solves the equation system $\mathcal{E}$ for the
predicate variable $X_{1}$. To solve the system $\mathcal{E}$ for all
variables, the algorithm has to be applied repeatedly.

\subsection{Solving a BES}

If the equation system $\mathcal{E}$ is a BES (i.e. the predicate variables
have no parameters), then the following simple approximate function can be
used to solve it:%
\begin{equation*}
\begin{array}{l}
\text{\textsc{Approximate-BES(}}\sigma X=\varphi \text{\textsc{)}} \\ 
\text{\textbf{if }}\sigma =\nu \text{ \textbf{then} }\psi _{0}:=true\text{ 
\textbf{else} }\psi _{0}:=false \\ 
\text{\textbf{return} \textsc{Simplify}(}\sigma X=\varphi \lbrack \psi
_{0}/X]\text{)}%
\end{array}%
\end{equation*}

\newpage 
