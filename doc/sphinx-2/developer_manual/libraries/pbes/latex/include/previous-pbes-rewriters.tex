%TCIDATA{Version=5.50.0.2890}
%TCIDATA{LaTeXparent=1,1,pbes-implementation-notes.tex}
                      

\section{PBES rewriters}

\subsection{Quantifier Elimination Rewriter}

This section describes a rewriter on predicate formulae that eliminates
quantifiers. It is based on the following property%
\begin{equation*}
\begin{array}{cc}
\forall _{x:X}.\varphi \equiv \dbigwedge\limits_{y:X}\varphi \lbrack
x:=y]\quad & \exists _{x:X}.\varphi \equiv \dbigvee\limits_{y:X}\varphi
\lbrack x:=y],%
\end{array}%
\end{equation*}

where the conjunction and disjunction on the right hand sides may be
infinite. Because of this, the rewriter we describe here is not guaranteed
to terminate. However, in many practical cases the rewriter can compute a
finite expression even if the quantifier variables are of infinite sort. An
example of this is the formula $\forall _{n:\mathbb{N}}.(n>2)\vee X(n)$ that
can be rewritten into $X(0)\wedge X(1)\wedge X(2)$.

We assume that the sorts of quantifier variables can be enumerated. By this
we mean the existence of a function $enum$ that maps an arbitrary term $d:D$
to a finite set of terms $\{d_{1},\cdots ,d_{k}\}$, such that $%
range(d)=\dbigcup\limits_{i=1\cdots k}range(d_{i})$, where $range(d)$ is the
set of closed terms obtained from $d$ by substituting values for the free
variables of $d$. For example, if natural numbers are represented by $%
S^{n}(0)$, with $S$ a function that expresses the successor of a number,
then possible enumerations of the term $n$ are $\{0,S(n^{\prime })\}$ and $%
\{0,S(0),S(S(n^{\prime \prime }))\}$. Let $id$ be the identity function and
let $\sigma \lbrack d_{1}:=e_{1},\cdots ,d_{n}:=e_{n}]$ be the function $%
\sigma ^{\prime }$ with $\sigma ^{\prime }(x)=e_{i}$ if $x=d_{i}$ and $%
\sigma ^{\prime }(x)=\sigma (x)$ otherwise.

A parameter of the \textsc{EliminateQuantifiers }algorithm is a rewriter $R$
on quantifier free predicate formulae, that is expected to have the
following properties:%
\begin{eqnarray*}
R(\bot ) &=&\bot \\
(R(t) &=&R(t^{\prime }))\Rightarrow t\simeq t^{\prime }\text{,}
\end{eqnarray*}%
where $t\simeq t^{\prime }$ indicates that $t$ and $t^{\prime }$ are
equivalent. 
\begin{equation*}
\begin{tabular}{l}
\textsc{EliminateQuantifiers(}$Q_{d_{1}:D_{1},\ldots ,d_{n}:D_{n}}.\varphi
,R $\textsc{)} \\ 
$\text{\textbf{if }}freevars(R(\varphi ))\cap \{d_{1},\cdots
,d_{n}\}=\emptyset $ $\text{\textbf{then return }}R(\varphi )$ \\ 
$V:=\emptyset $ \\ 
$\text{\textbf{for }}i\in \{1,\ldots ,n\}\text{ \textbf{do }}%
E_{i}:=\{d_{i}\} $ \\ 
$\text{\textbf{do}}$ \\ 
$\qquad \text{\textbf{choose }}e_{k}\in E_{k}$, such that $\mathsf{dvar}%
(e_{k})\neq \emptyset $ \\ 
$\qquad E_{k}:=E_{k}\backslash \{e_{k}\}$ \\ 
$\qquad \text{\textbf{for }}e\in enum(e_{k}):$ \\ 
$\qquad \qquad W:=\emptyset $ \\ 
$\qquad \qquad \text{\textbf{for }}\sigma \in \{id[d_{1}:=f_{1},\cdots
,d_{k-1}:=f_{k-1},d_{k}:=e,d_{k+1}:=f_{k+1},\cdots ,d_{n}:=f_{n}]$ \\ 
$\qquad \qquad \qquad \wedge f_{i}\in E_{i}\quad (i=1,\cdots ,n)\}:$ \\ 
$\qquad \qquad \qquad W:=W\cup \{R(\sigma (\varphi )\}$ \\ 
$\qquad \qquad \text{\textbf{if }}stop_{Q}\in W\text{ \textbf{then return }}%
stop_{Q}$ \\ 
$\qquad \qquad V:=V\cup \{w\in W\ |\ \mathsf{dvar}(w)\subset \mathsf{dvar}%
(\varphi )\}$ \\ 
$\qquad \qquad \text{\textbf{if }}\{w\in W\ |\ \mathsf{dvar}(w)\varsubsetneq 
\mathsf{dvar}(\varphi )\}\neq \emptyset $ $\text{\textbf{then }}%
E_{k}:=E_{k}\cup \{e\}$ \\ 
$\qquad $\textbf{rof} \\ 
$\text{\textbf{while }}\forall _{i\in \{1,\ldots ,n\}}.E_{i}\neq \emptyset $
\\ 
$\text{\textbf{return} }\dbigoplus\limits_{v\in V}v,$%
\end{tabular}%
\end{equation*}%
where $stop_{Q}=\bot $ and $\dbigoplus =\dbigwedge $ in case $Q=\forall $,
and where $stop_{Q}=\top $ and $\dbigoplus =\dbigvee $ in case $Q=\exists $

\subsection{OnePointRule Quantifier Elimination}

The OnePointRule Quantifier Elimination rewriter rewrites certain patterns
of PBES quantifier expressions as follows:%
\begin{equation*}
\begin{array}{lll}
\exists _{x:X}.(x=e)\wedge \varphi & \rightarrow & \varphi \lbrack x:=e] \\ 
\exists _{x:\mathbb{B}}.x\wedge \varphi & \rightarrow & \varphi \lbrack
x:=true] \\ 
\exists _{x:\mathbb{B}}.\lnot x\wedge \varphi & \rightarrow & \varphi
\lbrack x:=false] \\ 
\forall _{x:X}.(x\neq e)\vee \varphi & \rightarrow & \varphi \lbrack x:=e]
\\ 
\forall _{x:\mathbb{B}}.x\vee \varphi & \rightarrow & \varphi \lbrack
x:=false] \\ 
\forall _{x:\mathbb{B}}.\lnot x\vee \varphi & \rightarrow & \varphi \lbrack
x:=true]%
\end{array}%
\end{equation*}%
Note that it is straightforward to generalize these patterns to quantifier
expressions as they occur in the mCRL2 toolset ($\exists
_{x_{1}:X_{1},\cdots ,x_{n}:X_{n}}.\varphi $ and $\forall
_{x_{1}:X_{1},\cdots ,x_{n}:X_{n}}.\varphi $).

\subsubsection{Preprocessing}

It turns out that the data rewriters in the mCRL2 toolset do not rewrite $%
\lnot (\varphi =\psi )$ into $\varphi \neq \psi $. Therefore it is useful to
apply the following as a preprocessing step to data expressions:%
\begin{equation*}
\begin{array}{lll}
\lnot (\varphi _{1}\wedge \cdots \wedge \varphi _{n}) & \rightarrow & \lnot
\varphi _{1}\vee \cdots \vee \lnot \varphi _{n} \\ 
\lnot (\varphi _{1}\vee \cdots \vee \varphi _{n}) & \rightarrow & \lnot
\varphi _{1}\wedge \cdots \wedge \lnot \varphi _{n} \\ 
\lnot (\varphi =\psi ) & \rightarrow & \varphi \neq \psi \\ 
\lnot (\varphi \neq \psi ) & \rightarrow & \varphi =\psi \\ 
\lnot \lnot \varphi & \rightarrow & \varphi%
\end{array}%
\end{equation*}
