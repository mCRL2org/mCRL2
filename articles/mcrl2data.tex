\documentclass[a4paper,fleqn]{article}

% title and author
\title{Data types for mCRL2}
\author{Aad Mathijssen} 

% packages
\usepackage[english]{babel}
\usepackage[final]{graphics}
\usepackage{amsmath}
\usepackage{amsfonts}
\usepackage{array}
\usepackage{calc}
\usepackage{xspace}

% General layout
% --------------

%add headings
\pagestyle{headings}

% increase pagewidth
\addtolength{\textwidth}{20mm}
\addtolength{\oddsidemargin}{-10mm}
\addtolength{\evensidemargin}{10mm}

% set the indentation of the math environment to 10mm
\setlength{\mathindent}{10mm}

% equations are unique up to their section
\renewcommand{\theequation}{\arabic{equation}}

% do not put subsubsections in the table of contents
\addtocounter{tocdepth}{-1}

% column types that change column types l,c,r from math mode to LR
% and the other way round
\newcolumntype{L}{>{$}l<{$}}%stopzone%stopzone%stopzone
\newcolumntype{C}{>{$}c<{$}}%stopzone%stopzone%stopzone
\newcolumntype{R}{>{$}r<{$}}%stopzone%stopzone%stopzone

% Environments
% ------------

% equations: eqnarray environment with no outer column spacing and tighter
% intercolumn spacing
\newenvironment{equations}
  {\setlength{\arraycolsep}{2pt}%
   \begin{array}{@{}lll@{}}%
  }
  {\end{array}%
  }

% tightarray: array with no outcolumn spacing and tighter intercolumn spacing
\newenvironment{tightarray}[1]
  {\setlength{\arraycolsep}{2pt}%
   \begin{array}{@{}#1@{}}%
  }
  {\end{array}%
  }

% definitions: list of definitions where:
% - the optional argument denotes the space between successive items
%   (default 0.15em)
% - items are horizontally aligned at the math indent
\newenvironment{definitions}[1][0.15em]
  {\begin{list}%
    {}%
    {\setlength{\parsep}{0pt}%
     \setlength{\itemsep}{#1}%
     \setlength{\leftmargin}{\mathindent}%
     \setlength{\labelwidth}{\mathindent - \labelsep}%
    }
  }
  {\end{list}}

% tdefinitions: list of tagged definitions where:
% - the optional argument denotes the space between successive items
%   (default 0.15em)
% - items are horizontally aligned at the math indent
% - each item is tagged with the mandatory argument
\newenvironment{tdefinitions}[2][0.15em]
  {\begin{list}%
    {#2}%
    {\setlength{\parsep}{0pt}%
     \setlength{\itemsep}{#1}%
     \setlength{\leftmargin}{\mathindent}%
     \setlength{\labelwidth}{\mathindent - \labelsep}%
    }
  }
  {\end{list}}

% edefinitions: list of enumerated definitions where:
% - the optional argument denotes the space between successive items
%   (default 0.15em);
% - items are horizontally aligned at the math indent;
% - each item is numbered with a parenthesized roman numeral.
\newcounter{edefinitioncount}
\newenvironment{edefinitions}[1][0.15em]
  {\begin{list}%
    {(\roman{edefinitioncount})}%
    {\renewcommand{\theenumi}{\roman{enumi}}%
     \renewcommand{\labelenumi}{(\theenumi)}%
     \usecounter{edefinitioncount}%
     \setlength{\parsep}{0pt}%
     \setlength{\itemsep}{#1}%
     \setlength{\leftmargin}{\mathindent}%
     \setlength{\labelwidth}{\mathindent - \labelsep}%
    }
  }
  {\end{list}}
  
% entry: list of entries where:
% - the optional argument denotes the space between successive items
%   (default 0.15em);
% - items are horizontally aligned at the math indent;
% - each item is labelled.
\newenvironment{entry}[2][0.15em]%
  {\begin{list}{}%
    {\renewcommand{\makelabel}[1]{\textsf{##1:}\hfil}%
     \settowidth{\labelwidth}{\textsf{#2:}}%
     \setlength{\leftmargin}{\labelwidth+\labelsep}%     
     \setlength{\parsep}{0pt}%
     \setlength{\itemsep}{#1}%
    }%
  }%
  {\end{list}}

% derivation: calculational derivation where expressions (\expr) are related by
% means of transformations (\tran). A transformation is denoted by a symbol and
% a hint. Expressions and transformations can be broken into several lines
% using \breakexpr and \breaktran.
\newenvironment{derivation}
{\par\addtolength{\baselineskip}{1mm}\begin{tabbing}\hspace{5mm}\=\hspace{5mm}\=\hspace{5mm}\=\kill}
{\end{tabbing}\par}
\newcommand{\expr}[1]{\>\>$#1$}
\newcommand{\tran}[2]{\\*\>$#1$\>\>\{ #2 \}\\}
\newcommand{\breakexpr}{$\\*\>\>$}
\newcommand{\breaktran}{\\*\>\>\>\hspace{8pt}}

% Theorem-like environments that are numbered as s.n, where:
% - s is the section number
% - n is the number of occurrences of all theorem-like environments in s
% We have the following environments:
% - definition
% - theorem
% - lemma
% - corollary
% - property
% - example
% - remark
% - convention
% - specification
% - declaration

\newtheorem{thdefinition}{Definition}[section]
\newenvironment{definition}
  {\begin{thdefinition}\em}
  {\end{thdefinition}}

\newtheorem{ththeorem}[thdefinition]{Theorem}
\newenvironment{theorem}
  {\begin{ththeorem}\em}
  {\end{ththeorem}}

\newtheorem{thcorollary}[thdefinition]{Corollary}
\newenvironment{corollary}
  {\begin{thcorollary}\em}
  {\end{thcorollary}}

\newtheorem{thlemma}[thdefinition]{Lemma}
\newenvironment{lemma}
  {\begin{thlemma}\em}
  {\end{thlemma}}

\newtheorem{thproperty}[thdefinition]{Property}
\newenvironment{property}
  {\begin{thproperty}\em}
  {\end{thproperty}}

\newtheorem{thexample}[thdefinition]{Example}
\newenvironment{example}
  {\begin{thexample}\em}
  {\end{thexample}}

\newtheorem{thremark}[thdefinition]{Remark}
\newenvironment{remark}
  {\begin{thremark}\em}
  {\end{thremark}}

\newtheorem{thconvention}[thdefinition]{Convention}
\newenvironment{convention}
  {\begin{thconvention}\em}
  {\end{thconvention}}

\newtheorem{thspecification}[thdefinition]{Specification}
\newenvironment{specification}
  {\begin{thspecification}\em}
  {\end{thspecification}}

\newtheorem{thdeclaration}[thdefinition]{Declaration}
\newenvironment{declaration}
  {\begin{thdeclaration}\em}
  {\end{thdeclaration}}

% proof: proof of a theorem
\newenvironment{proof}
  {\textbf{Proof}}
  {\frm{\Box}
   \vspace{1ex}%
  }


% Commands
% --------
% --------


% math mode
% ---------

% improvement to $ ... $ such that mathematical formulas cannot be cramped
\newcommand{\frm}[1]{\mbox{\ensuremath{#1}}}

% frm with extra spacing
\newcommand{\for}[1]{\frm{\,#1\,}}


% functions and constants
% -----------------------

% constant
\newcommand{\f}[1]{\ensuremath{\mathit{#1}}}

% function application with 1 argument: f(arg0)
\newcommand{\fa}[2]{\ensuremath{\f{#1}(#2)}}

% function application with 2 arguments: f(arg0, arg1)
\newcommand{\faa}[3]{\ensuremath{\f{#1}(#2, #3)}}

% function application with 3 arguments: f(arg0, arg1, arg2)
\newcommand{\faaa}[4]{\ensuremath{\f{#1}(#2, #3, #4)}}

% function application with 4 arguments: f(arg0, arg1, arg2, arg3)
\newcommand{\faaaa}[5]{\ensuremath{\f{#1}(#2, #3, #4, #5)}}


% functions and types
% -------------------

% to: ->
\newcommand{\To}{\ensuremath{\rightarrow}}

% function application symbol: .
\newcommand{\fap}{\ensuremath{\!\cdot\!}}

% composition: 0
\newcommand{\comp}{\ensuremath{\circ}}

% function composition: o
\newcommand{\fcomp}{\ensuremath{%
  \hspace{0.08em}\mbox{\small\ensuremath{\circ}}\hspace{0.08em}}}
  
% function mapping
\newcommand{\fmap}{
  \hspace{0.08em}\raisebox{0.2ex}{%
  \tiny\ensuremath{\bullet}}\hspace{0.08em}}


% lambda calculus
% ---------------

% abstraction in the typed lambda calculus
\newcommand{\labst}[3]{\ensuremath{\lambda #1\!:\!#2.#3}}

% application in the typed lambda calculus
\newcommand{\lappl}[2]{\ensuremath{#1\ #2}}

% abstraction in the pure (untyped) lambda calculus
\newcommand{\pabst}[2]{\ensuremath{\lambda #1.#2}}

% application in the pure (untyped) lambda calculus
\newcommand{\pappl}[2]{\ensuremath{#1\ #2}}

% sets
% ----

% set of elements: {e}
\newcommand{\set}[1]{\ensuremath{\{\,#1\,\}}}

% bag of elements: {e}
\newcommand{\bag}[1]{\ensuremath{\set{#1}}}

% set difference: s \ t
\newcommand{\sdiff}[2]{\ensuremath{#1\ \backslash\ #2}}

% set comprehension: { e | c }, where e is an expression or a binding and c is
% a condition
\newcommand{\scompr}[2]{\ensuremath{\set{#1\ |\ #2}}}

% powerset: P(s)
\newcommand{\pow}[1]{\ensuremath{\fa{\mathcal{P}}{#1}}}


% tuples
% ------

% tuple of elements: <e>
\newcommand{\tpl}[1]{\ensuremath{\langle\,#1\,\rangle}}

% pair of elements: <e0,e1>
\newcommand{\pair}[2]{\ensuremath{\tpl{#1\hspace{0.08em},#2}}}


% lists
% -----

% list of a certain type: L(t)
\newcommand{\List}[1]{\ensuremath{\mathcal{L}{\I{#1}}}}

% list of elements: [e]
\newcommand{\lst}[1]{\ensuremath{[\,#1\,]}}

% empty list
\newcommand{\el}{\ensuremath{[\,]}}

% cons: |>
\newcommand{\cons}{\ensuremath{\hspace{0.12em}\triangleright\hspace{0.08em}}}
  
% snoc: <|
\newcommand{\snoc}{\ensuremath{\hspace{0.08em}\triangleleft\hspace{0.12em}}}

% concatenation: ++
\newcommand{\concat}{\ensuremath{\hspace{0.12em}\raisebox{0.2ex}
{\text{\scriptsize+}\!\text{\scriptsize+}}\hspace{0.12em}}}

% take
\newcommand{\take}{\ensuremath{\lceil}}

% drop
\newcommand{\drop}{\ensuremath{\lfloor}}


% logic
% -----

% boolean type: B
\newcommand{\bool}{\ensuremath{\mathbb{B}}}

% true
\newcommand{\true}{\ensuremath{\f{true}}}

% false
\newcommand{\false}{\ensuremath{\f{false}}}

% implies: =>
\newcommand{\limp}{\ensuremath{\Rightarrow}}

% follows: =>
\newcommand{\lfol}{\ensuremath{\Leftarrow}}

% bi-implies: <=>
\newcommand{\lbimp}{\ensuremath{\Leftrightarrow}}

% not with extra spacing
\newcommand{\Lnot}{\ensuremath{\ \lnot\ }}

% and with extra spacing
\newcommand{\Land}{\ensuremath{\ \land\ }}

% or with extra spacing
\newcommand{\Lor}{\ensuremath{\ \lor\ }}

% implies with extra spacing
\newcommand{\Limp}{\ensuremath{\ \limp\ }}

% bi-implies with extra spacing
\newcommand{\Lbimp}{\ensuremath{\ \lbimp\ }}

% quantification in Dijkstra notation: <Q x : y : z>
\newcommand{\quantD}[4]{\ensuremath{\langle\,{#1} {#2} : #3 : {#4}\,\rangle}}

% universal quantification in Dijkstra notation: <forall x : y : z>
\newcommand{\forallD}[3]{\ensuremath{\quantD{\forall}{#1}{#2}{#3}}}

% existential quantification in Dijkstra notation: <exists x : y : z>
\newcommand{\existsD}[3]{\ensuremath{\quantD{\exists}{#1}{#2}{#3}}}

% derivable in #1: |-_#1
\newcommand{\derivable}[1]{\ensuremath{\vdash_{_{#1}}}}

% valid in #1: |=_#1
\newcommand{\valid}[1]{\ensuremath{\models_{_{#1}}}}

%inference rule with 0 premises and 1 conclusion
\newcommand{\infC}[1]{\ensuremath{
  \begin{array}{c} 
    \\\hline 
    #1 
  \end{array}
}}

%inference rule with 1 premise and 1 conclusion
\newcommand{\infPC}[2]{\ensuremath{
  \begin{array}{c} 
    #1 \\\hline 
    #2 
  \end{array}
}}

% inference rule with 2 premises and 1 conclusion
\newcommand{\infPPC}[3]{\ensuremath{
  \begin{array}{c@{\hspace{2em}}c}
    #1 & #2 \\\hline
    \multicolumn{2}{c}{#3}
  \end{array}
}}

%inference rule with 3 premises and 1 conclusion
\newcommand{\infPPPC}[4]{\ensuremath{
  \begin{array}{c@{\hspace{2em}}c@{\hspace{2em}}c}
    #1 & #2 & #3\\\hline
    \multicolumn{3}{c}{#4}
  \end{array}
}}

% arithmetic
% ----------

% natural type: N
\newcommand{\nat}{\ensuremath{\mathbb{N}}}

% positive type: N+
\newcommand{\pos}{\ensuremath{\mathbb{N}^{+}}}

% integral type: Z
\newcommand{\tint}{\ensuremath{\mathbb{Z}}}

% integer division: "div"
\renewcommand{\div}{\ensuremath{\ \mathbf{div}\ }}

% integer remainder: "mod"
\renewcommand{\mod}{\ensuremath{\ \mathbf{mod}\ }}


% miscellaneous
% -------------

% language of a signature Sigma: L(Sigma)
\newcommand{\lang}[1]{\ensuremath{\fa{\mathcal{L}}{\f{#1}}}}

% meaning of a syntactic element
\newcommand{\mean}[1]{%
  [\hspace{-.15em}[\hspace{.15em}#1\hspace{.15em}]\hspace{-.15em}]%
}

% alternative choice
\newcommand{\alt}{\ensuremath{[\hspace{-0.04em}]}}

% bullet
\newcommand{\bul}{\mbox{\small\ensuremath{\bullet}}}

% complexity according to expression: O(e)
\newcommand{\bigo}[1]{\ensuremath{\mathcal{O}(#1)}}

% print #2 inside #3, where #2 is raised by #1
\newlength{\insidewd}%                        Define length command
\newcommand{\inside}[3][0pt]{%		      Definition of inside:
   \settowidth{\insidewd}{#3}%                - Save width of #3
   \raisebox{#1}[0pt]{%                       - Raise #2 by #1
     \makebox[0pt]{\hspace{\insidewd}#2}}%    - Print #2 centered
   #3}%				              - Print #3

% stack #2 on #3, where #2 is raised by #1
\newlength{\stackht}%                         Define length commands
\newcommand{\stack}[3][0pt]{%		      Definition of stack:
   \settoheight{\stackht}{#3}%                - Save height of #3
   \addtolength{\stackht}{#1}%                - Add to #1 to heigth
   \inside[\stackht]{#2}{#3}}%                - Print #2 in #3 at height #3 + #1

% abbreviations
% -------------

% mCRL
\newcommand{\mCRL}{\frm{\mu}CRL\xspace}

% Miscellaneous local definitions
% -------------------------------

% temporary lengths
\newlength{\tlength}

% mCRL2 keywords
\newcommand{\kwsort}{{\bf sort}}
\newcommand{\kwcons}{{\bf cons}}
\newcommand{\kwmap}{{\bf map}}
\newcommand{\kwvar}{{\bf var}}
\newcommand{\kweqn}{{\bf eqn}}
\newcommand{\kwact}{{\bf act}}
\newcommand{\kwstruct}{{\bf struct}}
\newcommand{\kwwhr}{{\bf whr}}
\newcommand{\kwend}{{\bf end}}


% mCRL Sorts
\newcommand{\srtbool}{\f{Bool}}
\newcommand{\srtpos}{\f{Pos}}
\newcommand{\srtnat}{\f{Nat}}
\newcommand{\srtint}{\f{Int}}

% mCRL Operations
\newcommand{\opzeronat}{{0n}}
\newcommand{\oponenat}{{1}}
\newcommand{\optwonat}{{2}}
\newcommand{\opzeroint}{{0i}}
\newcommand{\opposint}{\ensuremath{\f{pos}}}
\newcommand{\opnegint}{\ensuremath{\f{neg}}}
\newcommand{\opnot}{\ensuremath{\f{not}}}
\newcommand{\opand}{\ensuremath{\f{and}}}
\newcommand{\opor}{\ensuremath{\f{or}}}
\newcommand{\opforall}{\ensuremath{\f{forall}}}
\newcommand{\opexists}{\ensuremath{\f{exists}}}
\newcommand{\opsucc}{\ensuremath{\f{succ}}}
\newcommand{\oppred}{\ensuremath{\f{pred}}}
\newcommand{\opadd}{\ensuremath{\f{add}}}
\newcommand{\opminus}{\ensuremath{\f{minus}}}
\newcommand{\opmonus}{\ensuremath{\f{monus}}}
\newcommand{\optimes}{\ensuremath{\f{times}}}
\newcommand{\opdiv}{\ensuremath{\f{div}}}
\newcommand{\opmod}{\ensuremath{\f{mod}}}
\newcommand{\oppower}{\ensuremath{\f{power}}}
\newcommand{\opdouble}{\ensuremath{\f{double}}}
\newcommand{\opif}{\ensuremath{\f{if}}}
\newcommand{\opeq}{\ensuremath{\f{eq}}}
\newcommand{\opneq}{\ensuremath{\f{neq}}}
\newcommand{\opgrt}{\ensuremath{\f{grt}}}
\newcommand{\opsmt}{\ensuremath{\f{smt}}}
\newcommand{\opgreq}{\ensuremath{\f{greq}}}
\newcommand{\opsmeq}{\ensuremath{\f{smeq}}}
\newcommand{\opint}{\ensuremath{\f{int}}}
\newcommand{\opnat}{\ensuremath{\f{nat}}}
\newcommand{\oplambda}[1]{\ensuremath{\f{lambda\_#1}}}
\newcommand{\opappl}[1]{\ensuremath{\f{appl\_#1}}}
\newcommand{\opsubst}[2]{\ensuremath{\f{subst\_#1\_#2}}}
\newcommand{\opindex}[1]{\ensuremath{\f{index\_#1}}}
\newcommand{\opproj}[2]{\ensuremath{\f{proj_{#1}^{#2}}}}
\newcommand{\opheadsymbol}{\ensuremath{\f{opheadsymbol}}}

\newcommand{\Time}{{\bf Time}}
\newcommand{\Bool}{{\bf Bool}}
\newcommand{\sft}{{\sf t}}
\newcommand{\sff}{{\sf f}}
\newcommand{\nul}{{\bf 0}}
\font \aap cmmi10        
\newcommand{\at}[1]{\mbox{\aap ,} #1}
\newcommand{\before}{\mbox{\footnotesize{\frm{\ll}}}}
\newcommand{\ap}{{:}}
\newcommand{\qed}{\hfill\ensuremath{\quad\Box}}
\newcommand{\leftmerge}{\underline {\parallel}}

\newenvironment{mCRL2}%
{\par\bigskip\noindent%
 \begin{tabular}{@{}>{\bf}p{2.3em}L@{\ }L@{\ }L@{\ }L@{\ }L@{\ }L@{\ }L@{\ }L}%
}%
{\end{tabular}\bigskip\par%
}

\begin{document}

\maketitle

\noindent
We provide a syntax for the standard data types of the mCRL2 language. This
syntax is intended to be a practical mix between standard mathematical notation
and the syntax of specification languages and programming languages in general.

\section{Basic formalism}

The underlying theory of the mCRL2 data types is abstract data types.
Abstract data types consist of:
\begin{tdefinitions}{-}
\item sorts and operations on these sorts;
\item equations on terms made up from operations and variables, where the terms
      are of the same sort.
\end{tdefinitions}

\noindent
Sorts are declared using the keyword \frm{\kwsort}, operations using the
keyword \frm{\kwmap} and equations using the keyword \frm{\kweqn}. Variables
that are used in the equations need to be declared using the keyword
\frm{\kwvar}. To distinguish constructor operations from normal operations the
former may be declared using the keyword \frm{\kwcons}. However, the use of
these constructors is strongly discouraged, because in practise they are
implicitly defined via the concrete data types. With the keyword \frm{\kwsort},
also abbreviations for sort expressions may be introduced.

\subsection{Expressions}

There are three kind of expressions in mCRL2, namely expressions over sorts,
over data and over processes. Sort expressions are made up from existing sort
names and from representations of predefined data types and type constructors.
Data expressions are terms constructed from operations and variables. These
data expressions are in the standard functional notation, but in the next
sections also mixfix notation is introduced. To increase readability of the
operation declarations in the following sections, the sorts are left implicit
and underscores indicate the placement of the parameters.

\emph{Where clauses} may be used as an abbreviation mechanism in data
expressions. A where clause is of the form \frm{e\ \kwwhr\ a_{0} = e_{0},
\ldots a_{n} = a_{n}\ \kwend}, with \frm{n \in \nat}. Here, \frm{e} is a data
expression and, for all \frm{i}, \frm{0 \leq i \leq n}, \frm{a_{i}} is an
identifier and \frm{e_{i}} is a data expression. Expression \frm{e} is called
the body and each equation \frm{a_{i} = e_{i}} is called a \emph{definition}.
Each identifier \frm{a_{i}} is used as an abbreviation for \frm{e_{i}} in
\frm{e}, even if \frm{a_{i}} is already defined in the context. Also, an
identifier \frm{a_{i}} may not occur in any of the expressions \frm{e_{j}},
\frm{0 \leq j \leq n}. As a consequence, the order in which expressions occur
is irrelevant.

\section{Predefined data types}
\label{sec:StandardTypes}

A number of predefined data types are provided. For each data type a sort is
provided together with a number of predefined operations.

\subsection{Booleans}

The boolean type is represented by the sort \frm{\srtbool}. For this sort, we
have the following operations.

\bigskip
\begin{tabular}{|l@{\qquad}L@{\qquad}l|}
\hline
Operator                   & \textit{Rich}          & \verb+Plain+\\\hline
true                       & \true                  & \verb+true+\\
false                      & \false                 & \verb+false+\\
negation                   & \lnot \_               & \verb+!_+\\
conjunction                & \_ \land \_            & \verb+_ && _+\\
disjunction                & \_ \lor \_             & \verb+_ || _+\\
implication                & \_ \limp \_            & \verb+_ => _+\\
universal quantification   & \forall \_ {:} \_ . \_ & \verb+forall _:_._+\\
existential quantification & \exists \_ {:} \_ . \_ & \verb+exists _:_._+\\
\hline
\end{tabular}\bigskip

\noindent
In the quantifiers the first two parameters form a variable declaration,
consisting of a name and a sort; the third parameter is the body.

We also have operations that express the equality and inequality of two terms
of the same sort, and an operation that expresses the conditional. These
operations are available for any predefined sort, and are automatically
generated for user defined sorts.

\bigskip
\begin{tabular}{|l@{\qquad}L@{\qquad}l|}
\hline
Operator                   & \textit{Rich}          & \verb+Plain+\\\hline
equality                   & \_ == \_               & \verb+_ == _+\\
inequality                 & \_ \not = \_           & \verb+_ != _+\\
conditional                & \faaa{if}{\_}{\_}{\_}  & \verb+if(_,_,_)+\\
\hline
\end{tabular}\bigskip

\subsection{Numbers}

Positive numbers, natural numbers and integers are represented by the sorts
\frm{\srtpos}, \frm{\srtnat} and \frm{\srtint}. For these sorts, we have the
following operations, where \frm{A,B \in \set{\srtpos, \srtnat, \srtint}}:

\bigskip
\begin{tabular}{|l@{\qquad}L@{\qquad}l|}
\hline
Operator                   & \textit{Rich}              & \verb+Plain+\\\hline
positive numbers           & 1,2,3,\ldots               & \verb+1,2,3,...+\\
natural numbers            & 0,1,2,\ldots               & \verb+0,1,2,...+\\
integers                   & \ldots, -2,-1,0,1,2,\ldots
                                              & \verb+...,-2,-1,0,1,2,...+\\
conversion                 & \fa{A2B}{\_}               & \verb+A2B(_)+\\
less than or equal         & \_ \leq \_                 & \verb+_ <= _+\\
less than                  & \_ < \_                    & \verb+_ < _+\\
greater than or equal      & \_ \geq \_                 & \verb+_ >= _+\\
greater than               & \_ > \_                    & \verb+_ > _+\\
maximum                    & \faa{max}{\_\,}{\_}        & \verb+max(_,_)+\\
minimum                    & \faa{min}{\_\,}{\_}        & \verb+min(_,_)+\\
absolute value             & \fa{abs}{\_}               & \verb+abs(_)+\\
negation                   & - \_                       & \verb+-_+\\
successor                  & \fa{succ}{\_}              & \verb+succ(_)+\\
predecessor                & \fa{pred}{\_}              & \verb+pred(_)+\\
addition                   & \_ + \_                    & \verb-_ + _-\\
subtraction                & \_ - \_                    & \verb+_ - _+\\
multiplication             & \_ * \_                    & \verb+_ * _+\\
integer div                & \_ \div \_                 & \verb+_ div _+\\
integer mod                & \_ \mod \_                 & \verb+_ mod _+\\
exponentiation             & \faa{exp}{\_\,}{\_}        & \verb+exp(_,_)+\\
\hline
\end{tabular}\bigskip

\noindent
Explicit type conversions can be done using the conversion operation
\frm{\f{A2B}}. The other operations perform implicit type conversions, when
needed.

\section{Type constructors}
\label{sec:TypeConstructors}

Type constructors are predefined operations on sorts with which we can
construct sorts. For each sort that is constructed this way a number of
operations are provided.

\subsection{Lists}

Lists, where all elements are of sort \frm{A}, are declared by the sort
expression \frm{\fa{List}{A}}. The following operations are provided for this
sort.

\bigskip
\begin{tabular}{|l@{\qquad}L@{\qquad}l|}
\hline
Operator                       & \textit{Rich}          & \verb+Plain+\\\hline
construction                   & \lst{\_\,,\ldots,\_}   & \verb+[_,...,_]+\\
length                         & \# \_                  & \verb+#_+\\
cons                           & \_ \cons \_            & \verb+_ |> _+\\
snoc                           & \_ \snoc \_            & \verb+_ <| _+\\
concatenation                  & \_ \concat \_          & \verb-_ ++ _-\\
element at position            & \_\ .\ \_                & \verb+_ . _+\\
the first element of a list    & \fa{lhead}{\_}         & \verb+lhead(_)+\\
list without its first element & \fa{ltail}{\_}         & \verb+ltail(_)+\\
the last element of a list     & \fa{rhead}{\_}         & \verb+rhead(_)+\\
list without its last element  & \fa{rtail}{\_}         & \verb+rtail(_)+\\
\hline
\end{tabular}\bigskip

\noindent
The empty list is represented by an empty list construction, i.e.\ \frm{\el}.
Also note that the lists \frm{\lst{a,b}}, \frm{a \cons \lst{b}} and
\frm{\lst{a} \snoc b} are all equivalent.

\subsection{Sets and bags}

Possibly infinite sets and bags where all elements are of sort \frm{A} are
declared by the sort expressions \frm{\fa{Set}{A}} and \frm{\fa{Bag}{A}},
respectively. The following operations are provided for these sorts.

\bigskip
\begin{tabular}{|l@{\qquad}L@{\qquad}l|}
\hline
Operator                       & \textit{Rich}           & \verb+Plain+\\\hline
set enumeration                & \set{\_\,,\ldots,\_}    & \verb+{ _,...,_ }+\\
bag enumeration                & \bag{{\_:\_}\,,\ldots,{\_:\_}}
                                                      & \verb+{ _:_,...,_:_}+\\
comprehension                  & \scompr{\_:\_}{\_}      & \verb+{ _:_ | _ }+\\
element test                   & \_ \in \_               & \verb+_ in _+\\
bag multiplicity               & \faa{count}{\_}{\_}     & \verb+count(_,_)+\\
subset/subbag                  & \_ \subseteq \_         & \verb+_ <= _+\\
proper subset/subbag           & \_ \subset \_           & \verb+_ < _+\\
union                          & \_ \cup \_              & \verb-_ + _-\\
difference                     & \_ - \_                 & \verb+_ - _+\\
intersection                   & \_ \cap \_              & \verb+_ * _+\\
set complement                 & \stack[1.3ex]{\_}{\_}   & \verb+!_+\\
convert set to bag             & \fa{Set2Bag}{\_}        & \verb+Set2Bag(_)+\\
convert bag to set             & \fa{Bag2Set}{\_}        & \verb+Bag2Set(_)+\\
\hline
\end{tabular}\bigskip

\noindent
The empty set of bag is represented by an empty enumeration, i.e.\
\frm{\set{}}. Note that a set enumeration declares a set, not a bag. So e.g.\
\frm{\set{a,b,c}} declares the same set as \frm{\set{a,b,c,c,a,c}}. In a bag
enumeration the number of times an element occurs has to be declared explicitly.
So e.g.\ \frm{\bag{a:2,b:1}} declares a bag consisting of two \frm{a}'s and one
\frm{b}. Also \frm{\bag{a:1,b:1,a:1}} declares the same bag. A set
comprehension \frm{\scompr{x:A}{\fa{P}{x}}} declares the set consisting of all
elements \frm{x} of sort \frm{A} for which predicate \frm{\fa{P}{x}} holds,
i.e.\ \frm{\fa{P}{x}} is an expression of sort \frm{\srtbool}. A bag
comprehension \frm{\scompr{x:A}{\fa{f}{x}}} declares the bag in which each
element \frm{x} occurs \frm{\fa{f}{x}} times, i.e.\ \frm{\fa{f}{x}} is an
expression of sort \frm{\srtnat}. Note that functions \frm{P} and \frm{f} have
to be total.

\subsection{Function types}

A function type of total functions from \frm{X_{0} \times \ldots \times X_{n}}
to \frm{Y} is declared by the sort expression \frm{X_{0} \times \ldots \times
X_{n} \To Y}. The following operations are provided for these sorts.

\bigskip
\begin{tabular}{|l@{\qquad}L@{\qquad}l|}
\hline
Operator                   & \textit{Rich}          & \verb+Plain+\\\hline
function application       & \_(\_,\ldots,\_)       & \verb+_(_,...,_)+\\
lambda abstraction         & \lambda \_{:}X_{0}, \ldots, \_{:}X_{n}.\_
                                          & \verb+lambda _:X0,...,_:Xn._+\\
\hline
\end{tabular}\bigskip

\noindent
Function types may be nested. To make this unambiguous, function arrow
\frm{\To} is right associative, which may be overridden by using parentheses.
Also it is not allowed to have sort expressions with \frm{\To} as its head at
the left hand side of a function type. If this is needed, parentheses need to
be used.


\subsection{Structured types}

Structured types consist of \emph{sum} types and \emph{product} types. A
structured type is declared by the following sort expression, where \frm{n \in
\pos} and \frm{k_{i} \in \nat} with \frm{1 \leq i \leq n}:
\begin{mCRL2}
& \kwstruct
    &c_{1}(\f{pr}_{1,1}: A_{1,1}, & \ldots & , \f{pr}_{1,k_{1}}: A_{1,k_{1}}) 
      ?\f{is\_c_{1}}\\
&\hfill |
    &c_{2}(\f{pr}_{2,1}: A_{2,1}, & \ldots & , \f{pr}_{2,k_{2}}: A_{2,k_{2}})
      ?\f{is\_c_{2}}\\
&                                 & \multicolumn{1}{c}{\vdots}\\
&\hfill |
    &c_{n}(\f{pr}_{n,1}: A_{n,1}, & \ldots & , \f{pr}_{n,k_{n}}: A_{n,k_{n}})
      ?\f{is\_c_{n}}
\end{mCRL2}

\noindent
From this declaration it can be seen that at least \frm{1} summation has to be
specified and a summation may consist of \frm{0} products. Each summation
\frm{i} is labelled by a \emph{constructor} \frm{c_{i}} and optionally by a
\emph{recogniser} \frm{\f{is\_c_{i}}}. Recogniser \frm{\f{is\_c_{i}}} is used
to determine if a term of the above sort is constructed with \frm{c_{i}}. If a
recogniser label is left out, the corresponding \frm{?} is also left out.  Each
product \frm{(i,j)} is optionally labelled by a \emph{projection}
\frm{\f{pr}_{i,j}}. With this projection, the \frm{j}'th element of summation
\frm{i} can be obtained. If a projection label is left out, the corresponding
\frm{:} is left out. If a summation \frm{i} does not have any products, it is
written as \frm{c_{i}?\f{is\_c_{i}}} instead of \frm{c_{i}()?\f{is\_c_{i}}}.  

All labels have to be chosen such that no ambiguity can arise. Each sort
\frm{A_{i,j}} has to be a valid sort expression, in which forward references to
sort labels may occur. This means that it is allowed to specify systems of
equations of structured types.

The following operations are generated for the above sort. Projection and
recogniser operations are only generated if the user specified them.

\bigskip
\begin{tabular}{|l@{\qquad}L@{\qquad}l|}
\hline
Operator                            & \textit{Rich}      & \verb+Plain+\\\hline
constructor of summation \frm{i}    & \fa{c_{i}}{\_,\ldots,\_} 
                                                         & \verb+ci(_,...,_)+\\
recogniser for constructor \frm{i}  & \fa{is\_c_{i}}{\_}    & \verb+is_ci(_)+\\
projection \frm{(i,j)}, if declared & \fa{\f{pr}_{i,j}}{\_}  & \verb+prij(_)+\\
\hline
\end{tabular}\bigskip

We give a few examples of sort declarations involving structured sorts. For
finite \frm{n \in \nat}, an enumerated type can be declared by
\begin{mCRL2}
sort & Enum & = & \kwstruct\ \f{enum}_{0}?\f{is\_enum}_{0}\ |\ \ldots\ |\
\f{enum}_{n}?\f{is\_enum}_{n}
\end{mCRL2}

\noindent
This generates constructor operations \frm{\f{enum}_{i} : \f{Enum}}, together
with recogniser operations \frm{\f{is\_enum}_{i}: \f{Enum} \To \srtbool}, with
\frm{0 \leq i \leq n}.

Pairs of elements of sort \frm{A} and \frm{B} can be declared as follows:
\begin{mCRL2}
sort & \f{ABPair} & = & \kwstruct\ \faa{pair}{\f{fst}: A}{\f{snd}: B}
\end{mCRL2}

\noindent
For this declaration, constructor operation \frm{\f{pair}: A \times B \To
\f{ABPair}} and projection operations \frm{\f{fst}: \f{ABPair} \To A} and
\frm{\f{snd}: \f{ABPair} \To B} are generated.

Binary trees where all leaves and nodes are labelled with elements of sort
\frm{A} can be declared as follows:
Example:
\begin{mCRL2}
sort & \f{BATree} & = 
  & \kwstruct\ \fa{leaf}{A} \  | \  \faaa{node}{BATree}{A}{BATree}
\end{mCRL2}
\noindent
Or, with projection and recogniser labels:
\begin{mCRL2}
sort & \f{BATree} & = 
  & \kwstruct & \fa{leaf}{\f{lval}: A}?\f{is\_leaf}\\
&&& \hfill |  & \faaa{node}{
    \f{left}: \f{BATree}}{\f{nval}: A}{\f{right}:\f{BATree}}?\f{is\_node}
\end{mCRL2}
\noindent
The quantification of an associative operation \frm{f: A \times A \To A} over
all labels in a \frm{\f{BATree}}, can be defined in the same way for both
declarations of the tree:
\begin{mCRL2}
map  & \f{qf}: \f{BATree} \To A\\
var  & t,u: \f{BATree}\\
     & a: A\\
eqn  & \fa{qf}{\fa{leaf}{a}}         & = & a\\
     & \fa{qf}{\faaa{node}{t}{a}{u}} & =
       & \faa{f}{\fa{qf}{t}}{\faa{f}{a}{\fa{qf}{u}}}\\
\end{mCRL2}
\noindent
The last definition of sort \frm{\f{BATree}} also allows the definition of
operation \frm{\f{qf}} without pattern matching:
\begin{mCRL2}
var  & \multicolumn{3}{L}{t: \f{BATree}}\\
eqn  & \fa{qf}{t} & = & \faaa{if}{\fa{is\_leaf}{t}}{\fa{lval}{t}}{
         \faa{f}{\fa{qf}{\fa{left}{t}}}{
           \faa{f}{\fa{nval}{t}}{\fa{qf}{\fa{right}{t}}}}}\\
\end{mCRL2}

\subsubsection{Pattern matching versus projections and recognisers}

When defining \emph{data equations}, it is often the case that pattern matching
on the constructors of a structured type is more elegant than using projection
and recogniser operations. The last example is a typical instance of this.

When defining \emph{process equations}, pattern matching can not be applied
directly on the left hand side of the equation. However, we can apply it
indirectly through the use of the summation and conditional operators in the
right hand side. As an example, take the following process declaration without
pattern matching:
\begin{mCRL2}
proc & \fa{P}{t: \f{BATree}} & = &
    \fa{is\_leaf}{t} & \To
    & \multicolumn{3}{@{}L}{\fa{get}{\fa{lval}{t}}.\delta\ +}\\
&&& \fa{is\_node}{t} & \To
    & \fa{get}{\fa{nval}{t}}.(\fa{P}{\fa{left}{t}} + \fa{P}{\fa{right}{t}})\\
\end{mCRL2}

\noindent
This declaration is equivalent to the following declaration, which uses pattern
matching:
\begin{mCRL2}
proc & \fa{P}{t: \f{BATree}} & = &
    \sum_{a:A}         
    & (t & == & \fa{leaf}{a}) \To \fa{get}{a}.\delta\ +\\
&&& \sum_{a:A, u,v: \f{BATree}}
    & (t & == & \faaa{node}{u}{a}{v}) \To \fa{get}{a}.(\fa{P}{u} + \fa{P}{v})\\
\end{mCRL2}

\noindent
As can be seen from the above example, it is arguable which way is the most
elegant to specify process equations.

\section{Parsing issues}

The following issues are important not only for parsing by computer, but also
for parsing by humans.

\subsection{Relation with processes}

\noindent
Data terms occur in relation to processes in the following ways:
\begin{tdefinitions}{-}
\item action parameters
\item arguments of a process reference
\item left argument of conditional process terms (\frm{b \To p,q})
\item right argument of a timed process term (\frm{p @ t})
\end{tdefinitions}

\noindent
These last two are ambiguous or hard to read for where clauses, quantifications
and infix operations. For this reason, these operations need to be
parenthesized.

\subsection{Type inference}

Data expressions involving numbers, sets, bags or lists may be ambiguous. E.g.\
namely \frm{1} can be parsed as a term of sort \frm{\srtpos}, \frm{\srtnat} or
\frm{\srtint}, and \frm{\set{}} can be parsed as a term of an arbitrary set or
bag sort. For overloaded operations, we have similar problems.

These problems are solved by a type inference system: if the type of an
expression can be determined unambiguously, the system is able to infer the
type of this expression.

\subsection{Priorities and associativity}

The prefix operators together with universal and existential quantification
have the highest priority, followed by the infix operators, followed by the
lambda operator, followed by the where clause. Table~\ref{tab:precedence} lists
the infix operators by decreasing priority. The symbols are shown in plain text
format and may represent multiple rich text symbols. Operators on the same line
have the same priority and associativity. Note that the list operations
\frm{\cons}, \frm{\snoc} and \frm{\concat} are split into three priority levels
such that expressions with one of these operations as their head symbol are
allowed if and only if they match the following pattern, where \frm{b, \ldots,
c}, \frm{d, \ldots, e} and \frm{s, \ldots, t} are expressions with a priority
level greater than \frm{\concat}:
\[
b \cons \dots \cons c \cons s \concat \dots \concat t \snoc d \snoc \dots
\snoc e
\]
 
\begin{table}[h!bt]
\centering
\begin{tabular}{|ll|}
\hline
operators                                           & associativity\\\hline
\verb+*+, \verb+div+, \verb+mod+, \verb+.+          & left\\
\verb-+-, \verb+-+                                  & left\\
\verb+|>+                                           & right\\
\verb+<|+                                           & left\\
\verb-++-                                           & left\\
\verb+<+, \verb+>+, \verb+<=+, \verb+>=+, \verb+in+ & none\\
\verb+==+, \verb+!=+                                & left\\
\verb+&&+, \verb+||+                                & left\\
\verb+=>+                                           & right\\\hline
\end{tabular}
\caption{Precedence of infix operators}
\label{tab:precedence}
\end{table}

\subsection{Design decisions}

In the development of the language, a number of design decisions were taken.
The most important ones are listed here:
\begin{tdefinitions}{-}
\item Layout may not have any effect on the semantics of the language.
\item Declarations have to be terminated by a semicolon. If this was not
required the language would be ambiguous. E.g.\ consider the following
operation declarations:
\begin{mCRL2}
& X   & = & \fa{f}{g}\\
& (k) & = & Y\\
\end{mCRL2}
From the layout it is clear that \frm{X} is equal to \frm{\fa{f}{g}} and
\frm{Y} to \frm{(k)}. However, since layout may not have any effect on the
semantics of the language, it is also possible that \frm{X} is equal to
\frm{f} and \frm{Y} to \frm{\fa{(g)}{k}}.
\end{tdefinitions}

\section {Comparison of data languages}

The data part of both the \mCRL and the mCRL2 languages have a lot in common
with functional programming languages. For the features that involve functional
programming languages the following comparison is made.

\begin{table}[h!bt]
\centering
\begin{tabular}{|l|llll|}
\hline
Aspect \verb+\+ Language&\mCRL        &mCRL2       &Haskell/Clean&MetaOCaml\\
\hline
Purely functional       &yes          &yes         &yes          &no\\
Expressiveness          &first-order  &higher-order&higher-order &higher-order\\
Strict                  &no           &no          &no           &yes\\
Evaluation              &somewhat lazy&somewhat lazy&lazy        &eager\\
Control of eval.\ order & no       &yes         &yes          &yes\\
Partial evaluation      &yes          &yes         &no           &yes\\
Polymorphism            &no           &no          &yes          &yes\\
Modules                 &no           &no          &yes          &yes\\
Object orientation      &no           &no          &no           &yes\\
Concrete data types     &no           &yes         &yes          &yes\\
\hline
\end{tabular}
\caption{Comparison of data languages}
\label{tab:comparison}
\end{table}

\noindent
Note that originally the evaluation in \mCRL and mCRL2 was eager, but the
addition of just-in-time strategies has moved towards laziness. Since both
languages have a non-strict semantics and evaluation is not fully lazy, the
evaluation only \emph{approximates} the semantics. However, this problem rarely
pops up in practise and if it pops up, it can usually be circumvented by
controlling the evaluation order. This can be achieved by the use of
\emph{conditional rewrite rules}.

\section{Future work}

\subsection{Extensions of data types}

\emph{Predefined data types} that could be desirable include the following:
\begin{tdefinitions}{-}
\item rational numbers (\frm{\mathbb{Q}})
\item real numbers (\frm{\mathbb{R}})
\item characters
\item strings
\end{tdefinitions}

\noindent
These data types will be provided when the need arises. However, for the
rational and real numbers, a suitable representation has not yet been found.

Desirable \emph{type constructors} include the following:
\begin{tdefinitions}{-}
\item natural numbers modulo \frm{n}, for finite \frm{n \in \nat}
\item (infinite) tables, arrays
\item stacks, queues
\end{tdefinitions}

\noindent
These type constructors will be provided when the need arises. Note that most
of these type constructors can be easily constructed using function types and
structured types. However, for reasons of standardisation and better syntax, it
might be important to add these to the language.

\subsection{Miscellaneous}

Based on the remarks by Jan Friso Groote and Jaco van de Pol, these issues are
still open:
\begin{tdefinitions}{-}
\item It should investigated if the current bag notation is appropriate.
\item It should be possible to use unnamed product types in sort expressions.
\item Instead of having sort conversion operations, it should be possible to
explicitly state the sort of a data expression, e.g.\ \frm{1: \f{Nat}}
\item The \emph{if-then-else} construct should be in mixfix notation instead of
function notation.
\end{tdefinitions}

\end{document}
