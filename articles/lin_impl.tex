\documentclass{article}
\usepackage{textcomp,amsmath}
\usepackage{mcrl,mymath}


\title{A rewriting-strategies-based tool for transforming process-algebraic equations}
\author{Sergey V. Goncharov\inst{1}\and Arseniy Y. Rudich\inst{1}\and Yaroslav S. Usenko\inst{2}}

\institute{
  Department of Theoretical Cybernetics, \\
  National Taras Shevchenko University of Kyiv, Ukraine 
  \and
  Laboratory for Quality Software, \\
  Technical University of Eindhoven, The Netherlands }

\newcommand{\stratego}{\textsc{Stratego}}

\begin{document}

\maketitle

\section{Introduction}

In this abstract we describe an attempt to implement a
\emph{linearization} algorithm~\cite{Use02} for micro Common
Representation Language (\mcrl)~\cite{GP94a}.

This algorithm allows transforming specifications in \mcrl\ to
``simpler'' ones, containing just one Linear Process Equation (LPE).
For the implementation we use the program-transformation system
\stratego~\cite{Vis01} that is based on \emph{rewriting strategies}.
\stratego\ allows to specify complex term-rewriting systems and their
application strategies, and to generate efficient C code from them,
which can be further compiled and executed. The implementation has
been tested on a set of examples of \mcrl\ specifications, and the
results were compared with an existing implementation of \mcrl\ 
linearization in C~\cite{Wou01}.

\mcrl\ is an algebraic specification language based of ACP style
process algebra~\cite{BK84b}. A distinct feature of \mcrl\ in
comparison with many other process algebras is that it offers a
uniform equational framework for specification of data and processes.
Data are specified by equational specifications: one can declare sorts
and functions on these sorts, and describe the meaning of these
functions by equational axioms. Processes are described using
data-parametric systems of equations that describe process behavior
using process-algebraic operations extended with constructs for
conditional composition and data-parametric choice.

Linear Process Equations (LPE) is an interesting subclass of systems
of recursive equations, which contain only one linear equation of
special form. As it turns out, the restriction to LPE format still
yields an expressive setting. For instance, in the design and
construction of verification tools for \mcrl, LPEs establish a basic
and convenient format that can be seen as a symbolic representation of
Labeled Transition Systems (LTSs).

The algorithms for reducing \mcrl\ specifications to the linear form
are described in details in~\cite{Use02}. Each particular algorithm
consists of a chain of transformation steps that yield an equivalent
\mcrl\ specification with a form coming closer to the desired linear
one. Each transformation step is, in essence, an equivalent
transformation of a system of process equations.  The steps are
formally described using term rewriting systems and different forms of
their extensions to equation rewriting systems. In some steps extra
equations or data types are added to the original \mcrl\ 
specification.

For the implementation of the linearization algorithm we chose
\stratego~\cite{Vis01}. \stratego is a system for the specification of
fully automatic program transformation systems based on the rewriting
strategies paradigm. Its language, aimed to operate with terms, is
based on rewriting concepts~\cite{BN99}. It has clear and well
thought-out syntax and semantics, and a very efficient implementation.
\stratego\ is based on ATerm library~\cite{BJKO00} for term
representation, which supports maximal sharing and efficient automatic
garbage collection. Parsing of input programs to ATerms is done using
the SDF toolset~\cite{Vis97}, which generates parsers for an extension
of context-free grammars. All that makes writing clear, manageable and
efficient programs transformation programs easy, and therefore
motivates our choice of the implementation framework, in comparison
with implementation in plain C.

\section{Details of Implementation}
The basic concept of \stratego\ is a notion of rewriting strategy that
is a rule specifying how to apply given term rewriting system to a
term. Every program in \stratego\ contains the main strategy that, in
fact, transforms the input term into the output one.  This main
strategy is normally defined as a composition (nondeterministic
choice, sequential composition, congruent closure and others) of
simpler strategies.

For each linearization step described in~\cite{Use02}, we generate a
separate rewriting step (rewriting strategy), and then the main
strategy is, in essence, a sequential composition of these steps.

\section{Associative-Commutative Rewriting}
In SDF grammar it is possible to specify that certain operations are
commutative, associative, and/or idempotent; and the generated parser
will use lists, bags, or sets to represent the groups of operations.
For example, in \mcrl, alternative composition (+) is an associative,
commutative and idempotent operation, and sequential composition (.)
is an associative operation (there are more of such operations). In
the parse tree these operations will be represented by sets and lists,
respectively. However, to perform term rewriting module AC, we need to
adopt the rewrite rules to be able to deal with list and sets. As an
example, if the following rewrite rules
\begin{align*}
\act{a}+\act{b}\seqc \act{c}&\to \act{a}\seqc\act{c}+\act{b}\seqc\act{c}\\
\act{a}+0&\to \act{a}\\
0\seqc \act{a}&\to 0
\end{align*}
have to be applied modulo associativity and commutativity, then the
corresponding list operations will have the following form:
\begin{gather*}
\seq{\act{a}_1, \act{a}_2,\dots, \act{a}_n, [\act{b}_1, \act{b}_2,\dots,\act{b}_k],\act{a}_{n+2},\dots, \act{a}_m} \to\\
\t2\seq{\act{a}_1, \act{a}_2, \dots, \act{a}_n, [\seq{\act{b}_1, \act{a}_n,\dots,\act{a}_m},\seq{\act{b}_2,\act{a}_n,\dots,\act{a}_m},\dots,\seq{\act{b}_k,\act{a}_n,\dots,\act{a}_m}]}\\
[\act{a}_1, \act{a}_2,\dots,\act{a}_n, 0,\act{a}_{n+2},\dots,\act{a}_m] \to 
[\act{a}_1, \act{a}_2, \dots, \act{a}_m]\\
\seq{\act{a}_1, \act{a}_2,\dots, \act{a}_n, 0,\act{a}_{n+2},\dots,\act{a}_m} \to 
\seq{\act{a}_1, \act{a}_2, \dots \act{a}_n}
\end{gather*}

A procedure to obtain such lifting rules is rather straightforward.
Whenever we have a binary associative operation in the left-hand-side
of a rewrite rule, we consider its extension to $n$ terms, not just
$2$. Such a transformation was sufficient for the case of our
implementation; however it is not clear whether it can be applied in
more general cases.

\section{Efficient Implementation of Simple Rewriting}
One of the steps in linearization procedure consists of an application
of a set of simple term rewrite rules. It turns out that all terms in
the left-hand-sides of these rules have \emph{depth} not bigger than
2. Experiment shows that in this case the standard \emph{innermost}
rewriting strategy is suboptimal.  To improve on this, a custom
strategy, called \emph{double-traversal}, is used instead.

To use this strategy we split the set of rewrite rules in two. The
rules from one set are used when the rewriting process traverses the
term \emph{down}, and the other set is used when the process traverses
\emph{up}. Initially, the process traverses down from the top (as in
the outermost strategy) and once a rule is applied, the process
traverses one step up, and repeats itself recursively. The process
stops, when it reaches all bottom (leaf) nodes and no rule can be
applied any longer.

An interesting question is whether this strategy can be used with a
wider class of the rewrite systems.

\section{Experimental Results and Future Work}
At this point most of the linearization steps from~\cite{Use02} have
been implemented. Certain optimization steps, like ``regular
linearization'' and ``operations on LPEs'', have still to be
implemented. We tested our linearization procedure on the examples
available in the \mcrl\ Toolset distribution~\cite{Wou01}.

A good sign is that both the implementation in the \mcrl\ Toolset
distribution and the ours produce LPEs that lead to \emph{bisimilar}
transition systems. This was checked automatically using the
bisimilarity checker from the \mcrl\ Toolset. Another good sign is
that the performance of the new implementation is approximately the
same as of the existing one. However, we have still to try out more
realistic examples to compare the performances of the two
implementations.

The number of generated states and transitions from the LPEs produced
by our implementation is sometimes substantially larger than in the
original implementation. This has still to be investigated.  It can be
due to the fact that not all optimization steps have been implemented
yet.

In the future we plan to implement all optimization steps that are
described in the literature. It is also interesting to extend the
implementation to cover the case of timed \mcrl.  Trying larger
examples may lead to a need for optimization of the rewriting
procedures.

\paragraph{Acknowledgments}
The initial idea of the implementation is due to Jaco van de Pol and
Eelco Visser back in summer of 2001. Eelco has also helped us with
\stratego. An earlier attempt to implement the linearization algorithm
in \stratego was done by Konstantin O.  Savenkov in~\cite{Sav02}.

\bibliographystyle{plain}
\bibliography{lin}
\end{document}
